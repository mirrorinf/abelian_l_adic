\chapter{结论}

本文中详细地介绍了两个主定理\ref{main::open_image},\ref{main::surjective}的证明.
这两个定理及其证明过程中用到的方法都非常具有启发性.

从对椭圆曲线的认识来说,
可以看到,就$\rho_{\ell}$的像的大小,或者说Tate模中各个元素的“独立性”来说,复乘是一个关键的因素.
一方面,不带复乘的椭圆曲线比带复乘的椭圆曲线复杂得多;
另一方面,不同的带复乘的椭圆曲线之间依照$\mathrm{End}(E)\otimes \mathbb{Q}$的不同各有千秋,
而所有不带复乘的椭圆曲线却出乎意料地都是类似的.
% 某种意义上来说,带复乘的椭圆曲线各有各的简单之处,而不带复乘的椭圆曲线都是相似的复杂.
同时,复乘现象能够很好地被$\rho_{\ell}$的交换性来刻画,或者也可以被Serre群$S_{\mathfrak{m}}$刻画.
这启发我们利用Galois表示和Serre群这些工具来研究高维Abel簇的复乘现象.

从对绝对Galois群的认识来说,
通过取不同的不带复乘的椭圆曲线$E$和不同的素数$\ell$,可以得到一系列满射
$\mathrm{Gal}(\overline{K}/K) \to \mathrm{GL}_2(\mathbb{Z}_{\ell})$.
这就给出了很多Galois群是$\mathrm{GL}_2(\mathbb{Z}_{\ell})$的扩张.
(事实上,由Faltings的同源定理,椭圆曲线定义的$\ell$进表示有无穷多个同构类)
绝对Galois群的复杂程度可见一斑.
进一步地,如果有其它种类的Galois表示,也可以对表示的像的大小提问,甚至可以借鉴本文中用到的一些方法来进行研究.
