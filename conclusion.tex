\chapter{结论}

从对椭圆曲线的认识来说,
可以看到,就$\rho_{\ell}$的像的大小,或者说Tate模中各个元素的“独立性”来说,
带复乘的椭圆曲线和不带复乘的椭圆曲线有本质的差别,而所有不带复乘的椭圆曲线都是类似的.
复乘现象能够很好地被$\rho_{\ell}$的交换性来刻画.

从对绝对Galois群的认识来说,
通过取不同的不带复乘的椭圆曲线$E$和不同的素数$\ell$,可以得到一系列满射
$\mathrm{Gal}(\overline{K}/K) \to \mathrm{GL}_2(\mathbb{Z}_{\ell})$.
这就给出了很多Galois群是$\mathrm{GL}_2(\mathbb{Z}_{\ell})$的扩张.
(事实上,由Faltings的同源定理和Shafarevich的定理,椭圆曲线定义的$\ell$进表示有无穷多个同构类)


