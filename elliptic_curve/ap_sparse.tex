\subsection{\texorpdfstring{$\mathrm{Tr}(F_{v, \rho})=0$}{tr F v rho}的稀疏性}

先假设$\rho$是任意的有理$\ell$进表示,其像是$\mathrm{GL}_n(\mathbb{Z}_{\ell})$的开子群.

\begin{cprop}
    令$S_0$为满足以下条件的素点$v$的集合:

    (i) $\rho$在$v$处非分歧,且$F_{v,\rho}$的特征多项式的是有理系数的;

    (ii) $\mathrm{Tr}(F_{v,\rho}) = 0$.

    则$S_0$的密度为$0$.
\end{cprop}

\begin{proof}
    令$H = \{s\in \mathrm{GL}_n(\mathbb{Z}_{\ell})\mid \mathrm{Tr}(s)=0\}$,
    $H' = \mathrm{Im}\ \rho \bigcap H$.
    由于$H'$的维数严格小于$n^2$,$H'$的Haar测度为$0$,而$\mathrm{Im}\ \rho$的Haar测度不为$0$.

    记$H'_n$为$H$在$\mathbb{GL}_2(\mathbb{Z}/\ell^n \mathbb{Z})$上的投影,
    $G_n$为$\mathrm{Im}\ \rho$相应的投影. 那么$\lim\limits_{n\to \infty} \frac{\absn{H'_n}}{\absn{G_n}} = 0$.
    由Chebotarev密度定理,$S_0$的密度不大于每个$\frac{\absn{H'_n}}{\absn{G_n}}$,
    即$S_0$的密度为$0$.
\end{proof}

特别地,如果$\rho$是椭圆曲线$E$定义的表示,且$E$在$v$处有好约化.
则$\mathrm{Tr}(F_{v,\rho})=0$就等价于$E$在$v$处的约化高度为$2$,或者说,$\tilde{E}[p_v] = 0$.
因此如果证明了没有复乘的椭圆曲线的$\rho_{\ell}$的像是开子群,就有了
\begin{ccor}
    设$E$是没有复乘的椭圆曲线,则$E$有高度为$2$的好约化的素点$v$的密度为$0$.\label{height2::sparse}
\end{ccor}
