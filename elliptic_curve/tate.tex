\section{Tate模的基本性质}

令$K$是代数数域.
引言中定义了椭圆曲线的Tate模.
现在来证明,Tate模定义了一族严格相容的有理表示. 事实上,Frobenius元素的特征多项式的系数有很好的几何解释.

用$\rho_{\ell}$记$E$的$\ell$进Tate模定义的表示$\mathrm{Gal}(\overline{K}/K)\to \mathrm{GL}_2(\mathbb{Z}_{\ell})\subset \mathrm{GL}_2(\mathbb{Q}_{\ell})$.

设$v\in \Sigma_K$,$p_v\neq \ell$且$E$在$v$处有好约化. 选定$\overline{K}$的素点$w\mid v$.
记$k\cong \mathbb{F}_q$是$v$的剩余类域,则$w$的剩余类域同构于$\overline{k}$.
此时$E$模$\mathfrak{p}_v$的约化$\tilde{E}$仍然是椭圆曲线.

$\mathrm{mod}\ \mathfrak{p}_{w}$
诱导了单同态$E[\ell^n]\to \tilde{E}[\ell^n]$,
而且$\tilde{E}[\ell^n]\cong (\mathbb{Z}/\ell^n \mathbb{Z})^2$,
因此$\mathrm{mod}\ \mathfrak{p}_{w}$诱导了群同构$E[\ell^n]\to \tilde{E}[\ell^n]$.
分解子群$D=D_{w/v}$的作用保持赋值,因此$\mathrm{mod}\ \mathfrak{p}_{w}$与$D$的作用交换.

那么$\mathrm{mod}\ \mathfrak{p}_{w}$诱导了
保持$D$作用的同构$T_{\ell}(E)\cong T_{\ell}(\tilde{E})$.
但是$D$在$T_{\ell}[\tilde{E}]$上的作用可以通过$\mathrm{Gal}(\overline{k}/k)$作用,
那么$I=I_{w/v}$的作用的平凡的.

这就证明了$I$在$\rho_{\ell}$中的像是$1$,即$\rho_{\ell}$在$v$处非分歧.
记$F_{v,\rho}\in \mathrm{Im}\ \rho_{\ell}$为Frobenius元素.
$F_{v,\rho}$在剩余类域$k$上通过Frobenius自同构作用,那么$F_{v,\rho}$在$\tilde{E}$上通过Frobenius自同构$F_v$作用.

如果$\varphi\in\mathrm{End}_{k}(\tilde{E}, \tilde{E})$,则$\varphi$定义了Tate模到自身的线性映射$\varphi_{\ell}$.
利用Weil配对可以计算出$\det \varphi_{\ell} = \mathrm{deg}\ \varphi$.
(计算过程是很直接的,见{\parencite[][p. 99]{silverman2009arithmetic}})

因此$\det F_{v,\rho} = \deg F_v = q$,
$\det (1-F_{v,\rho}) = \deg (1-F_v) = 1 - \mathrm{Tr}(F_{v,\rho}) + q$.
于是$\mathrm{Tr}(F_{v,\rho}) = \deg F_v - \deg (1-F_v) + 1$,$\det F_{v,\rho} = q$都是与$\ell$无关的整数.

也就是说,当$E$在$v$处有好约化且$p_v\neq \ell$时,$\rho_{\ell}$在$v$处非分歧
且$F_{v,\rho}$的特征多项式是与$\ell$无关的整系数多项式.
因此$\{\rho_{\ell}\}$是一族严格相容的有理表示,例外集合$S$可以取为$E$有坏约化的素点集合.

再来考虑$\tilde{E}$中的$p_v$阶元.
称$E$在$v$处的约化有高度$1$是在说$\tilde{E}[p_v]$是$p_v$阶循环群,而高度为$2$是在说$\tilde{E}[p_v] = 0$.
关于约化的高度有一般的定理
\begin{cthm}[{\parencite[][p. 144]{silverman2009arithmetic}}, Theorem V.3.1]
    设$k$是特征为$p$的域,$E$是$k$上的椭圆曲线. 记$\phi_r: E\to E^{(p^r)}$为$p^r$次Frobenius同源\\
    (i) 以下各命题等价:\\
    \quad (1) 对某个(进而是所有的)$r\geq 1$,$E[p^r] = 0$;\\
    \quad (2) 对某个(进而是所有的)$r\geq 1$,$\hat{\phi_r}$不可分(进而是完全不可分的);\\
    \quad (3) $E$上的$p$倍映射完全不可分,且$j(E)\in \mathbb{F}_{p^2}$;\\
    \quad (4) $\mathrm{End}(E)$是一个四元数代数中的子环;\\
    \quad (5) $E$诱导的形式群有高度$2$.\\
    (ii) 如果以上的等价条件不成立,那么对所有的$r\geq 1$都有$E[p^r] \cong \mathbb{Z}/p^r\mathbb{Z}$,而且$E$诱导的形式群有高度$1$.
\end{cthm}
也就是说,上面所说的好约化的“高度”和$E$诱导的形式群的高度是吻合的.

记$\hat{F}_v$为$F_v$的对偶同源,$\hat{F}_{v,\rho}$为$\hat{F}_v$在$T_{\ell}$上的作用.
由$\mathrm{Tr}(F_v) = 1 - \deg(1-F_v)+\deg F_v = 1 - (1 - F_v)(1-\hat{F}_v) + F_v\hat{F}_v$
知$\hat{F}_v = \mathrm{Tr}(F_v) - F_v$.
而$E$在$v$处有高度$2$的好约化等价于$\hat{F}_v$不可分,也就等价于$p_v \mid \mathrm{Tr}(F_v)$.

%另一方面,注意到$\mathrm{Tr}(F_v) = \deg F_v - \deg(1-F_v) + 1 = q+1-\absn{E(k)}$.
%由Hasse的估计,$\absn{\mathrm{Tr}(F_v)} \leq 2\sqrt{q}$.
%那么当$p_v\geq 5$时,$p_v\mid \mathrm{Tr}(F_v)$当且仅当$\mathrm{Tr}(F_v)=0$.

%这也就是说,当$p_v\geq 5$是,$E$在$v$处的好约化高度为$2$当且仅当$\mathrm{Tr}(F_v) = 0$,也当且仅当
%$\absn{E(k)} = q + 1$.

%接下来证明高度为$2$的好约化是非常稀疏的.
