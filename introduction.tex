\chapter{引言}

本文中的讨论主要参考\parencite{serre1997abelian}和\parencite{serre1981quelques}.

志村五郎在\parencite{shimura1966reciprocity}中讨论了由椭圆曲线$E$的$\ell$阶点生成的数域$L/\mathbb{Q}$,
其中$\ell$是一个素数.
志村对$E=X_0(11)$和$7\leq \ell\leq 97$证明了
是$\mathrm{Gal}(L/\mathbb{Q})\cong \mathrm{GL}_2(\mathbb{F}_{\ell})$.

同时由熟知的结果,Frobenius元素的特征多项式能够由椭圆曲线的几何性质完全决定(本文将在第三章中证明这些结果).
对$\tau\in\mathrm{GL}_2(\mathbb{F}_{\ell})$中,特征多项式能几乎确定其共轭类:
如果特征多项式没有重根,那么$\tau$共轭于$\matbt{a}{0}{0}{b}$,其中$a,b$是特征多项式的两个根
(如果$a,b\in \mathbb{F}_{\ell^2} - \mathbb{F}_{\ell}$,则需要再共轭到一个$\matbt{c}{\sigma d}{d}{c}$,其中$\sigma\in\mathbb{F}_{\ell}^{\times} - (\mathbb{F}_{\ell}^{\times})^2;c, d\in\mathbb{F}_{\ell}$);
而如果特征多项式有重根$a$,则$\tau$或者共轭于$\matbt{a}{1}{0}{a}$,或者就是$\matbt{a}{0}{0}{a}$.

那么,假设$\mathrm{Gal}(L/\mathbb{Q})\cong \mathrm{GL}_2(\mathbb{F}_{\ell})$成立,
则素数在$L$中的分解情况几乎可以由椭圆曲线的几何性质决定.
于是我们可以构造一大类扩张,其Galois群是有限李型群,而且可以几乎确定素数在其中的分解.

同时,$X_0(11)$是模曲线,其几何性质被编码在一个模形式
\begin{equation}
    f(z) = q\prod_{n=1}^{\infty} \paten{1-q^n}^2 \paten{1 - q^{11n}}^2,q=\exp(2\pi i z)
\end{equation}
的Fourier系数中. 那么,$L$中素数的分解性质就(几乎,还差确定幂零共轭类)被编码在了一个模形式中.
这一事实可以被看作是某种非交换的类域论. 如果不是取椭圆曲线的有限阶点,
而是取乘法群$\mathbb{G}_m$的有限阶点,即单位根,
就得到了经典的类域论.

要完全实现上面的想法,需要解决以下的问题:
\begin{enumerate}
    \item 什么时候$L/\mathbb{Q}$的Galois群是$\mathrm{GL}_2(\mathbb{F}_{\ell})$?
    \item 以上讨论能否推广到一般的基域$K$?
    \item 什么时候椭圆曲线$E$是模曲线?
\end{enumerate}
问题3就是模性问题. 问题2的回答是肯定的,问题1和椭圆曲线的复乘有关,我们下面进行详细的分析.

在开始之前,还可以提出几个问题:
\begin{enumerate}
    \item 如果添加上$E$的合数阶点会发生什么?
    \item 在Frobenius元素的特征多项式有重根时,如何确定它是不是幂零的?
    \item 对高维的Abel簇是否存在类似的理论?
    \item 更一般地,能否构造Galois群是$G(\mathbb{F}_{q})$的扩张,其中$G$是Chevalley群,$q$是素数的幂,并确定其中的Frobenius元素?
\end{enumerate}

问题1是比较简单的,添加合数阶点就是添加所有的素数幂阶点,我们下面分析时会讨论素数幂阶点的情况.
问题2-4比较复杂,超出了本文讨论的范围.

\vskip0.3cm

现在开始正式的讨论.
添加椭圆曲线的有限阶点得到的扩张的Galois群就是绝对Galois群在有限阶点上的作用的像,
接下来我们采用更方便的Galois表示的语言.

设$E$是定义在域$K$上的椭圆曲线,$\ell$是素数且$\ell\neq \mathrm{char}\ K$. 对所有的正整数$n$,$E[\ell^n]$是$E(\overline{K})$的有限子群,且同构于$(\mathbb{Z}/\ell^n\mathbb{Z})^2$. $G = \mathrm{Gal}(\overline{K}/K)$作用在$E[\ell^n]$上,且作用与$E$上的加法交换,即有表示$\rho_n:G\to \mathrm{GL}_2(\mathbb{Z}/\ell^n \mathbb{Z})$.

记$K_n=K(E[\ell^n])$. $\sigma\in G$在$K_n$上作用平凡当且仅当$\sigma$在$E[\ell^n]$上的作用平凡,因此$K_n/K$是Galois扩张,且$\mathrm{Gal}(K_n/K)=\mathrm{Im}\ \rho_n$. $\mathrm{Im}\ \rho_n$的大小反映了$K_n$的复杂程度,也就是$E$的有限阶点之间相互独立的程度.

可以将所有的$\rho_n$放在一起考虑. 记$E[\ell^{\infty}]=\bigcup_n E[\ell^n]$,$K_{\infty} = K(E[\ell^{\infty}])$. $K_{\infty}$也是Galois扩张,且$\mathrm{Gal}(K_{\infty}/K) = \lim\limits_{\longleftarrow} \mathrm{Gal}(K_n/K)$. 记$\rho = \lim\limits_{\longleftarrow} \rho_n$,$\rho: G\to \mathrm{GL}_2(\mathbb{Z}_{\ell})$,其中转移映射$\mathrm{GL}_2(\mathbb{Z}/\ell^{n+1} \mathbb{Z}) \to \mathbb{Z}/\ell^n \mathbb{Z}$,$\tau \mapsto \tau \pmod{\ell^n}$.

Tate提出了另一种将$\rho_n$放在一起的方式. 令Tate模$T_{\ell}(E) = \lim\limits_{\longleftarrow} E[\ell^n]$,其中转移映射为$E[\ell^{n+1}]\to E[\ell^n], P\mapsto \ell P$,则$T_{\ell}\cong (\mathbb{Z}_{\ell})^2$. $G$在$E[\ell^n]$上的作用还与椭圆曲线上的加法交换,因此在同构$T_{\ell}(E)\cong (\mathbb{Z}_{\ell})^2$下$G$的作用是线性映射. 这就得到了表示$G\mapsto \mathrm{GL}_2(\mathbb{Z}_{\ell})$. 可以看出Tate模定义的表示在模每个$\ell^n$之后都成为$\rho_n$,因此就是上一段中定义的$\rho$.
记$V_{\ell} = T_{\ell}\otimes \mathbb{Q}_{\ell}$.

在$\mathrm{char}\ K = 0$,且$E$具有复乘的情况下,$\mathrm{Im}\ \rho$是相对简单的. 通过将$K$取成一个有限扩张,不妨设$\mathrm{End}_K(E) \cong \mathcal{O}$. 此时$T_{\ell}$是$\mathcal{O}_{\ell} = \lim\limits_{\longleftarrow}\mathcal{O}/\ell^n \mathcal{O}$上秩为$1$的模,$G$的作用与$\mathcal{O}_{\ell}$的作用交换. 此时$\rho$实际上是表示$G\to (\mathcal{O}_{\ell})^{\times}$,即$\mathrm{Im}\ \rho$是交换群.

如果$K$是代数数域,绝对Galois群的表示的像似乎遵从一个原则,“如果没有让像变小的理由,那么它就会尽可能地大”. $E$具有复乘就是一个让$\rho$的像变小的原因. Serre证明了复乘就是仅有的让像变小的原因,也就是本文的第一个主定理:

\begin{cthm}
    令$K$是代数数域,$E$是定义在$K$上的椭圆曲线. 假设$\mathrm{End}_{\overline{K}}(E) \cong \mathbb{Z}$. 则$\mathrm{Im}\ \rho$是$\mathrm{GL}_2(\mathbb{Z}_{\ell})$的开子群. \label{main::open_image}
\end{cthm}

定理\ref{main::open_image}的条件和结论都可以差一个有限扩张.
但既然证明了像是开子群,有一个自然的问题就是它是否是整个$\mathrm{GL}_2(\mathbb{Z}_{\ell})$.
这当然是不一定的. 就算$\rho_{\ell}$的像是$\mathrm{GL}_2(\mathbb{Z}_{\ell})$,
只要取一个真开子群$G'\subset \mathrm{GL}_2(\mathbb{Z}_{\ell})$,
并令$K' = K(E[\ell^{\infty}])^{G'}$,则$K'/K$是有限扩张且定义在$K'$上的$\ell$进表示
$\rho_{\ell}'$的像就不是$\mathrm{GL}_2(\mathbb{Z}_{\ell})$了.

但无论如何,还可以继续追问什么时候$\rho_{\ell}$的像是整个$\mathrm{GL}_2(\mathbb{Z}_{\ell})$.
首先,如果$\rho_{\ell}$的像是$\mathrm{GL}_2(\mathbb{Z}_{\ell})$,
则$\tilde{\rho}_{\ell} = \rho_{\ell} \pmod{\ell}$也是满射.
因此可以先问什么时候$\tilde{\rho}$是满射. 这就是第二个主定理

\begin{cthm}
    $K, E$同上. 对于除了有限多个以外的素数$\ell$,模$\ell$\quad Galois表示$\tilde{\rho}_{\ell}: \mathrm{Gal}(\overline{K}/K)\to \mathrm{GL}_2(\mathbb{F}_{\ell})$是满射. \label{main::surjective}
\end{cthm}

事实上,由几乎所有的$\tilde{\rho}_{\ell}$是满射,可以推出几乎所有的$\rho_{\ell}$都是满射.

两个主定理的证明思路是类似的,就是分类出$\mathrm{GL}_2$的子群,再一个一个排除. 这里$\mathrm{GL}_2$的模糊表述是故意的,同时指$\mathrm{GL}_2(\mathbb{Q}_{\ell})$和$\mathrm{GL}_2(\mathbb{F}_{\ell})$. 两者一个通过李代数的方法分类,另一个是有限群的子群的分类.

而另一方面,排除一个子群的可能性就需要控制$\rho_{\ell}$的行为. 具体来说,在$\ell$变化时,$\rho_{\ell}$遵守很强的相容性条件,$\rho_{\ell}$的性质可以通过分析其它$\rho_{\ell'}$得到,也就是说可以随时找一个方便的$\ell$在研究$\rho_{\ell}$的性质. 进一步地,相容性条件还允许结合无穷多个$\rho_{\ell}$的力量,反过来对椭圆曲线做出限制. 事实上,本文中最重要的定理可能是

\begin{cthm}
    如果$\rho_{\ell}$的像是交换的,则$E$在$K$上有复乘.
\end{cthm}

在证明以上三个定理的过程中,Serre发现了一类代数群$S_{\mathfrak{m}}$. 所有带复乘的椭圆曲线定义的$\rho_{\ell}$实际上都是从$S_{\mathfrak{m}}$的代数表示得到的. 而且逆命题也是对的,即如果$\rho_{\ell}$是从$S_{\mathfrak{m}}$的代数表示得到的,则$E$有复乘. 或者说$S_{\mathfrak{m}}$就刻画了复乘现象.

有了定理\ref{main::surjective},自然就可以进一步地问例外的素数可以有多少个.
记$\ell_{E}$是最大的使得$\tilde{\rho}_{\ell}$不是满射的素数.
$\ell_{E}$的大小可以作为例外的素数集合有多大的一种表现.
Serre在\parencite{serre1981quelques}中利用素数定理对$\ell_{E}$的大小进行了估计,并在GRH下证明了
\begin{cthm}
    如果$E/\mathbb{Q}$没有复乘,记$N_E$是所有$E$有坏约化的素数的乘积. 那么存在与$E$无关的常数$c$使得当
    \begin{equation}
        \ell \geq c(\log N_E)(\log\log 2N_E)^3
    \end{equation}
    时,$\tilde{\rho}_{\ell}$都是满射. 或者说,
    \begin{equation}
        \ell_{E} \leq c(\log N_E)(\log \log 2N_E)^3
    \end{equation}
\end{cthm}

猜测$\ell_{E}$仅与基域$K$有关,且$\ell_{\mathbb{Q}}=37$.
这被称为Serre的一致性猜想(Serre's uniformity conjecture).

本文其余部分的组织如下:
第二章定义$\ell$进Galois表示和严格相容的有理表示族的概念,
并证明一族严格相容的有理表示只要是交换的,就一定可以被某个代数群$S_{\mathfrak{m}}$刻画;
第三章证明一些椭圆曲线的Tate模的结论;
第四章证明第一个主定理,即无复乘时,$\ell$进表示$\rho_{\ell}$的像是开子群;
第五章证明第二个主定理,即无复乘时,对几乎所有的素数$\ell$,模$\ell$表示$\tilde{\rho}_{\ell}$是满射;
第六章不加证明地介绍一些对$\ell_{E}$的估计的结果.
