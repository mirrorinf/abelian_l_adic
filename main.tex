\documentclass[a4paper, UTF8, CJKmath=true]{ctexart}

\usepackage{mathtools}
\usepackage{graphicx}
\usepackage{amsthm}
\usepackage{mathrsfs}
\usepackage{amsfonts}
\usepackage{amssymb}
\usepackage{hyperref}
\usepackage{tikz-cd}
\usepackage{float}

\usepackage{xcolor}

\usepackage[left=2.5cm,right=2.5cm,top=3cm,bottom=3cm]{geometry}

\setmainfont[
    BoldFont={HYXuanSong-75S},
]{HYXuanSong}
\setCJKmainfont[
    BoldFont={HYXuanSong-75S},
]{HYXuanSong}

\usepackage[backend=biber]{biblatex}
\addbibresource[location=local]{main.bib}

\theoremstyle{definition}

\newtheorem{cthm}{定理}
\newtheorem{cprop}{命题}
\newtheorem{crem}{注记}
\newtheorem{cdef}{定义}
\newtheorem{ccor}{推论}
\newtheorem{clem}{引理}
\newtheorem{cyyy}{白日梦}

\newenvironment{pppmatrix}{\left(\begin{matrix}}{\end{matrix}\right)}

\newcommand{\paten}[1]{\left(#1\right)}
\newcommand{\deri}{\mathrm{d}}
\newcommand{\I}{\mathrm{i}}
\newcommand{\modular}{\mathcal{M}}
\newcommand{\cusp}{\mathcal{S}}
\newcommand{\diam}[1]{\big<#1\big>}
\newcommand{\matbt}[4]{\begin{pppmatrix}#1 &#2\\#3 &#4\end{pppmatrix}}
\newcommand{\absn}[1]{\left|#1\right|}
\newcommand{\order}{\mathcal{O}}
\newcommand{\Chebotarev}{Chebotarev}
\newcommand{\modulus}{理想}

\newcommand{\invlim}{\lim\limits_{\longleftarrow}}

%\setlength{\parindent}{0em}

\title{椭圆曲线定义的Galois表示的像}
\author{褚晓敏}
\date{\today}

\begin{document}
    \maketitle

    \begin{abstract}
        本文关于椭圆曲线的挠点定义的Galois表示的像的综述.
    \end{abstract}

    \tableofcontents

    \section{引言}

绝对Galois群是数论中最引人注目的对象之一,而对它的研究通常是通过研究它的表示进行的.
通过平展上同调群可以定义一类Galois表示。其中最简单的情形是考虑Galois群在椭圆曲线的挠点上的作用:

设$E$是定义在域$K$上的椭圆曲线,$\ell$是素数且$\ell\neq \mathrm{char}\ K$. 对所有的正整数$n$,$E[\ell^n]$是$E(\overline{K})$的有限子群,且同构于$(\mathbb{Z}/\ell^n\mathbb{Z})^2$. $G = \mathrm{Gal}(\overline{K}/K)$作用在$E[\ell^n]$上,且作用与$E$上的加法交换,即有表示$\rho_n:G\to \mathrm{GL}_2(\mathbb{Z}/\ell^n \mathbb{Z})$.

记$K_n=K(E[\ell^n])$. $\sigma\in G$在$K_n$上作用平凡当且仅当$\sigma$在$E[\ell^n]$上的作用平凡,因此$K_n/K$是Galois扩张,且$\mathrm{Gal}(K_n/K)=\mathrm{Im}\ \rho_n$. $\mathrm{Im}\ \rho_n$的大小反映了$K_n$的复杂程度,也就是$E$的挠点之间相互独立的程度.

可以将所有的$\rho_n$放在一起考虑. 记$E[\ell^{\infty}]=\bigcup_n E[\ell^n]$,$K_{\infty} = K(E[\ell^{\infty}])$. $K_{\infty}$也是Galois扩张,且$\mathrm{Gal}(K_{\infty}/K) = \lim\limits_{\longleftarrow} \mathrm{Gal}(K_n/K)$. 记$\rho = \lim\limits_{\longleftarrow} \rho_n$,$\rho: G\to \mathrm{GL}_2(\mathbb{Z}_{\ell})$,其中转移映射$\mathrm{GL}_2(\mathbb{Z}/\ell^{n+1} \mathbb{Z}) \to \mathbb{Z}/\ell^n \mathbb{Z}$,$\tau \mapsto \tau \pmod{\ell^n}$.

Tate提出了另一种将$\rho_n$放在一起的方式. 令Tate模$T_{\ell}(E) = \lim\limits_{\longleftarrow} E[\ell^n]$,其中转移映射为$E[\ell^{n+1}]\to E[\ell^n], P\mapsto \ell P$,则$T_{\ell}\cong (\mathbb{Z}_{\ell})^2$. $G$在$E[\ell^n]$上的作用还与椭圆曲线上的加法交换,因此在同构$T_{\ell}(E)\cong (\mathbb{Z}_{\ell})^2$下$G$的作用是线性映射. 这就得到了表示$G\mapsto \mathrm{GL}_2(\mathbb{Z}_{\ell})$. 可以看出Tate模定义的表示在模每个$\ell^n$之后都成为$\rho_n$,因此就是上一段中定义的$\rho$.

在$\mathrm{char}\ K = 0$,且$E$具有复乘的情况下,$\mathrm{Im}\ \rho$是相对简单的. 通过将$K$取成一个有限扩张,不妨设$\mathrm{End}_K(E) \cong \mathcal{O}$. 此时$T_{\ell}$是$\mathcal{O}_{\ell} = \lim\limits_{\longleftarrow}\mathcal{O}/\ell^n \mathcal{O}$上秩为$1$的模,$G$的作用与$\mathcal{O}_{\ell}$的作用交换. 此时$\rho$实际上是表示$G\to (\mathcal{O}_{\ell})^{\times}$,即$\mathrm{Im}\ \rho$是交换群.

如果$K$是代数数域,绝对Galois群的表示的像似乎遵从一个原则,“如果没有让像变小的理由,那么它就会尽可能地大”. $E$具有复乘就是一个让$\rho$的像变小的原因. Serre证明了复乘就是仅有的让像变小的原因,也就是本文的主定理:

\begin{cthm}
    令$K$是代数数域,$E$是定义在$K$上的椭圆曲线. 假设$\mathrm{End}_{\overline{K}}(E) \cong \mathbb{Z}$. 则$\mathrm{Im}\ \rho$是$\mathrm{GL}_2(\mathbb{Z}_{\ell})$的开子群. \label{main::open_image}
\end{cthm}


    \chapter{\texorpdfstring{$\ell$}{ELL}进Galois表示}

\subsection{一些记号}

\subsubsection{代数数论}

这一小节回顾一些代数数论和代数群理论的内容,用以固定后面要用到的记号.

令$K$是代数数域,$\mathfrak{m}$是$\order_K$的整理想. 记$U_{\mathfrak{m}, v}$为:当$K_v \cong \mathbb{C}$时,$K_v^{\times}$;当$K_v\cong \mathbb{R}$时,$K_v^{+}$;当$v\nmid\mathfrak{m}\infty$时,$U_{v}$;当$v(\mathfrak{m})=t>0$时,$1+\mathfrak{p}_v^{t}$. 记$U_{\mathfrak{m}}=\prod_v U_{\mathfrak{m}, v}$,$U_{\mathfrak{m}, finite} = \prod_{v\nmid \infty}U_{\mathfrak{m}, v}$(看作$\mathbb{A}_K^{\times}$的子群时,指要求$v\mid \infty$处取值为$1$). $E_{\mathfrak{m}} = K^{\times} \bigcap U_{\mathfrak{m}}$,$I_{\mathfrak{m}} = \mathbb{A}_K^{\times}/U_{\mathfrak{m}}$,$C_{\mathfrak{m}} = \mathbb{A}_K^{\times} / K^{\times}U_{\mathfrak{m}}$.

$T = \mathrm{Res}_{K/\mathbb{Q}}(\mathbb{G}_m/K)$,$T_{\mathfrak{m}} = T / \overline{E_{\mathfrak{m}}}$.

先给出Hecke特征的理想版本和idèle版本一一对应的精确形式:

\begin{cthm}
    以下两类对象之间存在一一对应:

    (1) $\chi: \mathbb{A}_K^{\times}/K^{\times} \to \mathbb{C}^{\times}$,满足$\chi(U_{\mathfrak{m}, finite}) = 1$

    (2) $\tilde{\chi}: J^{\mathfrak{m}}\to \mathbb{C}^{\times}$,使得存在$\chi_{\infty}:\prod_{v\mid \infty}K_v^{\times} \to \mathbb{C}^{\times}$,满足当$a\in K, a\equiv 1\pmod{\mathfrak{m}}$时,$\tilde{\chi}((a)) = \chi_{\infty}(a)$.

    其中映射$\chi\mapsto \tilde{\chi}$是,对所有$v\nmid \mathfrak{m}\infty$,选定一个素元$\pi_v\in \mathcal{O}_v$得到映射$J^{\mathfrak{m}}\to \mathbb{A}_K^{\times}$,再与$\chi$复合.
\end{cthm}

\begin{crem}
    在(2)中对$a$的要求再加上全正,对$\tilde{\chi}$的要求看上去是减弱了,但事实上是一样的. 可以通过强逼近定理转化到idèle版本再还原来看,也可以直接证明:令$K^{\mathfrak{m}} = \{\alpha\in K^{\times}, \alpha\equiv 1\pmod{\mathfrak{m}}\}$,$K^{\mathfrak{m}}_{+} = \{\alpha\in K^{\mathfrak{m}}, \alpha 全正\}$. 如果$\chi_{\infty}$满足当$a\in K^{\mathfrak{m}}_{+}$时,$\tilde{\chi}((a)) = \chi_{\infty}(a)$,则$\tilde{\chi}\chi_{\infty}^{-1}$定义了$K^{\mathfrak{m}}/K^{\mathfrak{m}}_{+}$的一个特征,但是则$K^{\mathfrak{m}}/K^{\mathfrak{m}}_{+} \cong \prod_{v实素点} \{\pm 1\}$. 因此存在一个$\phi: \prod_{v\mid \infty} K_v^{\times}\to \{\pm 1\}$使得在$K^{\mathfrak{m}}$上有$\tilde{\chi}\chi_{\infty}^{-1} = \phi$,于是$\chi_{\infty}\phi$满足定理中(看起来更强的)条件(2).
\end{crem}

\begin{cdef}
    Serre群$S_{\mathfrak{m}}$是一个$\mathbb{Q}$代数群,带着一个代数群态射$T_{\mathfrak{m}}\to S_{\mathfrak{m}}$和一个群同态$\mathbb{A}_K^{\times}/U_{\mathfrak{m}}\to S_{\mathfrak{m}}(\mathbb{Q})$使得图表交换,且是以上条件定义的范畴中的始对象
    \begin{figure}[H]
        \centering
        \begin{tikzcd}
            K^{\times}/E_{\mathfrak{m}}\arrow[r]\arrow[d] &\mathbb{A}_K^{\times}/U_{\mathfrak{m}}\arrow[d]\\
            T_{\mathfrak{m}}(\mathbb{Q})\arrow[r] &S_{\mathfrak{m}}(\mathbb{Q})
        \end{tikzcd}
    \end{figure}
\end{cdef}

\begin{cprop}
    如果$R$是$\mathbb{Q}$代数,且$\mathrm{Spec}\ R$连通,则有交换图表
    \begin{figure}[H]
        \centering
        \begin{tikzcd}
            0\arrow[r] &K^{\times}/E_{\mathfrak{m}}\arrow[r]\arrow[d] &\mathbb{A}_K^{\times}/U_{\mathfrak{m}}\arrow[d] \arrow[r] &C_{\mathfrak{m}}\arrow[d]\arrow[r] &0\\
            0\arrow[r] &T_{\mathfrak{m}}(R)\arrow[r] &S_{\mathfrak{m}}(R) \arrow[r] &C_{\mathfrak{m}}\arrow[r] &0
        \end{tikzcd}
    \end{figure}
    且图表中的两行都是正合列.
\end{cprop}

\begin{cprop}
    $X(S_{\mathfrak{m}}) = \mathrm{Hom}_{\overline{\mathbb{Q}}}(S_{\mathfrak{m}}/\overline{\mathbb{Q}}, \mathbb{G}_{m}/\overline{\mathbb{Q}})$可以被描述为以下对象的集合:$(\phi, \chi)$,其中$\phi: I_{\mathfrak{m}}\to \overline{\mathbb{Q}}^{\times}$,$\chi\in X(T_{\mathfrak{m}})$满足$\phi(y) = \chi(y), y\in K^{\times}/ E_{\mathfrak{m}}$.
\end{cprop}

以上两个命题可以直接通过$S_{\mathfrak{m}}$的构造来证明.

\begin{cprop}
    固定$\mathfrak{m}$. 存在$\mathbb{Q}$的有限扩张$E$使得对任意有限维表示$\rho:S_{\mathfrak{m}}\to GL(V)$,$\rho/E$可以对角化. \label{reps::split_finite}
\end{cprop}

\subsubsection{乘法型代数群}

命题\ref{reps::split_finite}的证明需要用到可对角化群和乘法型群的概念,因此专门开一小节来写.
假设$k$是一个特征$0$的域,以下讨论的代数群都是定义在$k$上的仿射代数群.

\paragraph*{可对角化群}

\begin{cprop}
    对任意有限生成交换群$M$(群运算用乘法记),函子$D(M): R\to \mathrm{Hom}(M, R^{\times})$是可由$\mathrm{Spec}\ k[M]$表出.
\end{cprop}

\begin{proof}
    要给出一个$k$代数的态射$k[M]\to R$就是给出一个群同态$M\to R^{\times}$.
\end{proof}

\begin{cdef}
    若一个代数群$G$同构于某个$D(M)$,则称$G$为可对角化的.
\end{cdef}

\begin{cprop}
    $M\to D(M)$和$X: G\mapsto \mathrm{Hom}_k(G, \mathbb{G}_m)$互为伪逆,定义了可对角化代数群和有限生成交换群范畴的反变等价,且$D,X$都是正合的.
\end{cprop}

\begin{proof}
    \cite{milne2017algebraic}, Theorem 12.9.
\end{proof}

\begin{cprop}
    如果$G$可对角化,则$G$的任何有限维表示都(在同一个域$F$上)可对角化,即可以通过$\sigma\in\mathrm{GL}_n(k)$的共轭成为对角矩阵.
\end{cprop}

\begin{proof}
    \cite{milne2017algebraic}, Theorem 12.12.
\end{proof}

\paragraph*{环面群}

\begin{cdef}
    若$G$在$k$的某个扩张上同构于有限个$\mathbb{G}_m$的乘积,则称$G$是环面群. 如果在$k$上就已经是同构的,则称$G$是分裂的环面群.
\end{cdef}

\paragraph*{乘法型群}

\begin{cdef}
    如果$G$在$k$的某个扩张下可对角化,则称$G$为乘法型的.
\end{cdef}

\begin{cprop}
    如果$G$是乘法型群,则$G$在$k$的某个有限扩张上可对角化.
\end{cprop}

\begin{proof}
    任何一个$G/\overline{k}\to \mathbb{G}_m/\overline{k}$可以定义在一个有限扩张上. 而$G$的一个对角化是由有限多个映射$G/\overline{k}\to \mathbb{G}_m/\overline{k}$定义的,因此也可以定义在一个有限扩张上.
\end{proof}

\begin{cprop}
    如果$G',G''$都是乘法型群,$G$是交换代数群,且有正合列
    \begin{equation}
        1\to G'\to G\to G''\to 1
    \end{equation}
    则$G$也是乘法型群.
\end{cprop}

\begin{proof}
    \cite{milne2017algebraic}, Corollary 12.22.
\end{proof}

\begin{cprop}
    如果$G$是乘法型群,那么$G$是群扩张
    \begin{equation}
        1\to G'\to G\to G''\to 1
    \end{equation}
    其中$G'$是环面群,$G''$是有限乘法型群.
\end{cprop}

\begin{proof}
    \cite{milne2017algebraic}, Corollary 12.24.
\end{proof}


\subsection{\texorpdfstring{$\ell$}{ELL}进表示的基本定义}

令$\ell$是一个素数,$k$是一个域,$k_s$是$k$的一个可分闭包. 记$G = \mathrm{Gal}(k_{s}/k)$.

\begin{cdef}
    $k$的一个$\ell$进表示是一个连续的同态$\rho: G\to GL(V)$,其中$V$是一个有限维$\mathbb{Q}_{\ell}$向量空间.
\end{cdef}

令$K$是一个代数数域. 记$\Sigma_K$是$K$的有限素点的集合.

\begin{cdef}
    设$v\in \Sigma_K$. 称一个$\ell$进表示$\rho$在$v$处非分歧,当且仅当对某一个$K_{s}$的素点$w\mid v$(等价于对所有的$w\mid v$)有$\rho(I_w) = 1$,$I_w$是惯性子群. 如果$\rho$在$v$处非分歧,则$D_w$中所有Frobenius元素在$\rho$下的像是同一个,称其为$\rho$下的Frobenius元素,记为$F_{w, \rho}$. 所有$w\mid v$定义的Frobenius元素形成$G$的一个共轭类,称为$v$处的Frobenius共轭类.
\end{cdef}

\begin{crem}
    如果$L$是$\ker \rho$的固定域,则$\rho$在$v$处非分歧当且仅当$L$在$v$处非分歧.
\end{crem}

我们需要\Chebotarev 密度定理的一个无穷版本,这里的密度指自然密度:

\begin{cthm}
    令$L$是$K$的Galois扩张,只在有限个素点处分歧. 给$H = \mathrm{Gal}(L/K)$赋Krull拓扑,令$\mu$是相应的Haar测度(注意到紧群都是幺模的),满足$\mu(H) = 1$,那么:

    (1) $L$的非分歧素点定义的Frobenius元素在$H$中稠密

    (2) 令$X$是$H$的一个在共轭下不变的子集,并假设$X$是开集或者$X$是闭集或者$X$的边界集在$\mu$下是零测集. 那么使得Frobenius元素落在$X$中的非分歧素点$v$的密度恰好为$\mu(X)$.
\end{cthm}

\begin{proof}
    只要证明(2). 令$K\subset L_1\subset L_2\subset \cdots \subset L$使得$L = \bigcup_i L_i$. 记$p_i: H\to \mathrm{Gal}(L_i/K)$为自然的投射. 如果$F$是闭集,则$F = \bigcap_i p_i^{-1}(p_i(F))$,因此由有限扩张的\Chebotarev 密度定理,结论对$F$成立. 如果$U$是开集,且$U$在共轭下不变,则$H-U$是在共轭下不变的闭集,结论对$H-U$成立,从而对$U$也成立. 取$F = \overline{X}, U = H - \overline{H-X}$. 但是$\mu(F) = \mu(U)$,因此结论对$X$成立.
\end{proof}

\begin{crem}
    如果$\rho$是一个$\ell$进表示,且在有限多个素点以外非分歧. 令$L$是$\ker(\rho)$的固定域,则$L$也在有限多个素点以外非分歧. 由\Chebotarev 密度定理,Frobenius元素在$\mathrm{Im}(\rho)$中稠密.
\end{crem}

\subsection{有理\texorpdfstring{$\ell$}{ELL}进表示}

令$\rho$是一个$\ell$进表示,$v$是一个素点使得$\rho$在$v$处非分歧. 记$P_{v, \rho}(T) = \det (1 - F_{w, \rho}T)$,其中$w\mid v$.

\begin{cdef}
    称$\rho$为一个有理$\ell$进表示当且仅当存在有限集合$S\subset \Sigma_K$使得当$v\in \Sigma - S$时,$\rho$在$v$处非分歧,而且$P_{v, \rho}$的系数都是有理数. 如果进一步的,$P_{v,\rho}$的系数都是整数,则称$\rho$为整$\ell$进表示.
\end{cdef}

\begin{cdef}
    令$\ell'$为一个素数(不一定和$\ell$不同),$\rho'$为$K$的$\ell'$进表示. 假设$\rho, \rho'$都是有理的. 称$\rho, \rho'$相容当且仅当存在有限集合$S\subset \Sigma_K$使得$\rho, \rho'$在所有$v\in \Sigma_K - S$处都非分歧,且$P_{v,\rho} = P_{v, \rho'}$.
\end{cdef}

令$\rho$为一个$\ell$进表示,$V$是表示空间. 则$V$有合成序列$0 = V_0\subset \cdots \subset V_q = V$,其中$V_{i+1}/V_i$是单$G$模. 令$V' = \bigoplus_i V_{i+1}/V_i$,则$V'$上定义了半单的有理$\ell$进表示,且和$\rho$相容. 称这个表示为$\rho$的半单化.

\begin{cdef}
    假设对每个素数$\ell$有一个有理$\ell$进表示$\rho_{\ell}$. 称$\{\rho_{\ell}\}$为严格相容的,当且仅当存在有限集合$S\subset \Sigma_K$满足:

    (1) 对任意的素数$\ell$,记$S_{\ell} = \{v\mid p_v = \ell\}$. 对所有的$v\in \Sigma_K - S\bigcup S_{\ell}$,$\rho_{\ell}$在$v$处非分歧,且$P_{v, \rho_{\ell}}$的系数都是有理数

    (2) 对任意的素数对$\ell, \ell'$,当$v\in \Sigma_K - S\bigcup S_{\ell}\bigcup S_{\ell'}$时有$P_{v, \rho_{\ell}} = P_{v, \rho_{\ell'}}$.
\end{cdef}

半单的表示可以被特征多项式决定,具体来说:

\begin{clem}
    令$k$是一个特征$0$的域,$A$是$k$代数,$M_1, M_2$是$A$的两个$k$-有限维半单模. 如果$\mathrm{Tr}\circ \rho_1 = \mathrm{Tr}\circ \rho_2$,则$M_1, M_2$同构.
\end{clem}

\begin{proof}
    只需要证明,如果$\{M_i\}$是有限多个互不同构的$k$-有限维不可约模,则$\mathrm{Tr}\ M_i: A\to k$线性无关. 令$N_i\subset A$是$M_i$的零化子,则$N_i$是双边理想,且作为左理想都是极大的. 由于$M_i$互不同构,$N_i$互不相同. 那么对每个$i$,存在$f_i\in A$使得$f_i\equiv 1\pmod{M_i}, f_i\equiv 0\pmod{M_j}, j\neq i$. 此时$\mathrm{Tr}\ M_i(f_i) = \dim_k M_i, \mathrm{Tr}\ M_j(f_i) = 0, j\neq i$.
\end{proof}

\begin{cthm}
    令$\rho$是有理$\ell$进表示,$\ell'$是一个素数(不一定和$\ell$不同). 那么在同构意义下至多有一个半单$\ell'$进表示$\rho'$,使得$\rho$和$\rho'$相容.
\end{cthm}

\begin{proof}
    令$\rho'_1, \rho'_2$是两个半单有理$\ell'$进表示,且都和$\rho$相容. 先证明$\mathrm{Tr}(\rho'_1(g)) = \mathrm{Tr}(\rho'_2(g))$对所有$g\in G$都成立. 记$J = \ker(\rho'_1)\bigcap \ker(\rho'_2)$,$H = G/J$,$M$是$J$对应的扩张. 那么$\mathrm{Gal}(M/K) = H$且$M/K$在有点多个素点以外非分歧. 由\Chebotarev 密度定理,Frobenius元素在$H$中稠密;而$\mathrm{Tr}\circ \rho'_1, \mathrm{Tr}\circ \rho'_1$可以定义为$H$上的连续函数,且在Frobenius元素上都相等,因此在整个$H$上相等,也就在整个$G$上相等. 在引理中取$k = \mathbb{Q}_{\ell'}, A = k[H]$,就得到$\rho'_1, \rho'_2$同构.
\end{proof}

\paragraph{例子:单位根}
假设$\ell\neq \mathrm{char}(K)$. 此时$x^{l^m}-1$是可分多项式,因此$K_s$中的$\ell^m$次单位根的乘法群$\mu_m$同构于$\mathbb{Z}/\ell^m \mathbb{Z}$. $G$作用在$\mu_m$上,因此作用在$\mathbb{Z}_{\ell} \cong T_{\ell}(\mu) = \invlim\mu_m$. 那么有$\chi_{\ell}:G\to \mathbb{Z}_{\ell}^{\times}\subset \mathbb{Q}_{\ell}^{\times}$,即一个一维的$\ell$进表示. 而如果$K$是代数数域,$v\in \Sigma_K, v\nmid \ell$,则$\chi_{\ell}$在$v$处非分歧,而且$F_{v, \chi_{\ell}} = Nv$. 因此$\chi_{\ell}$是一个整的$\ell$进表示. 当$\ell$取遍所有素数时,$\{\chi_{\ell}\}$形成了一族严格相容的有理$\ell$进表示,定义中的$S$可以取为空集.

\subsection{定义在线性代数群中的表示}

\begin{cdef}
    令$H$是一个$k$上的线性代数群,$k[H]$是$H$的坐标环. 若$f\in k[H]$满足对任意的交换$k$代数$k'$以及$x,y\in H(k')$,都有$f(xy)=f(yx)$,则称$f$是中心函数. 若$x\in H(k')$,且对任意的中心函数$f\in k[H]$都有$f(x)\in k$,则称$x$的共轭类是有理的.
\end{cdef}

\begin{cdef}
    令$H$是一个$\mathbb{Q}$上线性代数群. $K$的一个在$H$中取值的$\ell$进表示是指一个连续的$\rho: \mathrm{Gal}(\overline{K}/K)\to H(\mathbb{Q}_{\ell})$.
\end{cdef}

显然地定义非分歧、Frobenius元素.

\begin{cdef}
    称一个在$H$中取值的$\ell$进表示$\rho$为有理的,当且仅当存在有限子集$S\subset \Sigma_K$使得当$v\in \Sigma_K - S$时,$\rho$在$v$处非分歧,而且$F_{v, \rho}$的共轭类是有理的. ($k=\mathbb{Q}, k'=\mathbb{Q}_{\ell}$)
\end{cdef}

显然地定义相容性和严格相容性.

\begin{crem}
    如果$H$是交换的,则$F_{v,\rho}$的共轭类是有理的当且仅当$F_{v,\rho} \in H(\mathbb{Q})$.
\end{crem}

\begin{crem}
    如果$H = GL_n$,则由Chevalley关于不变多项式的定理,$H$上的中心函数形成的子代数就是$k[t_0,\ldots,t_{n-1}, (\det)^{-1}]$,其中$t_0=\det,t_1,\ldots,t_{n-2},t_{n-1}=\mathrm{tr}$是特征多项式的系数. 因此前一节定义的$\ell$进表示的有理性和本节定义的是一致的.
\end{crem}


\subsection{在\texorpdfstring{$S_{\mathfrak{m}}$}{Sm}中取值的表示}

记$\varepsilon : \mathbb{A}_K^{\times} \to I_{\mathfrak{m}}\to S_{\mathfrak{m}}(\mathbb{Q})$,$\pi: T \to T_{\mathfrak{m}}\to S_{\mathfrak{m}}$为定义$S_{\mathfrak{m}}$的映射. 对$\pi$取$\mathbb{Q}_{\ell}$点,得到$\pi_{\ell} : T(\mathbb{Q}_{\ell}) \to S_{\mathfrak{m}}(\mathbb{Q}_{\ell})$. 但是$T(\mathbb{Q}_{\ell}) = (K\otimes \mathbb{Q}_l)^{\times} = \prod_{v\mid l} K_v^{\times}$,由此得到映射$\alpha_{\ell} : \mathbb{A}_K^{\times} \xrightarrow{proj} T(\mathbb{Q}_{\ell}) \xrightarrow{\pi_{\ell}} S_{\mathfrak{m}}(\mathbb{Q}_{\ell})$.

\begin{clem}
    以下图表交换,其中的映射或者是上面定义的,或者是自然嵌入和投射
    \begin{figure}[H]
        \centering
        \begin{tikzcd}
            \mathbb{A}_K^{\times} \arrow[ddr, bend right] \arrow[drr, bend left]& & & \\
             &K^{\times} \arrow[ul] \arrow[r]\arrow[d] &K^{\times}/E_{\mathfrak{m}} \arrow[r]\arrow[d] &I_{\mathfrak{m}} \arrow[d]\\
             &T(\mathbb{Q}_{\ell}) \arrow[r] &T_{\mathfrak{m}}(\mathbb{Q}_{\ell}) \arrow[r] &S_{\mathfrak{m}}(\mathbb{Q}_{\ell})
        \end{tikzcd}
    \end{figure}
\end{clem}

因此,$\varepsilon_{\ell}(a) = \varepsilon(a) \alpha_{\ell}(a^{-1})$定义了$C_K \to S_{\mathfrak{m}}(\mathbb{Q}_{\ell})$的映射. 但是$S_{\mathfrak{m}}(\mathbb{Q}_{\ell})$的拓扑是完全不连通的,因此$C_K$的连通分支$D_K$的像为$1$. 那么,由类域论,$\varepsilon_{\ell}$定义了$G^{\mathrm{ab}}\to S_{\mathfrak{m}}(\mathbb{Q}_{\ell})$的映射.

记$F_v = \varepsilon(f_v) \in S_{\mathfrak{m}}(\mathbb{Q})$,其中$f_v$是任何一个在$v$处取素元,在其它地方取$1$的idèle.

\begin{cthm}
    (1) $\varepsilon_{\ell}$是一个取值在$S_{\mathfrak{m}}$中的有理$\ell$进表示

    (2) $\varepsilon_{\ell}$在$v\in \Sigma_K - \mathrm{supp}(\mathfrak{m})\bigcup S_{\ell}$处非分歧,且$F_{v, \varepsilon_{\ell}} = F_v \in S_{\mathfrak{m}}(\mathbb{Q})$.

    (3) $\{\varepsilon_{\ell}\}$形成了一族取值在$S_{\mathfrak{m}}$中的严格相容的$\ell$进表示.
\end{cthm}

\begin{proof}
    如果$v\in \Sigma_K - \mathrm{supp}(\mathfrak{m}), a\in U_v$,则$\epsilon(a) = 1$. 如果进一步地,$v\nmid \ell$,则$\alpha_{\ell}(a) = 1$,因此$\varepsilon_{\ell}$在$v$处非分歧. 同时,$\varepsilon_{\ell}(f_v)=\varepsilon(f_v) = F_v$,即$v$处的Frobenius元素是$F_v$.
\end{proof}

\begin{cthm}
    $\mathrm{Im}(\varepsilon_{\ell})$在$\basechange{S_{\mathfrak{m}}}{\mathbb{Q}_{\ell}}$中Zariski稠密.
\end{cthm}

\begin{proof}
    $\varepsilon$在$U_{\ell, \mathfrak{m}} = \prod_{v\mid \ell} U_{v, \mathfrak{m}}$上平凡,故$\epsilon_{\ell}(U_{\ell, \mathfrak{m}}) = \pi_{\ell}(U_{\ell, \mathfrak{m}})\subset T_{\mathfrak{m}}(\mathbb{Q}_{\ell}) \subset S_{\mathfrak{m}}(\mathbb{Q}_{\ell})$. 因此,$\mathrm{Im}(\varepsilon_{\ell})$是$S_{\mathfrak{m}}(\mathbb{Q}_{\ell})$的($\ell$进拓扑下的)开子群. 另一方面,由素理想在$C_{\mathfrak{m}}$中分布的经典结果,$f_v\in I_{\mathfrak{m}}$在$C_{\mathfrak{m}}$中的像取遍了整个$C_{\mathfrak{m}}$.
    因此$\mathrm{Im}(\varepsilon_{\ell})$在$S_{\mathfrak{m}}(\mathbb{Q}_{\ell})$的Zariski拓扑下稠密.
\end{proof}

\begin{ccor}
    $\{F_v\}$在$S_{\mathfrak{m}}$中Zariski稠密. \label{frob_dense}
\end{ccor}

\begin{proof}
    令$X$表示所有$F_v$的集合. 令$\ell$是素数. 令$\overline{X}, \overline{X}_{\ell}$分别为$X$在$S_{\mathfrak{m}}$(Zariski拓扑),$S_{\mathfrak{m}}(\mathbb{Q}_{\ell})$($\ell$进拓扑)中的闭包. 那么$\overline{X}_{\ell}\subset \overline{X}(\mathbb{Q}_{\ell})$. 但是\Chebotarev 密度定理说明,$X$在$\mathrm{Im}(\varepsilon_{\ell})$中($\ell$进)稠密,即$\overline{X}_{\ell} = \mathrm{Im}(\varepsilon_{\ell})$. $\mathrm{Im}(\varepsilon_{\ell})$在$\basechange{S_{\mathfrak{m}}}{\mathbb{Q}_{\ell}}$(Zariski)稠密,因此$\overline{X}(\mathbb{Q}_{\ell}) = S_{\mathfrak{m}}(\mathbb{Q}_{\ell})$.
\end{proof}

\subsection{通过\texorpdfstring{$S_{\mathfrak{m}}$}{Sm}定义的表示}

设$V_{\ell}$是$\mathbb{Q}_{\ell}$上的有限维线性空间,$\varphi: \basechange{S_{\mathfrak{m}}}{\mathbb{Q}_{\ell}} \to GL(V_{\ell})$是$\basechange{S_{\mathfrak{m}}}{\mathbb{Q}_{\ell}}$的表示. 令$\varphi_{\ell}$为映射$G^{\mathrm{ab}}\to S_{\mathfrak{m}}(\mathbb{Q}_{\ell})\xrightarrow{\varphi} GL(V_{\ell})$.

\begin{cthm}
    (1) $\varphi_{\ell}$是半单表示

    (2) 令$v\in \Sigma_K - \mathrm{supp}(\mathfrak{m})\bigcup S_{\ell}$,则$\varphi_{\ell}$在$v$处非分歧,且$F_{v, \varphi_{\ell}} = \varphi(F_v)$

    (3) $\varphi_{\ell}$是有理表示当且仅当$\varphi$可以在$\mathbb{Q}$上定义 \label{single_ell}
\end{cthm}

\begin{proof}
    $S_{\mathfrak{m}}$是乘法型群,任意表示都可以在有限扩张后对角化,由此得到(1). 由$\epsilon_{\ell}$的非分歧性和Frobenius元素的指定,(2)是显然的.
    为了不打断行文顺序,(3)的证明留到本小节最后给出.
\end{proof}

(3)告诉我们,考察有理表示时,只需要从一个$\mathbb{Q}$上的表示$S_{\mathfrak{m}}\to GL(V)$出发.

设$V$是$\mathbb{Q}$上的有限维空间,$\varphi: S_{\mathfrak{m}}\to GL(V)$是$S_{\mathfrak{m}}$的表示. 令$V_{\ell} = V\otimes \mathbb{Q}_{\ell}$,那么有$\varphi_{\ell}: \basechange{S_{\mathfrak{m}}}{\mathbb{Q}_{\ell}}\to GL(V_{\ell})$. 按照上面的讨论,这就是在说有表示$\varphi_{\ell}: G^{\mathrm{ab}}\to GL(V_{\ell})$. 称如此定义的$\{\varphi_{\ell}\}$为通过$S_{\mathfrak{m}}$定义的.

\begin{cthm}
    (1) $\{\varphi_{\ell}\}$形成了一族严格相容的有理$\ell$进表示,例外集不大于$\mathrm{supp}(\mathfrak{m})$

    (2) 当$v\in \Sigma_K - \mathrm{supp}(\mathfrak{m})\bigcup S_{\ell}$时,$F_{v, \varphi_{\ell}} = \varphi(F_v)$

    (3) 存在无穷多个素数$\ell$使得$\varphi_{\ell}$在$\mathbb{Q}_{\ell}$上可以对角化
\end{cthm}

\begin{proof}
    (1)和(2)由定理\ref{single_ell}对单个$\ell$的计算容易得到. 由命题\ref{reps::split_finite},存在$K$的有限扩张$E$使得$\varphi$在$E$上对角化. 如果$\ell$在$E$上完全分裂,则有嵌入$E\to \mathbb{Q}_{\ell}$,此时$\varphi_{\ell}$在$\mathbb{Q}_{\ell}$上可以对角化. 由\Chebotarev 密度定理,存在无穷多个这样的$\ell$.
\end{proof}

\begin{ccor}
    如果$\{\rho_{\ell}\}$是一族通过$S_{\mathfrak{m}}$定义的$\ell$进表示,则存在无穷很多个素数$\ell$使得$\rho_{\ell}$是一维表示的直和;特别地,$\rho_{\ell}$在$\mathbb{Q}_{\ell}$上可约.\label{sm_to_reducile}
\end{ccor}

现在来证明定理\ref{single_ell}的(3). 回顾,如果$H$是$k$上的代数群,记$\mathsf{Rep}_k(H)$为$H$的有限维半单表示的范畴. 当$k_1$是$k$的扩张是,有自然的$-\otimes k_1: \mathsf{Rep}_k(H)\to \mathsf{Rep}_{k_1}(\basechange{H}{k_1})$. 如果$\rho\in \mathsf{Rep}_{k_1}(H/k_1)$落在$-\otimes k_1$的像(essential image)中,则称$\rho$可以定义在$k$上.

\begin{cprop}
    令$\mathbb{Q}\subset k\subset k_1$. 对于$\varphi\in \mathsf{Rep}_{k_1}(\basechange{S_{\mathfrak{m}}}{k_1})$,以下性质等价:

    (1) $\varphi$可以定义在$k$上

    (2) 对任意的$v\not\in \mathrm{supp}(\mathfrak{m})$,$\varphi(F_v)$的特征多项式的系数都落在$k$中

    (3) 存在密度为$1$的子集$\Sigma \subset \Sigma_K$使得当$v\in \Sigma$时,$\mathrm{Tr}(\varphi(F_v))\in k$.
\end{cprop}

\begin{proof}
    (1)$\Rightarrow$(2)$\Rightarrow$(3)是显然的.
    %TODO: 22 avr
\end{proof}


\subsection{局部代数表示}

这一小节给出一个$\ell$进表示可以从$S_{\mathfrak{m}}$定义的充分条件.

先看局部域的情况.
令$L$是$\mathbb{Q}_p$的有限扩张,$T = \mathrm{Res}_{L/\mathbb{Q}_p}(\mathbb{G}_m / L)$.
假设$V$是有限维$\mathbb{Q}_p$向量空间,$\rho$是一个Abel的$p$进表示$\rho: \mathrm{Gal}(\overline{L}/L)^{ab} \to \mathrm{Aut}(V)$.

\begin{cdef}
    称$\rho$是局部代数表示是在说,存在代数群的态射$r:T\to \mathrm{GL}(V)$使得当$x\in L^{\times}$足够接近于$1$时,$\rho\circ \iota(x) = r(x^{-1})$,其中$\iota: L^{\times}\to \mathrm{Gal}(\overline{L}/L) $是局部Artin映射.(这里局部Artin映射约定为$\pi$映射到Frobenius元素的逆)
\end{cdef}

一维的局部代数表示就是代数Hecke特征的$\ell$进类比. 由于$T$是环面,而$\rho$“差不多”是一个$T$的代数表示,可以将$\rho$对角化为一系列一维表示来处理.

\begin{cprop}
    如果$\rho: \mathrm{Gal}(\overline{L}/L)^{ab} \to \mathrm{Aut}(V)$是局部代数的表示,则$\rho$限制在惯性子群上的表示是半单的.
\end{cprop}

\begin{proof}
    由定义,存在$U_L$的开子群$U'$和代数群的态射$r: T\to \mathrm{GL}(V)$使得当$x\in U'$时$\rho\circ \iota(x) = r(x^{-1})$. 如果$W$是$V$的$\rho\circ\iota(U_L)$-不变的子空间,那么$W$也是$\rho\circ\iota(U')=r(U')$-不变的. 又因为$U'$在$T$中Zariski-稠密,$W$也是$r(T)$不变的. 但是$T$是环面群,所有$T$的表示都是半单的. 因此$W$有补空间,即存在投影映射$\pi: V\to W$使得$\pi$和$r(T)$的作用交换. 此时$\pi$也和$\rho\circ\iota(U')$的作用交换. 再令$\pi' = \frac{1}{[U_L:U']} \sum_{s\in U_L/U'} \rho\circ \iota(s) \pi \rho\circ \iota(s^{-1})$,则$\pi'$是和$\rho\circ\iota(U_L)$的作用交换的投影映射. 从而$W$在$V$中有$\rho\circ\iota(U_L)$-不变的补空间.
\end{proof}

固定一个$\rho$,取一个足够大的有限扩张$E/\mathbb{Q}_p$使得(i)$\rho$在$U_L$上的限制在$E$上可以对角化;(ii)$E$包含了$L$的正规闭包. 令$\Gamma_{L}$为所有域嵌入$L\to E$形成的群. 记$\chi_i: U_L\to E^{\times}, i=1,2,\ldots,\mathrm{dim}\ \rho$满足$\rho \cong_{E} \mathrm{diag}(\chi_1,\ldots, \chi_{\mathrm{dim}\ \rho})$.

\begin{cprop}
    $\rho$是局部代数表示当且仅当存在整数$n_{\sigma}(i)$使得对所有的$i$以及足够接近于$1$的$u$都有
    \begin{equation}
        \chi_i(u) = \prod_{\sigma\in \Gamma_K} \sigma(u)^{-n_{\sigma}(i)} \label{eqns::temp::1}
    \end{equation}
\end{cprop}

\begin{proof}
    如果$\rho$是局部代数的,则每个$\chi_i$都是局部代数的,这就等价于\refeq{eqns::temp::1}.

    反之,\refeq{eqns::temp::1}定义了$T$的$E$上的态射$r$,且满足和$\rho$的相容性关系. 只要证明$r$可以定义在$\mathbb{Q}_p$上. 但是在某个小邻域$U'$上,$r$的迹就是$\rho$的迹,从而是$\mathbb{Q}_p$的元素;同时,$U'$在$T$中Zariski-稠密.
\end{proof}

现在可以证明这一小节的主定理:

\begin{cthm}
    设$\rho: \mathrm{Gal}(\overline{K}/K)^{ab} \to \mathrm{Aut}(V_{\ell})$是有理$\ell$进表示,且$\rho$是局部代数的,$\mathfrak{m}$是$\rho$的定义模. 那么存在$V_{\ell}$的$\mathbb{Q}$-向量子空间$V$使得$V_{\ell} = V\otimes \mathbb{Q}_{\ell}$,以及代数群的态射$\phi: S_{\mathfrak{m}}\to \mathrm{GL}(V)$使得$\rho$同构于$\phi_{\ell}$.
\end{cthm}

\begin{crem}
    由Waldschmidt的结果,实际上半单的Abel的有理$\ell$进表示都是局部代数的\cite{waldschmidt1981transcendance}.
\end{crem}


\subsection{交换表示的局部代数性}

Serre在\parencite{serre1997abelian}中为了证明$\ell$进表示的局部代数性质给出了三种方法,其中两种利用Hodge-Tate分解和对局部域上相应表示的分析. 但事实上这些表示是局部代数的事实与它们来自椭圆曲线并没有多大关系. Serre的第三种方法是用Lang关于超越数的结果证明以下简洁的结果,但由于Lang的结果所限只能局限在$K$是二次域的复合的情形. 现在,有了更强大的Waldschmidt关于超越数的结果,我们可以证明

\begin{cthm}
    如果$\rho$是有理、半单、交换的$\ell$进表示,那么$\rho$是局部代数的. \label{reps::abelian_loc_alg}
\end{cthm}

证明定理\ref{reps::abelian_loc_alg}前还需要一些准备工作.

\begin{cprop}
    如果$\rho$是有理、半单、交换的表示,且存在正整数$N$使得$\rho^N$是局部代数的,那么$\rho$是局部代数的.
\end{cprop}

\begin{proof}

    由半单性,$\rho$可以在$\mathbb{Q}_{\ell}$的一个有限扩张上对角化. 将$\rho$写成$\mathrm{diag}(\psi_1,\ldots,\psi_n)$,其中$n=\mathrm{dim}\ \rho$,$\psi_i: \mathbb{A}_{K}^{\times}/K^{\times}\to \overline{\mathbb{Q}_{\ell}}^{\times}$是连续特征.
    令$\chi_i = \psi_i^N$是$\rho^N$的特征. 由于$\rho^N$是局部代数的,对每个$\chi_i$都存在一个$\chi_i^{alg}\in X^{*}(T)$使得当$x$充分靠近$1$时$\chi_i(x) = \chi_i^{alg}(x^{-1})$. 而$X^{*}(T)$可以看作是$\mathrm{Hom}(\basechange{T}{\overline{\mathrm{Q}_{\ell}}}, \basechange{\mathbb{G}_m}{\overline{\mathbb{Q}_{\ell}}})$,即$\chi_i^{alg}$可以写成$\prod_{\sigma\in \Gamma} \sigma^{n_{\sigma}(i)}$,其中$\Gamma$是$K$到$\overline{Q_{\ell}}$的域嵌入的群.

    \vskip0.3cm

    先证明,每个$n_{\sigma}(i)$都被$N$整除. 令$U$是$\overline{\mathbb{Q}_{\ell}}^{\times}$的开子群,且不包含非平凡的$N$次单位根. 取$\mathfrak{m}$为一个充分大的理想,使得对

    (i) $\psi_i(U_{\mathfrak{m}})\subset U$;

    (ii) $\rho^N$可以模$\mathfrak{m}$定义;

    (iii) $\rho$在$v\not\in \mathrm{supp}(\mathfrak{m})$处非分歧,且$F_{v, \rho}$的特征多项式是有理系数的.

    令$K_{\mathfrak{m}}$是$K^{\times} U_{\mathfrak{m}}$对应的Abel扩张. 取一个包含$K_{\mathfrak{m}}$的Galois$L/\mathbb{Q}$. 再取一个充分大的素数$p$,使得$p>\ell$,$p>p_v,\forall v\in \mathrm{supp}(\mathfrak{m})$,而且$p$在$L$中完全分裂. 令$v$是$K$的素点,且$v\mid p$. 记$f_v$为在$v$处取素元,其余处取$1$的idèle.

    $v$在$K_{\mathfrak{m}}$中完全分解,因此$f_v \in N_{K_{\mathfrak{m}}/K}$,即$f_v\in K^{\times} U_{\mathfrak{m}}$. 因此素理想$p_v$是主理想$(\alpha)$,其中$\alpha\equiv 1\pmod{\mathfrak{m}\prod_{k_w\cong \mathbb{R}} w}$.

    记$x = \phi_i(f_v)$,$y = \chi_i(f_v)$,由定义有
    \begin{equation}
        y = \chi_i(f_v) = \chi_i(\alpha^{-1}_{\ell}) = \chi_i^{alg}(\alpha_{\ell}) = \prod_{\sigma\in \Gamma} \sigma(\alpha)^{n_{\sigma}(i)}
    \end{equation}
    其中$\sigma\in \Gamma$看成是$K\to L$的嵌入. $x,y$都看成是$L$的元素.

    此时$y\in \tilde{L}\subset \mathbb{Q}_p$,$\tilde{L}$是$\mathbb{Q}_p$中(唯一的)与$L$同构的子域. 如果$w_{\sigma}$是$L$的素点,使得$w_{\sigma}\circ \sigma$限制在$K$上是$v$,那么$w_{\sigma}(y) = n_{\sigma}(i)$.

    如果$N$不整除$n_{\sigma}(i)$,那么$x\not\in \overline{L}$. 那么存在一个非平凡的$N$次单位根$z$使得$x, zx$在$\overline{L}$上共轭,从而在$\mathbb{Q}$上共轭. 但是$F_{v,\rho}$的特征多项式是有理系数的,$F_{v,\rho}$的特征值的共轭也是特征值. 那么存在一个$j$使得$\psi_j(f_v) = zx = z\psi_i(f_v)$. 但是$\psi_j(f_v)\in U$,$\psi_i(f_v)\in U$,与$U$中没有非平凡的$N$次单位根矛盾.

    \vskip0.3cm

    回到命题本身的证明. 由于$n_{\sigma}(i)$都被$N$整除,存在$\phi_i\in X^{*}(T)$使得$\phi_i^N = \chi_i^{alg}$. 当$x\in K_{\ell}^{\times}$充分靠近$1$时,
    \begin{equation}
        \phi_i(x^{-1})^N = \chi_i^{alg}(x^{-1}) = \chi_i(x) = \psi_i(x)^N
    \end{equation}
    此时$\phi_i(x)\psi_i(x)$是$N$次单位根. 但是$N$次单位根的群是离散的,从而存在一个$1$的小邻域使得$\phi_i\psi_i$在这个邻域上是$1$. 令$\phi=\mathrm{diag}(\phi_1,\ldots,\phi_n)$,则在$1$的小邻域上$\rho = \psi^{-1}$. 由于$\rho$能定义在$\mathbb{Q}_{\ell}$上,$\phi$也能定义在$\mathbb{Q}_{\ell}$上.

    那么$\rho$是局部代数的.
\end{proof}

接下来需要一个环面群的特征的结论.
令$L$为$\overline{\mathbb{Q}_{\ell}}$的完备化,$T$为$\mathbb{Q}$上的$n$维环面子群.
设$f: T(\mathbb{Q}_{\ell})\to L^{\times}$是连续映射. 若存在$1$在$T(\mathbb{Q}_{\ell})$的($\ell$进拓扑下)开邻域$U$和$\phi\in X^{*}(T)$使得当$x\in U$时$f(x)=\phi(x)$,则称$f$是局部代数的. 如果存在正整数$N$使得$f^N$是局部代数的,则称$f$是几乎局部代数的.

设$S$是素数的一个有限集合. 对每个$p\in S$,令$W_p$是$T(\mathbb{Q}_p)$的开子群. 记$T(\mathbb{Q})_{W}$为所有的$x\in T(\mathbb{Q})$使得$x\in W_p\subset T(\mathbb{Q}_p)$. 那么$T(\mathbb{Q})_{W}$是$T(\mathbb{Q})$的子群.

\begin{cprop}
    如果存在一族$W_p, p\in S$使得$f(T(\mathbb{Q})_W)$都是$\mathbb{Q}$-代数数,则$f$是几乎局部代数的. \label{temp::reps::loc_alg}
\end{cprop}

需要用到Waldschmidt的定理和它的推论
\begin{cthm}[\parencite{waldschmidt1981transcendance}, Théorème 1.1.p]
    令$L$是完备的非阿基米德域,$\mathrm{char}\ L = 0$,剩余类域特征为$p$. 设$v(p) = 1$. 记$E=\{z\in K\mid v(z) > \frac{1}{p-1}\}$为$\exp$的收敛区域. 令$x_1,\ldots,x_d,y_1,\ldots,y_l\in L^n$,其中$ld>n(l+d)$,$\{x_i\}, \{y_j\}$分别$\mathbb{Z}$线性无关. 记$X,Y$分别为$\{x_i\}, \{y_j\}$生成的$\mathbb{Z}$模. 假设$\langle X, Y\rangle \subset E$.

    如果所有的$\exp \langle x, y\rangle, x\in X, y\in Y$都是代数数,那么有分解
    \begin{equation}
        X = X_1\oplus X_2, Y=Y_1\oplus Y_2
    \end{equation}
    使得$\langle X_1, Y_2\rangle = 0$,且如果$d_1 = \mathrm{rank}(X_1)$,$l_1 = \mathrm{rank}(Y_1)$,$n_1 = \dim_L X_1\otimes L$,则有
    \begin{equation}
        \frac{d_1}{n_1} > \frac{d}{n}, l_1d_1\leq n_1(l_1+d_1)
    \end{equation}
\end{cthm}

取$d=n+1$,$l$充分大,就得到
\begin{ccor}[\parencite{waldschmidt1981transcendance}, Corollaire 1.2.p]
    令$\alpha_{i,j} \in L^{\times}, 1\leq i\leq n, 1\leq j\leq m$是代数数,其中$m\geq n^2+n+1$.
    如果$t_1,\ldots,t_n\in L$使得对每个$j$
    \begin{equation}
        \exp\paten{\sum_i t_i\log \alpha_{i, j}}
    \end{equation}
    都收敛并且值为代数数,那么
    
\end{ccor}

\begin{proof}[命题\ref{temp::reps::loc_alg}的证明]

    令$\chi_1,\ldots,\chi_n$是$X^{*}(T)$的一组基.
    $T(\mathbb{Q})$是无限秩的$\mathbb{Z}$模.
    而$T(\mathbb{Q}_p)/W_p$都是秩不大于$n$的有限生成$\mathbb{Z}$模,于是$T(\mathbb{Q})/T(\mathbb{Q})_W$是有限生成的. 从而$T(\mathbb{Q})_W$是无限秩的$\mathbb{Z}$模.

    $T(\mathbb{Q}_{\ell})$是一个$\ell$进李群. 令$\mathfrak{t}$为其李代数. 则存在$\mathfrak{t}$的一个紧开子群$\mathfrak{t}_0$使得$\exp: \mathfrak{t}_0\to T(\mathbb{Q}_{\ell})$是到一个紧开子群的同胚,而且群结构可以有Campbell-Hausdorff公式计算. 由于$T(\mathbb{Q}_{\ell})$是交换的,$\mathfrak{t}$也是交换的. $\exp$是$(\mathbb{Z}_{\ell})^n\cong \mathfrak{t}_0$到$T(\mathbb{Q}_{\ell})$的一个紧开子群的解析同构.

    通过与$\exp$复合,可以得到$n+1$个连续映射
    \begin{equation}
        f\circ\exp, \chi_i\circ\exp, i=1,2,\ldots,n
    \end{equation}
    由于$\mathbb{Z}_{\ell}\to L^{\times}$的连续映射局部上都是指数映射,存在$b_0, b_1,\ldots,b_n \in L^n$使得在充分小的开子群上有
    \begin{equation}
        f\circ\exp(z) = \exp \langle b_0 z\rangle, \chi_i\circ\exp = \exp\langle b_i z\rangle, i=1,2,\ldots,n
    \end{equation}
    通过在$b_i$上乘上适当的$\ell^k$,不妨设所有的$\langle b_0, (\mathbb{Z}_{\ell})^n\rangle$都落在$\exp$的收敛区域内,且以上各式在$(\mathbb{Z}_{\ell})^n$上成立.

    令$\Lambda = \{z\in (\mathbb{Z}_{\ell})^{\ell}\mid \exp(z)\in T(\mathbb{Q})_{W}\}$. 由于$T(\mathbb{Q}_{\ell})/\exp(\mathbb{Z}_{\ell})$都是有限生成的,$\Lambda$是无限秩的$\mathbb{Z}$模. 而当$z\in \Lambda$时,$f\circ \exp(z)$和$\chi_i\circ\exp(z)$都是代数数. 由Waldschmidt定理的推论,$b_0,b_1,\ldots,b_n$ $\mathbb{Z}$线性相关. 但是$\{\chi_i\}$是$X^{*}(T)$的一组基,$b_1,\ldots,b_n$ $\mathbb{Z}$线性无关;那么只能是$f$在局部上是$\chi_i$的有理系数线性组合. 因此存在$N$使得$f^N$是局部代数的.
\end{proof}

现在开始定理\ref{reps::abelian_loc_alg}的证明.
\begin{proof}

    由半单性,$\rho$在$L$上对角化. 设$\rho \cong_{L} \mathrm{diag}(\psi_1,\ldots,\psi_n)$. 令$f_i$为$\psi_i$在$K_{\ell}^{\times}$上的限制.

    下面验证命题\ref{temp::reps::loc_alg}的条件.
\end{proof}



    \section{椭圆曲线的一些结论}

这一节汇总了一些要用到的椭圆曲线中的结论.

\subsection{Néron-Ogg-Shafarevich判别法}

当惯性子群的作用是平凡的时候,称一个Galois模为非分歧的.

\begin{cthm}[Néron-Ogg-Shafarevich]
    令$K$是局部域,$E$是$K$上的椭圆曲线,则以下命题等价:

    (i) $E$有好约化;

    (ii) 当$(m, \mathrm{char}\ k)=1$时,$E[m]$非分歧;

    (iii) 当$\ell\neq \mathrm{char}\ k$时,$T_{\ell}$非分歧;

    (iv) 对无穷多个与$\mathrm{char}\ k$互素的$m$,$E[m]$非分歧.
\end{cthm}

\begin{proof}
    \cite{silverman2009arithmetic}, Theorem 7.1.
\end{proof}

\begin{ccor}
    如果$K$是代数数域,$E, E'$是$K$上同源的椭圆曲线,$v$是$K$的有限素点. 则$E$在$v$处有好约化当且仅当$E'$在$v$处有好约化. \label{galois::same_reduction}
\end{ccor}

\begin{proof}
    设$E\to E'$是次数为$n$的同源,则有$K_v$上相同次数的同源. 令$m$是与$\mathrm{char}\ k_v$,$\mathrm{deg}(E\to E')$都互素的正整数,那么Galois模$E_v[m]$和$E'_v[m]$同构. 由Néron-Ogg-Shafarevich判别法,$E, E'$在$v$处同时有好约化或者同时有坏约化.
\end{proof}


\subsection{Shafarevich的定理和一些推论}

令$K$是代数数域.

\begin{cthm}[Shafarevich]
    如果$S$是$K$的素点的一个有限集合,那么$K$上在$S$以外有好约化的椭圆曲线(在$K$-同构意义下)只有有限多条.
\end{cthm}

\begin{ccor}
    给定$K$上的椭圆曲线$E$,则和$E$在$K$上同源的椭圆曲线(在$K$-同构意义下)只有有限多条.\label{galois::isogeny_finite_curves}
\end{ccor}

\begin{proof}
    Shafarevich的定理和\ref{galois::same_reduction}.
\end{proof}


\section{Galois模的不可约性}

设$K$是代数数域,$E$是定义在$K$上的椭圆曲线. 本小节要证明

\begin{cthm}
    如果$E$在$K$上没有复乘,则:

    (i) 对所有素数$\ell$,$V_{\ell}$不可约;

    (ii) 对除有限个以外的素数$\ell$,$E[\ell]$不可约.
    \label{galois::irreducible}
\end{cthm}

\begin{clem}
    如果$E$在$K$上没有复乘,$E'\to E$和$E''\to E$是两个定义在$K$上的同源,且核是不同构的循环群. 那么$E'$和$E''$在$K$上不同构.
\end{clem}

\begin{proof}
    假设$E'\to E$和$E''\to E$的核分别是阶为$n', n''$的循环群,$E'\to E''$是一个同构. 令$E\to E'$是$E'\to E$的转置,则它的核是$n'$阶循环群. 此时以上几个映射的复合$E\to E'\to E''\to E$的核是$n''$阶循环群关于$n'$阶循环群的群扩张. 但是$\mathrm{End}_K(E) \cong \mathrm{Z}$,于是这个映射只能是某个$a$倍映射,核只能是$(\mathbb{Z}/a\mathbb{Z})^2$的形式. 此时$n'\mid a, n''\mid a$且$n'n''=a^2$,那么只能是$a=n'=n''$,矛盾.
\end{proof}

\begin{proof}
    (定理\ref{galois::irreducible})
    (i) 假设$V_{\ell}$有一个一维的不变子空间$Y$,则$X=Y\bigcap T_{\ell}$是$T_{\ell}$的子模,且$X$和$T_{\ell}/X$都是秩为$1$的自由$\mathbb{Z}_{\ell}$-模. 令$n\geq 0$,记$X(n)$是$X$在$T_{\ell} / \ell^n T_{\ell}$中的像. $X(n)$是$E[\ell^n]$的子模,而且是阶为$\ell^n$的循环群. 那么$X(n)$是$E$的$K$-代数子群. 令$E(n) = E/X(n)$,那么$E\to E(n)$是$K$-同源,且核为$\ell^n$阶循环群. 上面的引理说明不同的$n$对应的$E(n)$互不同构,与Shafarevich的定理\ref{galois::isogeny_finite_curves}矛盾.
    % TODO: quotient of elliptic curves?

    (ii) 如果$E[\ell]$中有一维不变子空间$X_{\ell}$,则类似(i)中的情况,存在同源$E\to E/X_{\ell}$. 由引理,对于不同的$\ell$的取值,$E/X_{\ell}$互不同构. 如果存在无穷多个$\ell$使得$E[\ell]$不可约,则有无穷多条$E/X_{\ell}$,与Shafarevich的定理矛盾.
\end{proof}



先假设$\rho$是任意的有理$\ell$进表示,其像是$\mathrm{GL}_n(\mathbb{Z}_{\ell})$的开子群.

\begin{cprop}
    令$S_0$为满足以下条件的素点$v$的集合:

    (i) $\rho$在$v$处非分歧,且$F_{v,\rho}$的特征多项式的是有理系数的;

    (ii) $\mathrm{Tr}(F_{v,\rho}) = 0$.

    则$S_0$的密度为$0$.
\end{cprop}

\begin{proof}
    令$H = \{s\in \mathrm{GL}_n(\mathbb{Z}_{\ell})\mid \mathrm{Tr}(s)=0\}$,
    $H' = \mathrm{Im}\ \rho \bigcap H$.
    由于$H'$的维数严格小于$n^2$,$H'$的Haar测度为$0$,而$\mathrm{Im}\ \rho$的Haar测度不为$0$.

    记$H'_n$为$H$在$\mathbb{GL}_2(\mathbb{Z}/\ell^n \mathbb{Z})$上的投影,
    $G_n$为$\mathrm{Im}\ \rho$相应的投影. 那么$\lim\limits_{n\to \infty} \frac{\absn{H'_n}}{\absn{G_n}} = 0$.
    由Chebotarev密度定理,$S_0$的密度不大于每个$\frac{\absn{H'_n}}{\absn{G_n}}$,
    即$S_0$的密度为$0$.
\end{proof}

特别地,如果$\rho$是椭圆曲线$E$定义的表示,且$E$在$v$处有好约化.

对几乎所有的$v$,$\mathrm{Tr}(F_{v,\rho})=0$就等价于$E$在$v$处的约化高度为$2$.
因此如果证明了没有复乘的椭圆曲线的$\rho_{\ell}$的像是开子群,就有了
\begin{ccor}
    设$E$是没有复乘的椭圆曲线,则$E$有高度为$2$的好约化的素点$v$的密度为$0$.\label{height2::sparse}
\end{ccor}



    \chapter{Galois表示的像是开子群}

有了关于$\ell$进表示和椭圆曲线的准备工作之后,我们可以开始证明第一个主定理了.
在定理的证明过程中实质上是用到了$\mathrm{GL}_2$的子代数的分类,在这里体现为$\mathfrak{gl}_2$的子代数的分类.
在李代数的框架下,这是很简单的.

还是设$K$是代数数域,$E$是定义在$K$上的椭圆曲线.

\begin{cthm}
    如果$E$定义的$\ell$进表示$\rho_{\ell}$满足:存在理想$\mathfrak{m}$和$\mathbb{Q}$代数群的态射$\phi: S_{\mathfrak{m}}\to \mathrm{GL}(V)$使得$\rho_{\ell} = \phi_{\ell}$. 那么$E$在$K$上有复乘.
\end{cthm}

\begin{proof}
    假设$E$在$K$上没有复乘.
    $\phi$定义了有一族严格相容的Galois表示$\{\phi_{\ell'}\}$.
    由定理\ref{reps::finite_diagonal},可以选取一个$\ell'$使得$\phi_{\ell'}$在$\mathbb{Q}_{\ell'}$上就可以对角化. 
    由于$\phi_{\ell}$和$\rho_{\ell}$相容,$\phi_{\ell'}$与$\rho_{\ell}$也相容.
    $\rho_{\ell'}$是不可约的,而已知$\phi_{\ell}$是半单的. 那么$\phi_{\ell'}\cong \rho_{\ell'}$.
    即$\rho_{\ell'}$是对角化的,但这与不可约性矛盾.
\end{proof}

\begin{ccor}
    如果$E$定义的$\ell$进表示$\rho_{\ell}$是局部代数的,则$E$在$K$上有复乘. \label{reps::when_cm}
\end{ccor}

\begin{ccor}
    如果$E$定义的$\ell$进表示是交换的,则$E$在$K$上有复乘.
\end{ccor}

由于Galois群是紧的,$G_{\ell} = \mathrm{Im}(\rho_\ell)$是$\mathrm{GL}_2(\mathbb{Q}_{\ell})$的闭子群. 由非阿基米德的Cartan定理,$\mathrm{Im}(\rho_{\ell})$是一个$\ell$进李群. 令$\mathfrak{g}_{\ell}$为其李代数.

\begin{proof}[定理\ref{main::open_image}的证明]
    只需要证明$\mathfrak{g}_{\ell} = \mathrm{End}(V_{\ell})$.

    $E$在$K$的所有有限扩张上都没有复乘,从而$G_{\ell}$的任何开子群$U$,$V_{\ell}$都是不可约的$U$模. 那么,$V_{\ell}$是不可约的$\mathfrak{g}_{\ell}$模. 由Schur引理,中心化子$C(\mathfrak{g}_{\ell})$是一个域. 但是$\mathrm{dim}\ V_{\ell}=2$,所以$C(\mathfrak{g}_{\ell})$或者是$\mathbb{Q}_{\ell}$,或者是$\mathbb{Q}_{\ell}$的二次扩张.

    如果$C(\mathfrak{g}_{\ell}) = \mathbb{Q}_{\ell}$,则$\mathfrak{g}_{\ell}$或者是$\mathrm{End}(V_{\ell})$,或者是$\mathfrak{sl}(V_{\ell})$. 但如果$\mathfrak{g}_{\ell} = \mathfrak{sl}(V_{\ell})$,$\mathfrak{g}_{\ell}$在$\wedge^2 V_{\ell}$上的作用是平凡的. 但是由Weil配对,$\wedge^2 V_{\ell}$作为Galois模与$T_{\ell}(\mu)\otimes \mathbb{Q}_{\ell}$同构,矛盾.

    如果$C(\mathfrak{g}_{\ell}) = F$是$\mathbb{Q}_{\ell}$的一个二次扩张,则
    $\mathfrak{g}_{\ell}\subset C(F)) = F$是交换的. 那么$G_{\ell}$有一个交换的开子群.
    取$L$为$K$的有限扩张,使得$\rho_{\ell}$在$\mathrm{Gal}(\overline{K}/L)$上的限制是交换的.
    此时$E$在$L$上有复乘.

\end{proof}



    \section{模表示和满射性质}


    \printbibliography[heading=bibliography,title=参考文献]
\end{document}