\documentclass[a4paper, UTF8, CJKmath=true, fontset=macnew, zihao=5]{ctexart}

\usepackage{mathtools}
\usepackage{graphicx}
\usepackage{amsthm}
\usepackage{mathrsfs}
\usepackage{amsfonts}
\usepackage{amssymb}
\usepackage{hyperref}
\usepackage{tikz-cd}
\usepackage{float}

\usepackage{xcolor}

\usepackage[left=3.17cm,right=3.17cm,top=2.54cm,bottom=2.54cm]{geometry}

%\setmainfont[
%    BoldFont={HYXuanSong-75S},
%]{HYXuanSong}
%\setCJKmainfont[
%    BoldFont={HYXuanSong-75S},
%]{HYXuanSong}

\usepackage[backend=biber,style=gb7714-2015]{biblatex}
\addbibresource[location=local]{main.bib}

\theoremstyle{definition}

\newtheorem{cthm}{定理}
\newtheorem{cprop}{命题}
\newtheorem{crem}{注记}
\newtheorem{cdef}{定义}
\newtheorem{ccor}{推论}
\newtheorem{clem}{引理}
\newtheorem{cyyy}{白日梦}

\newenvironment{pppmatrix}{\left(\begin{matrix}}{\end{matrix}\right)}

\newcommand{\paten}[1]{\left(#1\right)}
\newcommand{\deri}{\mathrm{d}}
\newcommand{\I}{\mathrm{i}}
\newcommand{\modular}{\mathcal{M}}
\newcommand{\cusp}{\mathcal{S}}
\newcommand{\diam}[1]{\big<#1\big>}
\newcommand{\matbt}[4]{\begin{pppmatrix}#1 &#2\\#3 &#4\end{pppmatrix}}
\newcommand{\absn}[1]{\left|#1\right|}
\newcommand{\order}{\mathcal{O}}
\newcommand{\Chebotarev}{Chebotarev}
\newcommand{\modulus}{理想}

\newcommand{\invlim}{\lim\limits_{\longleftarrow}}

\newcommand{\basechange}[2]{{#1}_{/#2}}
\newcommand{\tame}{温和分歧}
\newcommand{\wild}{非温和分歧}

\setcounter{tocdepth}{2}

%\setlength{\parindent}{0em}

\title{椭圆曲线定义的Galois表示的像}
\author{褚晓敏}
\date{\today}

\begin{document}
    \maketitle

    \begin{abstract}
        本文关于椭圆曲线的挠点定义的Galois表示的像的综述.
    \end{abstract}

    \tableofcontents

    \clearpage

    \chapter{引言}

志村五郎在\parencite{shimura1966reciprocity}中讨论了由椭圆曲线$E$的$\ell$阶点生成的数域$L/\mathbb{Q}$,
其中$\ell$是一个素数.
志村对$E=X_0(11)$和$7\leq \ell\leq 97$证明了
是$\mathrm{Gal}(L/\mathbb{Q})\cong \mathrm{GL}_2(\mathbb{F}_{\ell})$.

同时由熟知的结果,Frobenius元素的特征多项式能够由椭圆曲线的几何性质完全决定(本文将在第三章中证明这些结果).
对$\tau\in\mathrm{GL}_2(\mathbb{F}_{\ell})$中,特征多项式能几乎确定其共轭类:
如果特征多项式没有重根,那么$\tau$共轭于$\matbt{a}{0}{0}{b}$,其中$a,b$是特征多项式的两个根
(如果$a,b\in \mathbb{F}_{\ell^2} - \mathbb{F}_{\ell}$,则需要再共轭到一个$\matbt{c}{\sigma d}{d}{c}$,其中$\sigma\in\mathbb{F}_{\ell}^{\times} - (\mathbb{F}_{\ell}^{\times})^2;c, d\in\mathbb{F}_{\ell}$);
而如果特征多项式有重根$a$,则$\tau$或者共轭于$\matbt{a}{1}{0}{a}$,或者就是$\matbt{a}{0}{0}{a}$.

那么,假设$\mathrm{Gal}(L/\mathbb{Q})\cong \mathrm{GL}_2(\mathbb{F}_{\ell})$成立,
则素数在$L$中的分解情况几乎可以由椭圆曲线的几何性质决定.
于是我们可以构造一大类扩张,其Galois群是有限李型群,而且可以几乎确定素数在其中的分解.

同时,$X_0(11)$是模曲线,其几何性质被编码在一个模形式
\begin{equation}
    f(z) = q\prod_{n=1}^{\infty} \paten{1-q^n}^2 \paten{1 - q^{11n}}^2,q=\exp(2\pi i z)
\end{equation}
的Fourier系数中. 那么,$L$中素数的分解性质就(几乎,还差确定幂零共轭类)被编码在了一个模形式中.
这一事实可以被看作是某种非交换的类域论. 如果不是取椭圆曲线的有限阶点,
而是取乘法群$\mathbb{G}_m$的有限阶点,即单位根,
就得到了经典的类域论.

要完全实现上面的想法,需要解决以下的问题:
\begin{enumerate}
    \item 什么时候$L/\mathbb{Q}$的Galois群是$\mathrm{GL}_2(\mathbb{F}_{\ell})$?
    \item 以上讨论能否推广到一般的基域$K$?
    \item 什么时候椭圆曲线$E$是模曲线?
\end{enumerate}
问题3就是模性问题. 问题2的回答是肯定的,问题1和椭圆曲线的复乘有关,我们下面进行详细的分析.

在开始之前,还可以提出几个问题:
\begin{enumerate}
    \item 如果添加上$E$的合数阶点会发生什么?
    \item 在Frobenius元素的特征多项式有重根时,如何确定它是不是幂零的?
    \item 对高维的Abel簇是否存在类似的理论?
    \item 更一般地,能否构造Galois群是$G(\mathbb{F}_{q})$的扩张,其中$G$是Chevalley群,$q$是素数的幂,并确定其中的Frobenius元素?
\end{enumerate}

问题1是比较简单的,添加合数阶点就是添加所有的素数幂阶点,我们下面分析时会讨论素数幂阶点的情况.
问题2-4比较复杂,超出了本文讨论的范围.

\vskip0.3cm

现在开始正式的讨论.
添加椭圆曲线的有限阶点得到的扩张的Galois群就是绝对Galois群在有限阶点上的作用的像,
接下来我们采用更方便的Galois表示的语言.

设$E$是定义在域$K$上的椭圆曲线,$\ell$是素数且$\ell\neq \mathrm{char}\ K$. 对所有的正整数$n$,$E[\ell^n]$是$E(\overline{K})$的有限子群,且同构于$(\mathbb{Z}/\ell^n\mathbb{Z})^2$. $G = \mathrm{Gal}(\overline{K}/K)$作用在$E[\ell^n]$上,且作用与$E$上的加法交换,即有表示$\rho_n:G\to \mathrm{GL}_2(\mathbb{Z}/\ell^n \mathbb{Z})$.

记$K_n=K(E[\ell^n])$. $\sigma\in G$在$K_n$上作用平凡当且仅当$\sigma$在$E[\ell^n]$上的作用平凡,因此$K_n/K$是Galois扩张,且$\mathrm{Gal}(K_n/K)=\mathrm{Im}\ \rho_n$. $\mathrm{Im}\ \rho_n$的大小反映了$K_n$的复杂程度,也就是$E$的有限阶点之间相互独立的程度.

可以将所有的$\rho_n$放在一起考虑. 记$E[\ell^{\infty}]=\bigcup_n E[\ell^n]$,$K_{\infty} = K(E[\ell^{\infty}])$. $K_{\infty}$也是Galois扩张,且$\mathrm{Gal}(K_{\infty}/K) = \lim\limits_{\longleftarrow} \mathrm{Gal}(K_n/K)$. 记$\rho = \lim\limits_{\longleftarrow} \rho_n$,$\rho: G\to \mathrm{GL}_2(\mathbb{Z}_{\ell})$,其中转移映射$\mathrm{GL}_2(\mathbb{Z}/\ell^{n+1} \mathbb{Z}) \to \mathbb{Z}/\ell^n \mathbb{Z}$,$\tau \mapsto \tau \pmod{\ell^n}$.

Tate提出了另一种将$\rho_n$放在一起的方式. 令Tate模$T_{\ell}(E) = \lim\limits_{\longleftarrow} E[\ell^n]$,其中转移映射为$E[\ell^{n+1}]\to E[\ell^n], P\mapsto \ell P$,则$T_{\ell}\cong (\mathbb{Z}_{\ell})^2$. $G$在$E[\ell^n]$上的作用还与椭圆曲线上的加法交换,因此在同构$T_{\ell}(E)\cong (\mathbb{Z}_{\ell})^2$下$G$的作用是线性映射. 这就得到了表示$G\mapsto \mathrm{GL}_2(\mathbb{Z}_{\ell})$. 可以看出Tate模定义的表示在模每个$\ell^n$之后都成为$\rho_n$,因此就是上一段中定义的$\rho$.
记$V_{\ell} = T_{\ell}\otimes \mathbb{Q}_{\ell}$.

在$\mathrm{char}\ K = 0$,且$E$具有复乘的情况下,$\mathrm{Im}\ \rho$是相对简单的. 通过将$K$取成一个有限扩张,不妨设$\mathrm{End}_K(E) \cong \mathcal{O}$. 此时$T_{\ell}$是$\mathcal{O}_{\ell} = \lim\limits_{\longleftarrow}\mathcal{O}/\ell^n \mathcal{O}$上秩为$1$的模,$G$的作用与$\mathcal{O}_{\ell}$的作用交换. 此时$\rho$实际上是表示$G\to (\mathcal{O}_{\ell})^{\times}$,即$\mathrm{Im}\ \rho$是交换群.

如果$K$是代数数域,绝对Galois群的表示的像似乎遵从一个原则,“如果没有让像变小的理由,那么它就会尽可能地大”. $E$具有复乘就是一个让$\rho$的像变小的原因. Serre证明了复乘就是仅有的让像变小的原因,也就是本文的第一个主定理:

\begin{cthm}
    令$K$是代数数域,$E$是定义在$K$上的椭圆曲线. 假设$\mathrm{End}_{\overline{K}}(E) \cong \mathbb{Z}$. 则$\mathrm{Im}\ \rho$是$\mathrm{GL}_2(\mathbb{Z}_{\ell})$的开子群. \label{main::open_image}
\end{cthm}

定理\ref{main::open_image}的条件和结论都可以差一个有限扩张.
但既然证明了像是开子群,有一个自然的问题就是它是否是整个$\mathrm{GL}_2(\mathbb{Z}_{\ell})$.
这当然是不一定的. 就算$\rho_{\ell}$的像是$\mathrm{GL}_2(\mathbb{Z}_{\ell})$,
只要取一个真开子群$G'\subset \mathrm{GL}_2(\mathbb{Z}_{\ell})$,
并令$K' = K(E[\ell^{\infty}])^{G'}$,则$K'/K$是有限扩张且定义在$K'$上的$\ell$进表示
$\rho_{\ell}'$的像就不是$\mathrm{GL}_2(\mathbb{Z}_{\ell})$了.

但无论如何,还可以继续追问什么时候$\rho_{\ell}$的像是整个$\mathrm{GL}_2(\mathbb{Z}_{\ell})$.
首先,如果$\rho_{\ell}$的像是$\mathrm{GL}_2(\mathbb{Z}_{\ell})$,
则$\tilde{\rho} = \rho \pmod{\ell}$也是满射.
因此可以先问什么时候$\tilde{\rho}$是满射. 这就是第二个主定理

\begin{cthm}
    $K, E$同上. 对于除了有限多个以外的素数$\ell$,模$\ell$Galois表示$\tilde{\rho}_{\ell}: \mathrm{Gal}(\overline{K}/K)\to \mathrm{GL}_2(\mathbb{F}_{\ell})$是满射. \label{main::surjective}
\end{cthm}

事实上,由几乎所有的$\tilde{\rho}_{\ell}$是满射,可以推出几乎所有的$\rho_{\ell}$都是满射.

两个主定理的证明思路是类似的,就是分类出$\mathrm{GL}_2$的子群,再一个一个排除. 这里$\mathrm{GL}_2$的模糊表述是故意的,同时指$\mathrm{GL}_2(\mathbb{Q}_{\ell})$和$\mathrm{GL}_2(\mathbb{F}_{\ell})$. 两者一个通过李代数的方法分类,另一个是有限群的子群的分类.

而另一方面,排除一个子群的可能性就需要控制$\rho_{\ell}$的行为. 具体来说,在$\ell$变化时,$\rho_{\ell}$遵守很强的相容性条件,$\rho_{\ell}$的性质可以通过分析其它$\rho_{\ell'}$得到,也就是说可以随时找一个方便的$\ell$在研究$\rho_{\ell}$的性质. 进一步地,相容性条件还允许结合无穷多个$\rho_{\ell}$的力量,反过来对椭圆曲线做出限制. 事实上,本文中最重要的定理可能是

\begin{cthm}
    如果$\rho_{\ell}$的像是交换的,则$E$在$K$上有复乘.
\end{cthm}

在证明以上三个定理的过程中,Serre发现了一类代数群$S_{\mathfrak{m}}$. 所有带复乘的椭圆曲线定义的$\rho_{\ell}$实际上都是从$S_{\mathfrak{m}}$的代数表示得到的. 而且逆命题也是对的,即如果$\rho_{\ell}$是从$S_{\mathfrak{m}}$的代数表示得到的,则$E$有复乘. 或者说$S_{\mathfrak{m}}$就刻画了复乘现象.

有了定理\ref{main::surjective},自然就可以进一步地问例外的素数可以有多少个.
记$\ell_{E}$是最大的使得$\tilde{\rho}_{\ell}$不是满射的素数.
Serre在\parencite{serre1981quelques}中证明了,如果$E/\mathbb{Q}$没有复乘,记$N_E$是所有$E$有坏约化的素数的乘积.
假设GRH,有
\begin{equation}
    \ell_{E} = O((\log N_E)(\log \log 2N_E)^3)
\end{equation}

猜测$\ell_{E}$仅与基域$K$有关,且$\ell_{\mathbb{Q}}=37$.
这被称为Serre的一致性猜想(Serre's uniformity conjecture).

本文其余部分的组织如下:
第二章定义$\ell$进Galois表示和严格相容的有理表示族的概念,
并证明一族严格相容的有理表示只要是交换的,就一定可以被某个代数群$S_{\mathfrak{m}}$刻画;
第三章证明一些椭圆曲线的Tate模的结论;
第四章证明第一个主定理,即无复乘时,$\ell$进表示$\rho_{\ell}$的像是开子群;
第五章证明第二个主定理,即无复乘时,对几乎所有的素数$\ell$,模$\ell$表示$\tilde{\rho}_{\ell}$是满射;
第六章不加证明地介绍一些对$\ell_{E}$的估计的结果.


    \chapter{\texorpdfstring{$\ell$}{ELL}进Galois表示}

\section{一些背景知识}

如果$G$是$k$上的仿射代数群,$k'$是$k$的扩张,则记$\basechange{G}{k'}$为$G$通过基域扩张到$k'$得到的$k'$上的代数群.

现在来回顾一些代数数论和代数群理论的内容,用以固定后面要用到的记号.

\subsection{乘法型代数群}

命题\ref{reps::split_finite}的证明需要用到可对角化群和乘法型群的概念.
假设$k$是一个特征$0$的域,以下讨论的代数群都是定义在$k$上的仿射代数群.

\paragraph*{可对角化群}

\begin{cprop}
    对任意有限生成交换群$M$(群运算用乘法记),函子$D(M): R\to \mathrm{Hom}(M, R^{\times})$是可由$\mathrm{Spec}\ k[M]$表出.
\end{cprop}

\begin{proof}
    要给出一个$k$代数的态射$k[M]\to R$就是给出一个群同态$M\to R^{\times}$.
\end{proof}

\begin{cdef}
    若一个代数群$G$同构于某个$D(M)$,则称$G$为可对角化的.
\end{cdef}

\begin{cprop}[{\parencite[][p. 233]{milne2017algebraic}}]
    $M\to D(M)$和$X: G\mapsto \mathrm{Hom}_k(G, \mathbb{G}_m)$互为拟逆,定义了可对角化代数群和有限生成交换群范畴的反变等价,且$D,X$都是正合的.
\end{cprop}

%\begin{proof}
%    \parencite[][p. 233]{milne2017algebraic} % Theorem 12.9
%\end{proof}

\begin{cprop}[{\parencite[][p. 234]{milne2017algebraic}}]
    如果$G$可对角化,则$G$的任何有限维表示都(在同一个域$F$上)可对角化,即可以通过$\sigma\in\mathrm{GL}_n(k)$的共轭成为对角矩阵.
\end{cprop}

%\begin{proof}
%    \parencite[][p. 234]{milne2017algebraic} % Theorem 12.12.
%\end{proof}

\paragraph*{乘法型群}

\begin{cdef}
    若$G$在$k$的某个扩张上同构于有限个$\mathbb{G}_m$的乘积,则称$G$是环面群. 如果在$k$上就已经是同构的,则称$G$是分裂的环面群.

    如果$G$在$k$的某个扩张下可对角化,则称$G$为乘法型的.
\end{cdef}

既然可对角化群的范畴与有限Abel群的范畴反变等价,
自然可以想到乘法型群的范畴与有限Abel群的范畴加上某种额外的结构之后是反变等价的.
由于乘法型群和可对角化群的差别在于需要一个域扩张,额外的结构其实就是Galois群的作用.

具体来说,令$\Gamma = \mathrm{Gal}(\overline{k}/k)$,
$X^{*}(T) = \mathrm{Hom}_{\overline{k}}(\basechange{T}{\overline{k}}, \basechange{\mathbb{G}_m}{\overline{k}})$.
$\Gamma$自然地连续作用在$X^{*}(G)$上.
\begin{cprop}[{\parencite[][p. 240]{milne2017algebraic}}]
    $X^{*}$是乘法型群范畴到带$\Gamma$连续作用的有限Abel群范畴的反变等价,
    且在这个等价中正合列对应到正合列.
\end{cprop}

\begin{cprop}
    如果$G$是乘法型群,则$G$在$k$的某个有限扩张上可对角化.
\end{cprop}

\begin{cprop}[{\parencite[][p. 239]{milne2017algebraic}}]
    如果$G',G''$都是乘法型群,$G$是交换代数群,且有正合列
    \begin{equation}
        1\to G'\to G\to G''\to 1
    \end{equation}
    则$G$也是乘法型群.
\end{cprop}

%\begin{proof}
%    {\parencite[][p. 239]{milne2017algebraic}} % Theorem 12.22
%\end{proof}

\begin{cprop}[{\parencite[][p. 240]{milne2017algebraic}}]
    如果$G$是乘法型群,那么$G$是群扩张
    \begin{equation}
        1\to G'\to G\to G''\to 1
    \end{equation}
    其中$G'$是环面群,$G''$是有限乘法型群.
\end{cprop}

%\begin{proof}
%    \parencite[][p. 240]{milne2017algebraic} % Corollary 12.24
%\end{proof}


\subsection{代数数论}

令$K$是代数数域,$\mathfrak{m}$是$\order_K$的整理想. 记$U_{\mathfrak{m}, v}$为:当$K_v \cong \mathbb{C}$时,$K_v^{\times}$;当$K_v\cong \mathbb{R}$时,$K_v^{+}$;当$v\nmid\mathfrak{m}\infty$时,$U_{v}$;当$v(\mathfrak{m})=t>0$时,$1+\mathfrak{p}_v^{t}$. 

记$U_{\mathfrak{m}}=\prod_v U_{\mathfrak{m}, v}$,$U_{\mathfrak{m}, finite} = \prod_{v\nmid \infty}U_{\mathfrak{m}, v}$(看作$\mathbb{A}_K^{\times}$的子群时,指要求$v\mid \infty$处取值为$1$).
令$E_{\mathfrak{m}} = K^{\times} \bigcap U_{\mathfrak{m}}$,$I_{\mathfrak{m}} = \mathbb{A}_K^{\times}/U_{\mathfrak{m}}$,$C_{\mathfrak{m}} = \mathbb{A}_K^{\times} / K^{\times}U_{\mathfrak{m}}$.

$T = \mathrm{Res}_{K/\mathbb{Q}}(\basechange{\mathbb{G}_m}{K})$,$T_{\mathfrak{m}} = T / \overline{E_{\mathfrak{m}}}$,其中$\overline{E_{\mathfrak{m}}}$是$E_{\mathfrak{m}}$在$T$中的Zariski闭包.

设$\sigma_1,\ldots,\sigma_n$是$K$到$\overline{Q}$的域嵌入,
则$\basechange{T}{\overline{Q}}\to \basechange{\mathbb{G}_m}{\overline{\mathbb{Q}}}\times \basechange{\mathbb{G}_m}{\overline{\mathbb{Q}}} \times \cdots \times \basechange{\mathbb{G}_m}{\overline{\mathbb{Q}}}$,$x\otimes \alpha \mapsto (\sigma_1(x)\alpha, \sigma_2(x)\alpha, \ldots,\sigma_n(x)\alpha)$是同构.
$T$是一个环面群,由上一小节中的一般理论,这时有两个正合列
\begin{equation}
    0 \rightarrow \overline{E_{\mathfrak{m}}} \rightarrow T \rightarrow T_{\mathfrak{m}}\rightarrow 0
\end{equation}
\begin{equation}
    0 \leftarrow X^{*}(E_{\mathfrak{m}}) \leftarrow X^{*}(T) \leftarrow X^{*}(T_{\mathfrak{m}}) \leftarrow 0
\end{equation}

例如,当$K$是二次域时,$T$是二维环面群. 记$1,\sigma$为$K$的自同构,也就是$K$到$\overline{\mathbb{Q}}$的嵌入.
那么$X^{*}(T)\cong \mathbb{Z}^2$.
实际上$\basechange{T}{K}$已经分裂了,$X^{*}(T)$中的元素可以写成$x\mapsto x^{n_1}\cdot \sigma(x)^{n_2}$的形式.
$T_{\mathfrak{m}}$的特征标就是一对整数$(n_1,n_2)$,
满足当$x\in E_{\mathfrak{m}}$时,$x^{n_1}\cdot \sigma(x)^{n_2}=1$.
如果$K$是虚二次域,$E_{\mathfrak{m}}$是有限群,此时$T_{\mathfrak{m}}$还是二维环面群;
如果$K$是实二次域,$E_{\mathfrak{m}}$是秩为$1$的自由交换群,此时$T_{\mathfrak{m}}$是一维环面群.

先给出Hecke特征的理想版本和idèle版本一一对应的精确形式:

\begin{cthm}
    以下两类对象之间存在一一对应:

    (1) $\chi: \mathbb{A}_K^{\times}/K^{\times} \to \mathbb{C}^{\times}$,满足$\chi(U_{\mathfrak{m}, finite}) = 1$

    (2) $\tilde{\chi}: J^{\mathfrak{m}}\to \mathbb{C}^{\times}$,使得存在$\chi_{\infty}:\prod_{v\mid \infty}K_v^{\times} \to \mathbb{C}^{\times}$,满足当$a\in K, a\equiv 1\pmod{\mathfrak{m}}$时,$\tilde{\chi}((a)) = \chi_{\infty}(a)$.

    其中映射$\chi\mapsto \tilde{\chi}$是,对所有$v\nmid \mathfrak{m}\infty$,选定一个素元$\pi_v\in \mathcal{O}_v$得到映射$J^{\mathfrak{m}}\to \mathbb{A}_K^{\times}$,再与$\chi$复合.
\end{cthm}

\begin{crem}
    在(2)中对$a$的要求再加上全正,对$\tilde{\chi}$的要求看上去是减弱了,但事实上是一样的. 可以通过强逼近定理转化到idèle版本再还原来看,也可以直接证明:令$K^{\mathfrak{m}} = \{\alpha\in K^{\times}, \alpha\equiv 1\pmod{\mathfrak{m}}\}$,$K^{\mathfrak{m}}_{+} = \{\alpha\in K^{\mathfrak{m}}, \alpha 全正\}$. 如果$\chi_{\infty}$满足当$a\in K^{\mathfrak{m}}_{+}$时,$\tilde{\chi}((a)) = \chi_{\infty}(a)$,则$\tilde{\chi}\chi_{\infty}^{-1}$定义了$K^{\mathfrak{m}}/K^{\mathfrak{m}}_{+}$的一个特征标,但是则$K^{\mathfrak{m}}/K^{\mathfrak{m}}_{+} \cong \prod_{v实素点} \{\pm 1\}$. 因此存在一个$\phi: \prod_{v\mid \infty} K_v^{\times}\to \{\pm 1\}$使得在$K^{\mathfrak{m}}$上有$\tilde{\chi}\chi_{\infty}^{-1} = \phi$,于是$\chi_{\infty}\phi$满足定理中(看起来更强的)条件(2).
\end{crem}


\subsection{代数群的表示}

令$G$是仿射交换代数群.

设$\Lambda = \mathscr{O}(G)$,$\overline{\Lambda} = \Lambda \otimes \overline{k} = \mathscr{O}(\basechange{G}{\overline{k}})$.
每个$\chi\in X^{*}(G)$都可以看作是$\overline{\Lambda}$的元素.
那么有映射$\alpha: \overline{k}[X^{*}(G)] \to \overline{\Lambda}$,其中$\overline{k}[X^{*}(G)]$是$X^{*}$的群代数.(这里$X^{*}(G)$用乘法记号)

$\Gamma = \mathrm{Gal}(\overline{k}/k)$通过$s(\sum a_{\chi} \chi) = \sum s(a_{\chi}) s(\chi)$作用在$\overline{k}[X^{*}(H)]$上.
通过直接计算可以验证$\Gamma$的作用和$\alpha$交换.
当$G$是乘法型群时,$\basechange{G}{\overline{k}}$是对角化群,此时$\alpha$是双射.

设$V$是$k$上的有限维线性空间,$\phi: G\to \mathrm{GL}(V)$是一个半单表示.
记$\theta_{\phi}$为$\phi$的迹$\theta_{\phi} = \sum n_{\chi}(\phi)\chi$,
其中$n_{\phi}(\chi)$是$\phi$在$\overline{k}$上的分解中$\chi$的重数.
对于任意的(交换)$k$代数$R$和$h\in G(R)$,都有$\theta_{\phi}(h) = \mathrm{Tr}(\phi(h))$.

记$\mathsf{Rep}_{k}(G)$为$G$的半单表示的范畴. 如果$k'/k$是域扩张,
那么通过基域扩张可以得到映射$\mathsf{Rep}_{k}(G)\to \mathsf{Rep}_{k'}(\basechange{G}{k'})$.
称$V\in \mathsf{Rep}_{k'}(\basechange{G}{k'})$可以定义在$k$上是指$V$落在基域扩张的像(essential image)中.

\begin{cprop}
    $\phi\mapsto \theta_{\phi}$是$\mathsf{Rep}_{k}(G)$的同构类和$S \subset \mathbb{Z}[X^{*}(G)]$之间的双射,
    其中$S$包含了满足以下条件的元素$\theta$:

    (i) $\theta$在$\Gamma$的作用下不变;

    (ii) 所有的$n_{\chi}\geq 0$.
\end{cprop}

\begin{ccor}
    $\phi'\in \mathsf{Rep}_{k'}(\basechange{G}{k'})$可以定义在$k$上当且仅当$\theta_{\phi'}\in \Lambda \otimes k'$属于$\Lambda$.
\end{ccor}


\section{\texorpdfstring{$\ell$}{ELL}进表示的基本定义}

令$\ell$是一个素数,$k$是一个域,$k_s$是$k$的一个可分闭包. 记$G = \mathrm{Gal}(k_{s}/k)$.

\begin{cdef}
    $k$的一个$\ell$进表示是一个连续的同态$\rho: G\to GL(V)$,其中$V$是一个有限维$\mathbb{Q}_{\ell}$向量空间.
\end{cdef}

令$K$是一个代数数域. 记$\Sigma_K$是$K$的有限素点的集合.

\begin{cdef}
    设$v\in \Sigma_K$. 称一个$\ell$进表示$\rho$在$v$处非分歧,当且仅当对某一个$K_{s}$的素点$w\mid v$(等价于对所有的$w\mid v$)有$\rho(I_w) = 1$,$I_w$是惯性子群. 如果$\rho$在$v$处非分歧,则$D_w$中所有Frobenius元素在$\rho$下的像是同一个,称其为$\rho$下的Frobenius元素,记为$F_{w, \rho}$. 所有$w\mid v$定义的Frobenius元素形成$G$的一个共轭类,称为$v$处的Frobenius共轭类.
\end{cdef}

\begin{crem}
    如果$L$是$\ker \rho$的固定域,则$\rho$在$v$处非分歧当且仅当$L$在$v$处非分歧.
\end{crem}

我们需要\Chebotarev 密度定理的一个无穷版本,这里的密度指自然密度:

\begin{cthm}
    令$L$是$K$的Galois扩张,只在有限个素点处分歧. 给$H = \mathrm{Gal}(L/K)$赋Krull拓扑,令$\mu$是相应的Haar测度(注意到紧群都是幺模的),满足$\mu(H) = 1$,那么:

    (1) $L$的非分歧素点定义的Frobenius元素在$H$中稠密

    (2) 令$X$是$H$的一个在共轭下不变的子集,并假设$X$是开集或者$X$是闭集或者$X$的边界集在$\mu$下是零测集. 那么使得Frobenius元素落在$X$中的非分歧素点$v$的密度恰好为$\mu(X)$.
\end{cthm}

\begin{proof}
    只要证明(2). 令$K\subset L_1\subset L_2\subset \cdots \subset L$使得$L = \bigcup_i L_i$. 记$p_i: H\to \mathrm{Gal}(L_i/K)$为自然的投射. 如果$F$是闭集,则$F = \bigcap_i p_i^{-1}(p_i(F))$,因此由有限扩张的\Chebotarev 密度定理,结论对$F$成立. 如果$U$是开集,且$U$在共轭下不变,则$H-U$是在共轭下不变的闭集,结论对$H-U$成立,从而对$U$也成立. 取$F = \overline{X}, U = H - \overline{H-X}$. 但是$\mu(F) = \mu(U)$,因此结论对$X$成立.
\end{proof}

\begin{crem}
    如果$\rho$是一个$\ell$进表示,且在有限多个素点以外非分歧. 令$L$是$\ker(\rho)$的固定域,则$L$也在有限多个素点以外非分歧. 由\Chebotarev 密度定理,Frobenius元素在$\mathrm{Im}(\rho)$中稠密.
\end{crem}

\section{有理\texorpdfstring{$\ell$}{ELL}进表示}

令$\rho$是一个$\ell$进表示,$v$是一个素点使得$\rho$在$v$处非分歧. 记$P_{v, \rho}(T) = \det (1 - F_{w, \rho}T)$,其中$w\mid v$.

\begin{cdef}
    称$\rho$为一个有理$\ell$进表示当且仅当存在有限集合$S\subset \Sigma_K$使得当$v\in \Sigma - S$时,$\rho$在$v$处非分歧,而且$P_{v, \rho}$的系数都是有理数. 如果进一步的,$P_{v,\rho}$的系数都是整数,则称$\rho$为整$\ell$进表示.
\end{cdef}

\begin{cdef}
    令$\ell'$为一个素数(不一定和$\ell$不同),$\rho'$为$K$的$\ell'$进表示. 假设$\rho, \rho'$都是有理的. 称$\rho, \rho'$相容当且仅当存在有限集合$S\subset \Sigma_K$使得$\rho, \rho'$在所有$v\in \Sigma_K - S$处都非分歧,且$P_{v,\rho} = P_{v, \rho'}$.
\end{cdef}

令$\rho$为一个$\ell$进表示,$V$是表示空间. 则$V$有合成序列$0 = V_0\subset \cdots \subset V_q = V$,其中$V_{i+1}/V_i$是单$G$模. 令$V' = \bigoplus_i V_{i+1}/V_i$,则$V'$上定义了半单的有理$\ell$进表示,且和$\rho$相容. 称这个表示为$\rho$的半单化.

\begin{cdef}
    假设对每个素数$\ell$有一个有理$\ell$进表示$\rho_{\ell}$. 称$\{\rho_{\ell}\}$为严格相容的,当且仅当存在有限集合$S\subset \Sigma_K$满足:

    (1) 对任意的素数$\ell$,记$S_{\ell} = \{v\mid p_v = \ell\}$. 对所有的$v\in \Sigma_K - S\bigcup S_{\ell}$,$\rho_{\ell}$在$v$处非分歧,且$P_{v, \rho_{\ell}}$的系数都是有理数

    (2) 对任意的素数对$\ell, \ell'$,当$v\in \Sigma_K - S\bigcup S_{\ell}\bigcup S_{\ell'}$时有$P_{v, \rho_{\ell}} = P_{v, \rho_{\ell'}}$.
\end{cdef}

半单的表示可以被特征多项式决定,具体来说:

\begin{clem}
    令$k$是一个特征$0$的域,$A$是$k$代数,$M_1, M_2$是$A$的两个$k$-有限维半单模. 如果$\mathrm{Tr}\circ \rho_1 = \mathrm{Tr}\circ \rho_2$,则$M_1, M_2$同构.
\end{clem}

\begin{proof}
    只需要证明,如果$\{M_i\}$是有限多个互不同构的$k$-有限维不可约模,则$\mathrm{Tr}\ M_i: A\to k$线性无关. 令$N_i\subset A$是$M_i$的零化子,则$N_i$是双边理想,且作为左理想都是极大的. 由于$M_i$互不同构,$N_i$互不相同. 那么对每个$i$,存在$f_i\in A$使得$f_i\equiv 1\pmod{N_i}, f_i\equiv 0\pmod{N_j}, j\neq i$. 此时$\mathrm{Tr}\ M_i(f_i) = \dim_k M_i, \mathrm{Tr}\ M_j(f_i) = 0, j\neq i$.
\end{proof}

\begin{cthm}
    令$\rho$是有理$\ell$进表示,$\ell'$是一个素数(不一定和$\ell$不同). 那么在同构意义下至多有一个半单$\ell'$进表示$\rho'$,使得$\rho$和$\rho'$相容.
\end{cthm}

\begin{proof}
    令$\rho'_1, \rho'_2$是两个半单有理$\ell'$进表示,且都和$\rho$相容. 先证明$\mathrm{Tr}(\rho'_1(g)) = \mathrm{Tr}(\rho'_2(g))$对所有$g\in G$都成立. 记$J = \ker(\rho'_1)\bigcap \ker(\rho'_2)$,$H = G/J$,$M$是$J$对应的扩张. 那么$\mathrm{Gal}(M/K) = H$且$M/K$在有点多个素点以外非分歧. 由\Chebotarev 密度定理,Frobenius元素在$H$中稠密;而$\mathrm{Tr}\circ \rho'_1, \mathrm{Tr}\circ \rho'_1$可以定义为$H$上的连续函数,且在Frobenius元素上都相等,因此在整个$H$上相等,也就在整个$G$上相等. 在引理中取$k = \mathbb{Q}_{\ell'}, A = k[H]$,就得到$\rho'_1, \rho'_2$同构.
\end{proof}

\paragraph{例子:单位根}
假设$\ell\neq \mathrm{char}(K)$. 此时$x^{l^m}-1$是可分多项式,因此$K_s$中的$\ell^m$次单位根的乘法群$\mu_m$同构于$\mathbb{Z}/\ell^m \mathbb{Z}$. $G$作用在$\mu_m$上,因此作用在$\mathbb{Z}_{\ell} \cong T_{\ell}(\mu) = \invlim\mu_m$. 那么有$\chi_{\ell}:G\to \mathbb{Z}_{\ell}^{\times}\subset \mathbb{Q}_{\ell}^{\times}$,即一个一维的$\ell$进表示. 而如果$K$是代数数域,$v\in \Sigma_K, v\nmid \ell$,则$\chi_{\ell}$在$v$处非分歧,而且$F_{v, \chi_{\ell}} = Nv$. 因此$\chi_{\ell}$是一个整的$\ell$进表示. 当$\ell$取遍所有素数时,$\{\chi_{\ell}\}$形成了一族严格相容的有理$\ell$进表示,定义中的$S$可以取为空集.

\section{定义在线性代数群中的表示}

\begin{cdef}
    令$H$是一个$k$上的线性代数群,$k[H]$是$H$的坐标环. 若$f\in k[H]$满足对任意的交换$k$代数$k'$以及$x,y\in H(k')$,都有$f(xy)=f(yx)$,则称$f$是中心函数. 若$x\in H(k')$,且对任意的中心函数$f\in k[H]$都有$f(x)\in k$,则称$x$的共轭类是有理的.
\end{cdef}

\begin{cdef}
    令$H$是一个$\mathbb{Q}$上线性代数群. $K$的一个在$H$中取值的$\ell$进表示是指一个连续的$\rho: \mathrm{Gal}(\overline{K}/K)\to H(\mathbb{Q}_{\ell})$.
\end{cdef}

显然地定义非分歧、Frobenius元素.

\begin{cdef}
    称一个在$H$中取值的$\ell$进表示$\rho$为有理的,当且仅当存在有限子集$S\subset \Sigma_K$使得当$v\in \Sigma_K - S$时,$\rho$在$v$处非分歧,而且$F_{v, \rho}$的共轭类是有理的. ($k=\mathbb{Q}, k'=\mathbb{Q}_{\ell}$)
\end{cdef}

显然地定义相容性和严格相容性.

\begin{crem}
    如果$H$是交换的,则$F_{v,\rho}$的共轭类是有理的当且仅当$F_{v,\rho} \in H(\mathbb{Q})$.
\end{crem}

\begin{crem}
    如果$H = GL_n$,则由Chevalley关于不变多项式的定理,$H$上的中心函数形成的子代数就是$k[t_0,\ldots,t_{n-1}, (\det)^{-1}]$,其中$t_0=\det,t_1,\ldots,t_{n-2},t_{n-1}=\mathrm{tr}$是特征多项式的系数. 因此前一节定义的$\ell$进表示的有理性和本节定义的是一致的.
\end{crem}


\subsection{在\texorpdfstring{$S_{\mathfrak{m}}$}{Sm}中取值的表示}

记$\varepsilon : \mathbb{A}_K^{\times} \to I_{\mathfrak{m}}\to S_{\mathfrak{m}}(\mathbb{Q})$,$\pi: T \to T_{\mathfrak{m}}\to S_{\mathfrak{m}}$为定义$S_{\mathfrak{m}}$的映射. 对$\pi$取$\mathbb{Q}_{\ell}$点,得到$\pi_{\ell} : T(\mathbb{Q}_{\ell}) \to S_{\mathfrak{m}}(\mathbb{Q}_{\ell})$. 但是$T(\mathbb{Q}_{\ell}) = (K\otimes \mathbb{Q}_l)^{\times} = \prod_{v\mid l} K_v^{\times}$,由此得到映射$\alpha_{\ell} : \mathbb{A}_K^{\times} \xrightarrow{proj} T(\mathbb{Q}_{\ell}) \xrightarrow{\pi_{\ell}} S_{\mathfrak{m}}(\mathbb{Q}_{\ell})$.

\begin{clem}
    以下图表交换,其中的映射或者是上面定义的,或者是自然嵌入和投射
    \begin{figure}[H]
        \centering
        \begin{tikzcd}
            \mathbb{A}_K^{\times} \arrow[ddr, bend right] \arrow[drr, bend left]& & & \\
             &K^{\times} \arrow[ul] \arrow[r]\arrow[d] &K^{\times}/E_{\mathfrak{m}} \arrow[r]\arrow[d] &I_{\mathfrak{m}} \arrow[d]\\
             &T(\mathbb{Q}_{\ell}) \arrow[r] &T_{\mathfrak{m}}(\mathbb{Q}_{\ell}) \arrow[r] &S_{\mathfrak{m}}(\mathbb{Q}_{\ell})
        \end{tikzcd}
    \end{figure}
\end{clem}

因此,$\varepsilon_{\ell}(a) = \varepsilon(a) \alpha_{\ell}(a^{-1})$定义了$C_K \to S_{\mathfrak{m}}(\mathbb{Q}_{\ell})$的映射. 但是$S_{\mathfrak{m}}(\mathbb{Q}_{\ell})$的拓扑是完全不连通的,因此$C_K$的连通分支$D_K$的像为$1$. 那么,由类域论,$\varepsilon_{\ell}$定义了$G^{\mathrm{ab}}\to S_{\mathfrak{m}}(\mathbb{Q}_{\ell})$的映射.

记$F_v = \varepsilon(f_v) \in S_{\mathfrak{m}}(\mathbb{Q})$,其中$f_v$是任何一个在$v$处取素元,在其它地方取$1$的idèle.

\begin{cthm}
    (1) $\varepsilon_{\ell}$是一个取值在$S_{\mathfrak{m}}$中的有理$\ell$进表示

    (2) $\varepsilon_{\ell}$在$v\in \Sigma_K - \mathrm{supp}(\mathfrak{m})\bigcup S_{\ell}$处非分歧,且$F_{v, \varepsilon_{\ell}} = F_v \in S_{\mathfrak{m}}(\mathbb{Q})$.

    (3) $\{\varepsilon_{\ell}\}$形成了一族取值在$S_{\mathfrak{m}}$中的严格相容的$\ell$进表示.
\end{cthm}

\begin{proof}
    如果$v\in \Sigma_K - \mathrm{supp}(\mathfrak{m}), a\in U_v$,则$\epsilon(a) = 1$. 如果进一步地,$v\nmid \ell$,则$\alpha_{\ell}(a) = 1$,因此$\varepsilon_{\ell}$在$v$处非分歧. 同时,$\varepsilon_{\ell}(f_v)=\varepsilon(f_v) = F_v$,即$v$处的Frobenius元素是$F_v$.
\end{proof}

\begin{cthm}
    $\mathrm{Im}(\varepsilon_{\ell})$在$\basechange{S_{\mathfrak{m}}}{\mathbb{Q}_{\ell}}$中Zariski稠密.
\end{cthm}

\begin{proof}
    $\varepsilon$在$U_{\ell, \mathfrak{m}} = \prod_{v\mid \ell} U_{v, \mathfrak{m}}$上平凡,故$\epsilon_{\ell}(U_{\ell, \mathfrak{m}}) = \pi_{\ell}(U_{\ell, \mathfrak{m}})\subset T_{\mathfrak{m}}(\mathbb{Q}_{\ell}) \subset S_{\mathfrak{m}}(\mathbb{Q}_{\ell})$. 因此,$\mathrm{Im}(\varepsilon_{\ell})$是$S_{\mathfrak{m}}(\mathbb{Q}_{\ell})$的($\ell$进拓扑下的)开子群. 另一方面,由素理想在$C_{\mathfrak{m}}$中分布的经典结果,$f_v\in I_{\mathfrak{m}}$在$C_{\mathfrak{m}}$中的像取遍了整个$C_{\mathfrak{m}}$.
    因此$\mathrm{Im}(\varepsilon_{\ell})$在$S_{\mathfrak{m}}(\mathbb{Q}_{\ell})$的Zariski拓扑下稠密.
\end{proof}

\begin{ccor}
    $\{F_v\}$在$S_{\mathfrak{m}}$中Zariski稠密. \label{reps::sm::frob_dense}
\end{ccor}

\begin{proof}
    令$X$表示所有$F_v$的集合. 令$\ell$是素数. 令$\overline{X}, \overline{X}_{\ell}$分别为$X$在$S_{\mathfrak{m}}$(Zariski拓扑),$S_{\mathfrak{m}}(\mathbb{Q}_{\ell})$($\ell$进拓扑)中的闭包. 那么$\overline{X}_{\ell}\subset \overline{X}(\mathbb{Q}_{\ell})$. 但是\Chebotarev 密度定理说明,$X$在$\mathrm{Im}(\varepsilon_{\ell})$中($\ell$进)稠密,即$\overline{X}_{\ell} = \mathrm{Im}(\varepsilon_{\ell})$. $\mathrm{Im}(\varepsilon_{\ell})$在$\basechange{S_{\mathfrak{m}}}{\mathbb{Q}_{\ell}}$(Zariski)稠密,因此$\overline{X}(\mathbb{Q}_{\ell}) = S_{\mathfrak{m}}(\mathbb{Q}_{\ell})$.
\end{proof}

\subsection{通过\texorpdfstring{$S_{\mathfrak{m}}$}{Sm}定义的表示}

设$V_{\ell}$是$\mathbb{Q}_{\ell}$上的有限维线性空间,$\varphi: \basechange{S_{\mathfrak{m}}}{\mathbb{Q}_{\ell}} \to GL(V_{\ell})$是$\basechange{S_{\mathfrak{m}}}{\mathbb{Q}_{\ell}}$的表示. 令$\varphi_{\ell}$为映射$G^{\mathrm{ab}}\to S_{\mathfrak{m}}(\mathbb{Q}_{\ell})\xrightarrow{\varphi} GL(V_{\ell})$.

\begin{cthm}
    (1) $\varphi_{\ell}$是半单表示

    (2) 令$v\in \Sigma_K - \mathrm{supp}(\mathfrak{m})\bigcup S_{\ell}$,则$\varphi_{\ell}$在$v$处非分歧,且$F_{v, \varphi_{\ell}} = \varphi(F_v)$

    (3) $\varphi_{\ell}$是有理表示当且仅当$\varphi$可以在$\mathbb{Q}$上定义 \label{single_ell}
\end{cthm}

\begin{proof}
    $S_{\mathfrak{m}}$是乘法型群,任意表示都可以在有限扩张后对角化,由此得到(1). 由$\epsilon_{\ell}$的非分歧性和Frobenius元素的指定,(2)是显然的.
    为了不打断行文顺序,(3)的证明留到本小节最后给出.
\end{proof}

(3)告诉我们,考察有理表示时,只需要从一个$\mathbb{Q}$上的表示$S_{\mathfrak{m}}\to GL(V)$出发.

设$V$是$\mathbb{Q}$上的有限维空间,$\varphi: S_{\mathfrak{m}}\to GL(V)$是$S_{\mathfrak{m}}$的表示. 令$V_{\ell} = V\otimes \mathbb{Q}_{\ell}$,那么有$\varphi_{\ell}: \basechange{S_{\mathfrak{m}}}{\mathbb{Q}_{\ell}}\to GL(V_{\ell})$. 按照上面的讨论,这就是在说有表示$\varphi_{\ell}: G^{\mathrm{ab}}\to GL(V_{\ell})$. 称如此定义的$\{\varphi_{\ell}\}$为通过$S_{\mathfrak{m}}$定义的.

\begin{cthm}
    (i) $\{\varphi_{\ell}\}$形成了一族严格相容的有理$\ell$进表示,例外集不大于$\mathrm{supp}(\mathfrak{m})$

    (ii) 当$v\in \Sigma_K - \mathrm{supp}(\mathfrak{m})\bigcup S_{\ell}$时,$F_{v, \varphi_{\ell}} = \varphi(F_v)$

    (iii) 存在无穷多个素数$\ell$使得$\varphi_{\ell}$在$\mathbb{Q}_{\ell}$上可以对角化
    \label{reps::finite_diagonal}
\end{cthm}

\begin{proof}
    (1)和(2)由定理\ref{single_ell}对单个$\ell$的计算容易得到. 由命题\ref{reps::split_finite},存在$K$的有限扩张$E$使得$\varphi$在$E$上对角化. 如果$\ell$在$E$上完全分裂,则有嵌入$E\to \mathbb{Q}_{\ell}$,此时$\varphi_{\ell}$在$\mathbb{Q}_{\ell}$上可以对角化. 由\Chebotarev 密度定理,存在无穷多个这样的$\ell$.
\end{proof}

\begin{ccor}
    如果$\{\rho_{\ell}\}$是一族通过$S_{\mathfrak{m}}$定义的$\ell$进表示,则存在无穷很多个素数$\ell$使得$\rho_{\ell}$是一维表示的直和;特别地,$\rho_{\ell}$在$\mathbb{Q}_{\ell}$上可约.\label{sm_to_reducile}
\end{ccor}

现在来证明定理\ref{single_ell}的(3).

\begin{cprop}
    令$\mathbb{Q}\subset k\subset k_1$. 对于$\varphi\in \mathsf{Rep}_{k_1}(\basechange{S_{\mathfrak{m}}}{k_1})$,以下性质等价:

    (i) $\varphi$可以定义在$k$上

    (ii) 对任意的$v\not\in \mathrm{supp}(\mathfrak{m})$,$\varphi(F_v)$的特征多项式的系数都落在$k$中

    (iii) 存在密度为$1$的子集$\Sigma \subset \Sigma_K$使得当$v\in \Sigma$时,$\mathrm{Tr}(\varphi(F_v))\in k$.
\end{cprop}

\begin{proof}
    (i)$\Rightarrow$(ii)$\Rightarrow$(iii)是显然的.
    
    现在证明(iii)$\Rightarrow$(i). 记$\Lambda = \mathscr{O}(\basechange{S_{\mathfrak{m}}}{k})$,
    $\theta_{\varphi}$为$\varphi$的特征.
    令$\{\ell_{\alpha}\}$为$k_1$作为$k$向量空间的一组基,$\ell_{0} = 1$.
    将$\theta_{\varphi}$写成$\sum_{\alpha} \lambda_{\alpha} \ell_{\alpha}$的形式,
    其中$\lambda_{\alpha}\in \Lambda$.
    那么当$h\in S_{\mathfrak{m}}(k_1)$时,
    $\mathrm{Tr}(\varphi(h)) = \theta_{\varphi}(h) = \sum_{\alpha} \lambda_{\alpha}(h)\ell_{\alpha}$.
    令$h=F_v$,由$\mathrm{Tr}(\varphi(F_v))\in k$可知当$\alpha\neq 0$时$\lambda_{\alpha}F_v = 0$.
    但是由推论\ref{reps::sm::frob_dense},$F_v,v\in \Sigma$在$\mathbb{S}_{\mathfrak{m}}$中
    Zariski稠密. 从而当$\alpha\neq 0$时,$\lambda_{\alpha} = 0$.
\end{proof}


\subsection{局部代数表示}

这一小节给出一个$\ell$进表示可以从$S_{\mathfrak{m}}$定义的充分条件.

先看局部域的情况.
令$L$是$\mathbb{Q}_p$的有限扩张,$T = \mathrm{Res}_{L/\mathbb{Q}_p}(\mathbb{G}_m / L)$.
假设$V$是有限维$\mathbb{Q}_p$向量空间,$\rho$是一个Abel的$p$进表示$\rho: \mathrm{Gal}(\overline{L}/L)^{ab} \to \mathrm{Aut}(V)$.

\begin{cdef}
    称$\rho$是局部代数表示是在说,存在代数群的态射$r:T\to \mathrm{GL}(V)$使得当$x\in L^{\times}$足够接近于$1$时,$\rho\circ \iota(x) = r(x^{-1})$,其中$\iota: L^{\times}\to \mathrm{Gal}(\overline{L}/L) $是局部Artin映射.(这里局部Artin映射约定为$\pi$映射到Frobenius元素的逆)
\end{cdef}

一维的局部代数表示就是代数Hecke特征的$\ell$进类比. 由于$T$是环面,而$\rho$“差不多”是一个$T$的代数表示,可以将$\rho$对角化为一系列一维表示来处理.

\begin{cprop}
    如果$\rho: \mathrm{Gal}(\overline{L}/L)^{ab} \to \mathrm{Aut}(V)$是局部代数的表示,则$\rho$限制在惯性子群上的表示是半单的.
\end{cprop}

\begin{proof}
    由定义,存在$U_L$的开子群$U'$和代数群的态射$r: T\to \mathrm{GL}(V)$使得当$x\in U'$时$\rho\circ \iota(x) = r(x^{-1})$. 如果$W$是$V$的$\rho\circ\iota(U_L)$-不变的子空间,那么$W$也是$\rho\circ\iota(U')=r(U')$-不变的. 又因为$U'$在$T$中Zariski-稠密,$W$也是$r(T)$不变的. 但是$T$是环面群,所有$T$的表示都是半单的. 因此$W$有补空间,即存在投影映射$\pi: V\to W$使得$\pi$和$r(T)$的作用交换. 此时$\pi$也和$\rho\circ\iota(U')$的作用交换. 再令$\pi' = \frac{1}{[U_L:U']} \sum_{s\in U_L/U'} \rho\circ \iota(s) \pi \rho\circ \iota(s^{-1})$,则$\pi'$是和$\rho\circ\iota(U_L)$的作用交换的投影映射. 从而$W$在$V$中有$\rho\circ\iota(U_L)$-不变的补空间.
\end{proof}

固定一个$\rho$,取一个足够大的有限扩张$E/\mathbb{Q}_p$使得(i)$\rho$在$U_L$上的限制在$E$上可以对角化;(ii)$E$包含了$L$的正规闭包. 令$\Gamma_{L}$为所有域嵌入$L\to E$形成的群. 记$\chi_i: U_L\to E^{\times}, i=1,2,\ldots,\mathrm{dim}\ \rho$满足$\rho \cong_{E} \mathrm{diag}(\chi_1,\ldots, \chi_{\mathrm{dim}\ \rho})$.

\begin{cprop}
    $\rho$是局部代数表示当且仅当存在整数$n_{\sigma}(i)$使得对所有的$i$以及足够接近于$1$的$u$都有
    \begin{equation}
        \chi_i(u) = \prod_{\sigma\in \Gamma_K} \sigma(u)^{-n_{\sigma}(i)} \label{eqns::temp::1}
    \end{equation}
\end{cprop}

\begin{proof}
    如果$\rho$是局部代数的,则每个$\chi_i$都是局部代数的,这就等价于\refeq{eqns::temp::1}.

    反之,\refeq{eqns::temp::1}定义了$T$的$E$上的态射$r$,且满足和$\rho$的相容性关系. 只要证明$r$可以定义在$\mathbb{Q}_p$上. 但是在某个小邻域$U'$上,$r$的迹就是$\rho$的迹,从而是$\mathbb{Q}_p$的元素;同时,$U'$在$T$中Zariski-稠密.
\end{proof}

现在回到整体的情况. 仍然令$K$为代数数域. 给定$\ell$进表示$\rho: \mathrm{Gal}(\overline{K}/K)^{ab}\to \mathrm{Aut}(V_{\ell})$. 令$v$是$K$的有限素点,且$p_v = \ell$. 选定一个$\overline{K}$的素点$w\mid v$,则有
\begin{equation}
    \mathrm{Gal}(\overline{K_v}/K_v) \xrightarrow{D_{w}} \mathrm{Gal}(\overline{K}/K) \to \mathrm{Gal}(\overline{K}/K)^{ab}
\end{equation}
由于最终的像是交换群,映射与$w$的选取无关,并可以通过$\mathrm{Gal}(\overline{K_v}/K_v)^{ab}$分解. 因此有表示
\begin{equation}
    \rho_v: \mathrm{Gal}(\overline{K_v}/K_v)^{ab} \to \mathrm{Gal}(\overline{K}/K)^{ab} \xrightarrow{\rho} \mathrm{Aut}(V_{\ell})
\end{equation}

\begin{cdef}
    称$\rho$是局部代数表示是指对所有满足$p_v=\ell$的素点$v$,表示$\rho_v$是局部代数的.
\end{cdef}

局部代数表示也可以用整体的方式来刻画. 记$\iota_{\ell}: K_{\ell}^{\times}=(K\otimes \mathbb{Q}_{\ell})^{\times} \to \mathbb{A}_{K}^{\times} \to \mathrm{Gal}(\overline{K}/K)^{ab}$为显然的单射与整体Artin映射的复合.

\begin{cprop}
    $\rho$是局部代数表示当且仅当存在态射$f: T/\mathbb{Q}_{\ell} \to \mathrm{GL}(V_{\ell})$使得对所有足够接近$1$的$x\in K_{\ell}^{\times}$都有$\rho\circ \iota_{\ell}(x) = f(x^{-1})$.
\end{cprop}

\begin{cdef}
    令$\mathfrak{m}$是$K$的整理想. 称$\rho$可以模$\mathfrak{m}$定义,或者说$\mathfrak{m}$是$\rho$的定义理想,是在说

    (i) 当$p_v\neq \ell$时,$\rho\circ \iota$在$U_{v, \mathfrak{m}}$上平凡;

    (ii) 当$x\in \prod_{v\mid \ell} U_{v, \mathfrak{m}}$时,$\rho\circ\iota_{\ell} (x) = f(x^{-1})$.
\end{cdef}

\begin{prop}
    每个局部代数表示$\rho$可以模某个$\mathfrak{m}$定义.
\end{prop}

要证明这个命题,只需要利用一下拓扑性质:

\begin{clem}
    令$H$是$\ell$进李群,设有连续映射$\alpha: \mathbb{A}_{K}^{\times}\to H$. 那么:

    (i) 如果$p_v\neq \ell$,则$\alpha$限制在$K_v^{\times}$上之后,在某个开子群上平凡;

    (ii) 对除了有限多个以外的素点$v$,$\alpha$限制在$U_v$上平凡.

    特别地,所有Abel的$\ell$进表示都在一个有限的素点集合以外非分歧.
\end{clem}

\begin{proof}
    (i) 注意到$K_v^{\times}$是$p_v$进李群,而$p_v\neq \ell$,此时任何连续映射$K_v^{\times}\to H$都通过有限群分解.

    (ii) 令$N$是$H$中$1$的邻域,且$N$中不包含非平凡的有限子群. 对几乎所有的$v$,$\alpha(U_v)\in N$,而且当$p_v\neq \ell$时像是有限子群. 此时$\alpha(U_v)=1$.
\end{proof}

满足$\rho$可以模$\mathfrak{m}$定义的最小理想$\mathfrak{m}$称为$\rho$的导子.

现在可以证明这一小节的主定理:

\begin{cthm}
    设$\rho: \mathrm{Gal}(\overline{K}/K)^{ab} \to \mathrm{Aut}(V_{\ell})$是有理$\ell$进表示,且$\rho$是局部代数的,$\mathfrak{m}$是$\rho$的定义\modulus . 那么存在$V_{\ell}$的$\mathbb{Q}$-向量子空间$V$使得$V_{\ell} = V\otimes \mathbb{Q}_{\ell}$,以及代数群的态射$\phi: S_{\mathfrak{m}}\to \mathrm{GL}(V)$使得$\rho=\phi_{\ell}$.
\end{cthm}



\section{交换表示的局部代数性}

Serre在{\parencite{serre1997abelian}}中为了证明$\ell$进表示的局部代数性质给出了三种方法,其中两种利用Hodge-Tate分解和对局部域上相应表示的分析. 但事实上这些表示是局部代数的事实与它们来自椭圆曲线并没有多大关系. Serre的第三种方法是用Lang关于超越数的结果证明以下简洁的结果,但由于Lang的结果所限只能局限在$K$是二次域的复合的情形. 现在,有了更强大的Waldschmidt关于超越数的结果,我们可以证明

\begin{cthm}[{\parencite[][p. 117]{henniart}}]
    如果$\rho$是有理、半单、交换的$\ell$进表示,那么$\rho$是局部代数的. \label{reps::abelian_loc_alg}
\end{cthm}

证明定理\ref{reps::abelian_loc_alg}前还需要一些准备工作.

\begin{cprop}
    如果$\rho$是有理、半单、交换的$\ell$进表示,且存在正整数$N$使得$\rho^N$是局部代数的,那么$\rho$是局部代数的. \label{reps::power_N}
\end{cprop}

\begin{proof}

    由半单性,$\rho$可以在$\mathbb{Q}_{\ell}$的一个有限扩张上对角化. 将$\rho$写成$\mathrm{diag}(\psi_1,\ldots,\psi_n)$,其中$n=\mathrm{dim}\ \rho$,$\psi_i: \mathbb{A}_{K}^{\times}/K^{\times}\to \overline{\mathbb{Q}_{\ell}}^{\times}$是连续特征标.
    令$\chi_i = \psi_i^N$是$\rho^N$的特征标. 由于$\rho^N$是局部代数的,对每个$\chi_i$都存在一个$\chi_i^{alg}\in X^{*}(T)$使得当$x$充分靠近$1$时$\chi_i(x) = \chi_i^{alg}(x^{-1})$. 而$X^{*}(T)$可以看作是$\mathrm{Hom}(\basechange{T}{\overline{\mathrm{Q}_{\ell}}}, \basechange{\mathbb{G}_m}{\overline{\mathbb{Q}_{\ell}}})$,即$\chi_i^{alg}$可以写成$\prod_{\sigma\in \Gamma} \sigma^{n_{\sigma}(i)}$,其中$\Gamma$是$K$到$\overline{Q_{\ell}}$的域嵌入的群.

    \vskip0.3cm

    先证明,每个$n_{\sigma}(i)$都被$N$整除. 令$U$是$\overline{\mathbb{Q}_{\ell}}^{\times}$的开子群,且不包含非平凡的$N$次单位根. 取$\mathfrak{m}$为一个充分大的理想,使得对

    (i) $\psi_i(U_{\mathfrak{m}})\subset U$;

    (ii) $\rho^N$可以模$\mathfrak{m}$定义;

    (iii) $\rho$在$v\not\in \mathrm{supp}(\mathfrak{m})$处非分歧,且$F_{v, \rho}$的特征多项式是有理系数的.

    令$K_{\mathfrak{m}}$是$K^{\times} U_{\mathfrak{m}}$对应的Abel扩张. 取一个包含$K_{\mathfrak{m}}$的Galois$L/\mathbb{Q}$. 再取一个充分大的素数$p$,使得$p>\ell$,$p>p_v,\forall v\in \mathrm{supp}(\mathfrak{m})$,而且$p$在$L$中完全分裂. 令$v$是$K$的素点,且$v\mid p$. 记$f_v$为在$v$处取素元,其余处取$1$的idèle.

    $v$在$K_{\mathfrak{m}}$中完全分解,因此$f_v \in N_{K_{\mathfrak{m}}/K}$,即$f_v\in K^{\times} U_{\mathfrak{m}}$. 因此素理想$p_v$是主理想$(\alpha)$,其中$\alpha\equiv 1\pmod{\mathfrak{m}\prod_{k_w\cong \mathbb{R}} w}$.

    记$x = \phi_i(f_v)$,$y = \chi_i(f_v)$,由定义有
    \begin{equation}
        y = \chi_i(f_v) = \chi_i(\alpha^{-1}_{\ell}) = \chi_i^{alg}(\alpha_{\ell}) = \prod_{\sigma\in \Gamma} \sigma(\alpha)^{n_{\sigma}(i)}
    \end{equation}
    其中$\sigma\in \Gamma$看成是$K\to L$的嵌入. $x,y$都看成是$L$的元素.

    此时$y\in \tilde{L}\subset \mathbb{Q}_p$,$\tilde{L}$是$\mathbb{Q}_p$中(唯一的)与$L$同构的子域. 如果$w_{\sigma}$是$L$的素点,使得$w_{\sigma}\circ \sigma$限制在$K$上是$v$,那么$w_{\sigma}(y) = n_{\sigma}(i)$.

    如果$N$不整除$n_{\sigma}(i)$,那么$x\not\in \overline{L}$. 那么存在一个非平凡的$N$次单位根$z$使得$x, zx$在$\overline{L}$上共轭,从而在$\mathbb{Q}$上共轭. 但是$F_{v,\rho}$的特征多项式是有理系数的,$F_{v,\rho}$的特征值的共轭也是特征值. 那么存在一个$j$使得$\psi_j(f_v) = zx = z\psi_i(f_v)$. 但是$\psi_j(f_v)\in U$,$\psi_i(f_v)\in U$,与$U$中没有非平凡的$N$次单位根矛盾.

    \vskip0.3cm

    回到命题本身的证明. 由于$n_{\sigma}(i)$都被$N$整除,存在$\phi_i\in X^{*}(T)$使得$\phi_i^N = \chi_i^{alg}$. 当$x\in K_{\ell}^{\times}$充分靠近$1$时,
    \begin{equation}
        \phi_i(x^{-1})^N = \chi_i^{alg}(x^{-1}) = \chi_i(x) = \psi_i(x)^N
    \end{equation}
    此时$\phi_i(x)\psi_i(x)$是$N$次单位根. 但是$N$次单位根的群是离散的,从而存在一个$1$的小邻域使得$\phi_i\psi_i$在这个邻域上是$1$. 令$\phi=\mathrm{diag}(\phi_1,\ldots,\phi_n)$,则在$1$的小邻域上$\rho = \psi^{-1}$. 由于$\rho$能定义在$\mathbb{Q}_{\ell}$上,$\phi$也能定义在$\mathbb{Q}_{\ell}$上.

    那么$\rho$是局部代数的.
\end{proof}

接下来需要一个环面群的特征标的结论.
令$C$为$\overline{\mathbb{Q}_{\ell}}$的完备化. 仍然令$T=\mathrm{Res}_{K/\mathbb{Q}}(\basechange{\mathbb{G}_m}{K})$为$\mathbb{Q}$上的$n$维环面子群.
设$f: T(\mathbb{Q}_{\ell})\to C^{\times}$是连续映射. 若存在$1$在$T(\mathbb{Q}_{\ell})$的($\ell$进拓扑下)开邻域$U$和$\phi\in X^{*}(T)$使得当$x\in U$时$f(x)=\phi(x)$,则称$f$是局部代数的. 如果存在正整数$N$使得$f^N$是局部代数的,则称$f$是几乎局部代数的.

需要的关于超越数的结论是
\begin{cprop}[{\parencite[][p. 117]{henniart}}; {\parencite[][p. 124]{waldschmidt1981transcendance}}]
    令$V$是$C$上的有限维线性空间,$\Gamma$是$V$的有限秩子群. 记$\mu_{V}(\Gamma) = \min_W \left\{ \frac{\mathrm{rank}(\Gamma) - \mathrm{rank}(\Gamma \bigcap W)}{\dim V-\dim W} \right\}$,其中$W$取遍$V$的真子空间. 如果$\Lambda, \Gamma$是$V, V^{*}$的有限秩子群,$\mathrm{rank}(\Gamma)>\dim V$,且$\exp\langle \lambda, \gamma\rangle, \lambda \in \Lambda, \gamma\in \Gamma$都收敛并取代数数值. 那么
    \begin{equation}
        \mu_{V^{*}}(\Gamma)\leq \frac{\mathrm{rank}(\Lambda)}{\mathrm{rank}(\Lambda) - \dim V}
    \end{equation}\label{temp::theo8}
\end{cprop}

设$S$是素数的一个有限集合. 对每个$p\in S$,令$W_p$是$T(\mathbb{Q}_p)$的开子群. 记$T(\mathbb{Q})_{W}$为所有的$x\in T(\mathbb{Q})$使得$x\in W_p\subset T(\mathbb{Q}_p)$. 那么$T(\mathbb{Q})_{W}$是$T(\mathbb{Q})$的子群.

\begin{cprop}
    如果存在一族$W_p, p\in S$使得$f(T(\mathbb{Q})_W)$都是$\mathbb{Q}$-代数数,则$f$是几乎局部代数的. \label{temp::reps::loc_alg}
\end{cprop}

\begin{clem}
    如果$\chi\in X^{*}(T)$不是$0$,则$\chi(T(\mathbb{Q}))$是无限秩的.
\end{clem}

\begin{proof}
    也就是要证明
    \begin{equation}
        K^{\times}\to L^{\times}, x\mapsto \prod_{\sigma} \sigma(x)^{n_{\sigma}}
    \end{equation}
    的像是无限秩的,其中$L$是$K$的正规闭包. 设$p$是在$L$中完全分解的素数,$v$是$K$的素点且$v\mid p$.
    如果$w_{\sigma}$是$L$的满足$w_{\sigma}\circ\sigma=v$的素点,则$w_{\sigma}(\chi(x)) = n_{\sigma}v(x)$.
    特别地,如果$x\in K^{\times}$使得$v(x) > 0$,且$v'\neq v$时$v'(x) = 0$,则$w_{\sigma}(\chi(x)) = n_{\sigma}v(x)$,而当$w'\nmid p$时$w'(\chi(x))=0$.

    对不同的$p$构造的$x$的像是线性无关的. 由于在$L$中完全分解的素数有无穷多个,$\chi$的像的秩是无穷的.
\end{proof}

\begin{proof}[命题\ref{temp::reps::loc_alg}的证明]

    令$\chi_1,\ldots,\chi_n$是$X^{*}(T)$的一组基.
    $T(\mathbb{Q})$是无限秩的$\mathbb{Z}$模.
    而$T(\mathbb{Q}_p)/W_p$都是秩不大于$n$的有限生成$\mathbb{Z}$模,于是$T(\mathbb{Q})/T(\mathbb{Q})_W$是有限生成的. 从而$T(\mathbb{Q})_W$是无限秩的$\mathbb{Z}$模.

    $T(\mathbb{Q}_{\ell})$是一个$\ell$进李群. 令$\mathfrak{t}$为其李代数. 则存在$\mathfrak{t}$的一个紧开子群$\mathfrak{t}_0$使得$\exp: \mathfrak{t}_0\to T(\mathbb{Q}_{\ell})$是到一个紧开子群的同胚,而且群结构可以有Campbell-Hausdorff公式计算. 由于$T(\mathbb{Q}_{\ell})$是交换的,$\mathfrak{t}$也是交换的. $\exp$是$(\mathbb{Z}_{\ell})^n\cong \mathfrak{t}_0$到$T(\mathbb{Q}_{\ell})$的一个紧开子群的解析同构.

    通过与$\exp$复合,可以得到$n+1$个连续映射
    \begin{equation}
        f\circ\exp, \chi_i\circ\exp, i=1,2,\ldots,n
    \end{equation}
    由于$\mathbb{Z}_{\ell}\to C^{\times}$的连续映射局部上都是指数映射,存在$b_0, b_1,\ldots,b_n \in C^n$使得在充分小的开子群上有
    \begin{equation}
        f\circ\exp(z) = \exp \langle b_0 z\rangle, \chi_i\circ\exp = \exp\langle b_i z\rangle, i=1,2,\ldots,n
    \end{equation}
    通过在$b_i$上乘上适当的$\ell^k$,不妨设所有的$\langle b_0, (\mathbb{Z}_{\ell})^n\rangle$都落在$\exp$的收敛区域内,且以上各式在$(\mathbb{Z}_{\ell})^n$上成立. 记$\Gamma$为$b_0, \ldots,b_n$生成的$\mathbb{Z}$模.

    令$\Lambda = \{z\in (\mathbb{Z}_{\ell})^{\ell}\mid \exp(z)\in T(\mathbb{Q})_{W}\}$. 由于$T(\mathbb{Q}_{\ell})/\exp(\mathbb{Z}_{\ell})$都是有限生成的,$\Lambda$是无限秩的$\mathbb{Z}$模. 而当$z\in \Lambda$时,$f\circ \exp(z)$和$\chi_i\circ\exp(z)$都是代数数.

    记$V\cong \mathbb{Q}_{\ell}^n$为$\Lambda$生成的空间,通过二次型$\sum_i x_iy_i$确定$V$和$V^{*}$的同构,并将$b_i$视为$V^{*}$的元素.
    
    注意到$\{\chi_i\}$是$X^{*}(T)$的一组基,$b_1,\ldots,b_n$ $\mathbb{Z}$线性无关. 因此只要证明$\mathrm{rank}(\Gamma)= n$,就能得到$b_0$是其它$b_i$的有理系数线性组合,于是$f$是$\chi_i$的有理系数线性组合. 这就证明了$f$是几乎局部代数的.

    假设$\mathrm{rank}(\Gamma) = n+1$. 对$\Gamma$和$\Lambda$的秩任意大的有限秩子群用命题\ref{temp::theo8},可知$\mu_{V^{*}}(\Gamma)\leq 1$,即存在真子空间$W\subset V^{*}$使得$\mathrm{rank}(\Gamma) - \mathrm{rank}(\Gamma\bigcap W) \leq \dim V - \dim W$,即$\mathrm{rank}(\Gamma\bigcap W)\geq \dim W + 1$. 取$W$为满足条件的空间中维数最小的. 注意到无论如何$\mathrm{rank}(\Gamma\bigcap W)\geq 2$,从而存在非零的$w\in \Gamma\bigcap W$是$b_1, b_2,\ldots, b_n$张成的($b_0$的系数为$0$).

    先验证$\Lambda$在$V/w^{\perp}$的投影还是无限秩的. 记$\pi_w: V\to V/w^{\perp}$设$w = \sum_i n_i b_i$. 令$\chi = \prod_i \chi_i^{n_i}$. 由假设,$\chi$是非平凡特征标.
    $T(\mathbb{Q})/\exp(\Lambda)$是有限生成的,而$\chi(T(\mathbb{Q}))$的秩无限. 而且当$\lambda\in \Lambda$时,
    \begin{equation}
        \exp \langle w, \pi_w(\lambda)\rangle = \exp \langle w, \lambda\rangle = \chi \circ \exp (\lambda)
    \end{equation}
    因此$\pi_w(\Lambda)$还是无限秩的.

    此时$\Lambda$在$V/W^{\perp}$上的投影$\pi(\Lambda)$还是无限秩的. 对$\Gamma\bigcap W$和$\pi(\Lambda)$再用一次命题\ref{temp::theo8},有$\mu_{W^{*}}(\Gamma\bigcap W)\leq 1$,即存在真子空间$W'$满足$\mathrm{rank}(\Gamma\bigcap W) - \mathrm{rank}(\Gamma\bigcap W') \leq \dim W - \dim W'$,即$\mathrm{\Gamma \bigcap W'}\geq \dim W' + 1$. 但这与$W$的极小性矛盾.

\end{proof}

现在开始定理\ref{reps::abelian_loc_alg}的证明.
\begin{proof}

    由半单性,$\rho$在$C$上对角化. 设$\rho \cong_{C} \mathrm{diag}(\psi_1,\ldots,\psi_n)$. 令$f_i$为$\psi_i$在$K_{\ell}^{\times}$上的限制.

    固定一个$i$.
    令$S$是有限的素数集合,使得$\ell\not\in S$,而且当$v\in \Sigma_K, p_v\neq \ell, p_v\not\in S$时,$\rho$在$v$处非分歧并且$F_{v,\rho}$的特征多项式是有理系数的.
    当$p\in S$时,取$W_p$为$K_{p}^{\times}$的开子群,使得$\phi_i(W_p)=1$.

    下面对以上$S$和$\{W_p\}$验证命题\ref{temp::reps::loc_alg}的条件. 设$x\in K^{\times}$. 将$x$作为idèle分解为各个分量之积
    \begin{equation}
        x = x_{\infty} x_{\ell} x_{S} x'
    \end{equation}
    由$\psi_i(x) = 1$和$\psi_i(x_{\ell}) = f_i(x)$有
    \begin{equation}
        f_i(x^{-1}) = \psi_i(x_{\infty})\psi_i(x_S)\psi_i(x')
    \end{equation}
    由构造,$\psi_i(x_S)=1$;由完全不连通性,$\phi_i(x_{\infty})=\pm 1$. 对于$v\not\in S, p_v\neq \ell$的情况,此时$\rho$在$v$处非分歧,$F_{v,\rho}$的特征多项式是有理系数多项式,因此$\psi_i(f_v)$作为一个特征值是代数数. 而$\psi_i(x')$是有限多个$\psi_i(f_v)$的乘积,因此也是代数数. 那么$f_i(x)$是代数数.

    由命题\ref{temp::reps::loc_alg},每个$f_i$都是几乎局部代数的. 那么存在一个$N$使得每个$f_i$都是局部代数的,此时$\rho^N$也是局部代数的. 再由命题\ref{reps::power_N},$\rho$是局部代数的.
\end{proof}



    \section{椭圆曲线的一些结论}

这一节汇总了一些要用到的椭圆曲线中的结论.

\section{Néron-Ogg-Shafarevich判别法}

当惯性子群的作用是平凡的时候,称一个Galois模为非分歧的.

\begin{cthm}[Néron-Ogg-Shafarevich]
    令$K$是局部域,$k$是$K$的剩余类域.  $E$是$K$上的椭圆曲线,则以下命题等价:

    (i) $E$有好约化;

    (ii) 对所有的$(m, \mathrm{char}\ k)=1$时,$E[m]$非分歧;

    (iii) 对所有的$\ell\neq \mathrm{char}\ k$时,$T_{\ell}$非分歧;

    (iv) 对无穷多个与$\mathrm{char}\ k$互素的$m$,$E[m]$非分歧.
\end{cthm}

\begin{proof}
    {\parencite[][p. 201]{silverman2009arithmetic}}
\end{proof}

\begin{ccor}
    如果$K$是代数数域,$E, E'$是$K$上同源的椭圆曲线,$v$是$K$的有限素点. 则$E$在$v$处有好约化当且仅当$E'$在$v$处有好约化. \label{galois::same_reduction}
\end{ccor}

\begin{proof}
    设$E\to E'$是次数为$n$的同源,则有$K_v$上相同次数的同源. 令$m$是与$\mathrm{char}\ k_v$,$\mathrm{deg}(E\to E')$都互素的正整数,那么Galois模$E_v[m]$和$E'_v[m]$同构. 由Néron-Ogg-Shafarevich判别法,$E, E'$在$v$处同时有好约化或者同时有坏约化.
\end{proof}


\section{Shafarevich的定理和一些推论}

令$K$是代数数域.

\begin{cthm}[Shafarevich]
    如果$S$是$K$的素点的一个有限集合,那么$K$上在$S$以外有好约化的椭圆曲线(在$K$-同构意义下)只有有限多条.
\end{cthm}

\begin{ccor}
    给定$K$上的椭圆曲线$E$,则和$E$在$K$上同源的椭圆曲线(在$K$-同构意义下)只有有限多条.\label{galois::isogeny_finite_curves}
\end{ccor}

\begin{proof}
    Shafarevich的定理和推论\ref{galois::same_reduction}.
\end{proof}


\section{Galois模的不可约性}

设$K$是代数数域,$E$是定义在$K$上的椭圆曲线. 本节要证明

\begin{cthm}
    如果$E$在$K$上没有复乘,则:

    (i) 对所有素数$\ell$,$V_{\ell}$不可约;

    (ii) 对除有限个以外的素数$\ell$,$E[\ell]$不可约.
    \label{galois::irreducible}
\end{cthm}

\begin{clem}
    如果$E$在$K$上没有复乘,$E'\to E$和$E''\to E$是两个定义在$K$上的同源,且核是不同构的循环群. 那么$E'$和$E''$在$K$上不同构.
\end{clem}

\begin{proof}
    假设$E'\to E$和$E''\to E$的核分别是阶为$n', n''$的循环群,$E'\to E''$是一个同构. 令$E\to E'$是$E'\to E$的转置,则它的核是$n'$阶循环群. 此时以上几个映射的复合$E\to E'\to E''\to E$的核是$n''$阶循环群关于$n'$阶循环群的群扩张. 但是$\mathrm{End}_K(E) \cong \mathrm{Z}$,于是这个映射只能是某个$a$倍映射,核只能是$(\mathbb{Z}/a\mathbb{Z})^2$的形式. 此时$n'\mid a, n''\mid a$且$n'n''=a^2$,那么只能是$a=n'=n''$,矛盾.
\end{proof}

\begin{proof}
    (定理\ref{galois::irreducible})
    (i) 假设$V_{\ell}$有一个一维的不变子空间$Y$,则$X=Y\bigcap T_{\ell}$是$T_{\ell}$的子模,且$X$和$T_{\ell}/X$都是秩为$1$的自由$\mathbb{Z}_{\ell}$-模. 令$n\geq 0$,记$X(n)$是$X$在$T_{\ell} / \ell^n T_{\ell}$中的像. $X(n)$是$E[\ell^n]$的子模,而且是阶为$\ell^n$的循环群. 那么$X(n)$是$E$的$K$-代数子群. 令$E(n) = E/X(n)$,那么$E\to E(n)$是$K$-同源,且核为$\ell^n$阶循环群. 上面的引理说明不同的$n$对应的$E(n)$互不同构,与Shafarevich的定理\ref{galois::isogeny_finite_curves}矛盾.
    % TODO: quotient of elliptic curves?

    (ii) 如果$E[\ell]$中有一维不变子空间$X_{\ell}$,则类似(i)中的情况,存在同源$E\to E/X_{\ell}$. 由引理,对于不同的$\ell$的取值,$E/X_{\ell}$互不同构. 如果存在无穷多个$\ell$使得$E[\ell]$不可约,则有无穷多条$E/X_{\ell}$,与Shafarevich的定理矛盾.
\end{proof}



    \section{Galois表示的像是开子群}

\subsection{Galois模的不可约性}

本小节要证明

\begin{cthm}
    如果$E$在$K$上没有复乘,则:

    (i) 对所有素数$\ell$,$V_{\ell}$不可约;

    (ii) 对除有限个以外的素数$\ell$,$E[\ell]$不可约.
    \label{galois::irreducible}
\end{cthm}

\begin{clem}
    如果$E$在$K$上没有复乘,$E'\to E$和$E''\to E$是两个定义在$K$上的同源,且核是不同构的循环群. 那么$E'$和$E''$在$K$上不同构.
\end{clem}

\begin{proof}
    假设$E'\to E$和$E''\to E$的核分别是阶为$n', n''$的循环群,$E'\to E''$是一个同构. 令$E\to E'$是$E'\to E$的转置,则它的核是$n'$阶循环群. 此时以上几个映射的复合$E\to E'\to E''\to E$的核是$n''$阶循环群关于$n'$阶循环群的群扩张. 但是$\mathrm{End}_K(E) \cong \mathrm{Z}$,于是这个映射只能是某个$a$倍映射,核只能是$(\mathbb{Z}/a\mathbb{Z})^2$的形式. 此时$n'\mid a, n''\mid a$且$n'n''=a^2$,那么只能是$a=n'=n''$,矛盾.
\end{proof}

\begin{proof}
    (定理\ref{galois::irreducible})
    (i) 假设$V_{\ell}$有一个一维的不变子空间$Y$,则$X=Y\bigcap T_{\ell}$是$T_{\ell}$的子模,且$X$和$T_{\ell}/X$都是秩为$1$的自由$\mathbb{Z}_{\ell}$-模. 令$n\geq 0$,记$X(n)$是$X$在$T_{\ell} / \ell^n T_{\ell}$中的像. $X(n)$是$E[\ell^n]$的子模,而且是阶为$\ell^n$的循环群. 那么$X(n)$是$E$的$K$-代数子群. 令$E(n) = E/X(n)$,那么$E\to E(n)$是$K$-同源,且核为$\ell^n$阶循环群. 上面的引理说明不同的$n$对应的$E(n)$互不同构,与Shafarevich的定理\ref{galois::isogeny_finite_curves}矛盾.
    % TODO: quotient of elliptic curves?

    (ii) 如果$E[\ell]$中有一维不变子空间$X_{\ell}$,则类似(i)中的情况,存在同源$E\to E/X_{\ell}$. 由引理,对于不同的$\ell$的取值,$E/X_{\ell}$互不同构. 如果存在无穷多个$\ell$使得$E[\ell]$不可约,则有无穷多条$E/X_{\ell}$,与Shafarevich的定理矛盾.
\end{proof}


\subsection{主定理}

\begin{cthm}
    如果$E$定义的$\ell$进表示$\rho_{\ell}$满足:存在理想$\mathfrak{m}$和$\mathbb{Q}$代数群的态射$\phi: S_{\mathfrak{m}}\to \mathrm{GL}(V)$使得$\rho = \phi_{\ell}$. 那么$E$在$K$上有复乘.
\end{cthm}

\begin{proof}
    假设$E$在$K$上没有复乘. 此时,有一族严格相容的Galois表示$\{\phi_{\ell'}\}$. 选取一个$\ell'$使得$\phi_{\ell'}$在$\mathbb{Q}_{\ell'}$上就可以对角化. 由于$\phi_{\ell'}$和$\rho_{\ell'}$相容,且两者都是半单的,$\phi_{\ell'}\cong \rho_{\ell'}$. 因此$\rho_{\ell'}$是对角化的,与不可约性矛盾.
\end{proof}

\begin{cthm}
    如果$E$定义的$\ell$进表示$\rho_{\ell}$是局部代数的,则$E$在$K$上有复乘. \label{reps::when_cm}
\end{cthm}

\begin{proof}
    由定理\ref{galois::when_sm}. 
\end{proof}

\begin{ccor}
    如果$E$定义的$\ell$进表示是交换的,则$E$在$K$上有复乘.
\end{ccor}

\begin{proof}
    % TODO
\end{proof}

由于Galois群是紧的,$G_{\ell} = \mathrm{Im}(\rho_\ell)$是$\mathrm{GL}_2(\mathbb{Q}_{\ell})$的闭子群. 由非阿基米德的Cartan定理,$\mathrm{Im}(\rho_{\ell})$是一个$\ell$进李群. 令$\mathfrak{g}_{\ell}$为其李代数.

\begin{proof}
    (定理\ref{main::open_image})
    只需要证明$\mathfrak{g}_{\ell} = \mathrm{End}(V_{\ell})$.

    $E$在$K$的所有有限扩张上都没有复乘,从而$G_{\ell}$的任何开子群$U$,$V_{\ell}$都是不可约的$U$模. 那么,$V_{\ell}$是不可约的$\mathfrak{g}_{\ell}$模. 由Schur引理,$\mathrm{End}(V_{\ell})$中与$\mathfrak{g}_{\ell}$交换的元素形成了一个域$\mathfrak{g}'_{\ell}$. 但是$\mathrm{dim}\ V_{\ell}=2$,所以$\mathfrak{g}'_{\ell}$或者是$\mathbb{Q}_{\ell}$,或者是$\mathbb{Q}_{\ell}$的二次扩张.

    如果$\mathfrak{g}'_{\ell} = \mathbb{Q}_{\ell}$,则$\mathfrak{g}_{\ell}$或者是$\mathrm{End}(V_{\ell})$,或者是$\mathfrak{sl}(V_{\ell})$. 但如果$\mathfrak{g}_{\ell} = \mathfrak{sl}(V_{\ell})$,$\mathfrak{g}_{\ell}$在$\wedge^2 V_{\ell}$上的作用是平凡的. 但是由Weil配对,$\wedge^2 V_{\ell}$作为Galois模与$T_{\ell}(\mu)\otimes \mathbb{Q}_{\ell}$同构,矛盾.

    如果$\mathfrak{g}'_{\ell}$是$\mathbb{Q}_{\ell}$的二次扩张,则$V_{\ell}$是$\mathbb{Q}_{\ell}$的二次扩张. 通过将$K$取成有限扩张,不妨设$G_{\ell}$本身是交换的.
    那么$E$在$\overline{K}$上有复乘.

\end{proof}



    \section{模表示和满射性质}

接下来来考虑模$\ell$的表示. $\rho_{\ell}$的像包含在$\mathrm{GL}_2(\mathbb{Z}_{\ell})$中.
令$\overline{\rho}_{\ell}$为$\rho_{\ell}$模$\ell$的表示
$\overline{\rho}_{\ell}: \mathrm{Gal}(\overline{K}/K)\to \mathrm{GL}_2(\mathbb{F}_{\ell})$.
由构造,$\overline{\rho}_{\ell}$就是$\mathrm{Gal}(\overline{K}/K)$作用在$E[\ell]$上得到的表示.

一般地,设$\rho$是任意$n$维$\ell$进表示$\mathrm{Gal}(\overline{K}/K)\to \mathrm{GL}(V)$.
$\rho$的像是紧的,

因为在$E$没有复乘时,已知$\rho_{\ell}$的像是开子群. 如果$\overline{\rho}_{\ell}$是满射,则$\mathrm{Im}\ \rho_{\ell} = \mathrm{GL}_2{\mathbb{Z}_{\ell}} \subset \mathbb{GL}_2(\mathbb{Q}_{\ell})$.

主定理的证明还需要对惯性子群的像以及$\mathrm{GL}_2(\mathbb{F}_{\ell})$的子群进行相当精细的讨论.

\subsection{\texorpdfstring{$\mathrm{GL}_2(\mathbb{F}_p)$}{GL2Fp}的子群}

记$V$是$\mathbb{F}_p$上的二维线性空间,将$\mathrm{GL}_2(\mathbb{F}_p)$看作是$\mathrm{Aut}(V)$.
为了避免小特征带来的麻烦,只考虑$p\geq 7$的情况.

\paragraph{Cartan子群}

Cartan子群是Cartan子代数的类比. 由于基域不是代数闭的,还需要讨论是否分裂的问题.

设$D_1,D_2$是$V$的一维子空间,使得$V=D_1\oplus D_2$.
称$C=\{s\in \mathrm{Aut}(V)\mid sD_1=D_1, sD_2=D_2\}$
为一个分裂的Cartan子群.
选取$D_1, D_2$中的非零元素作为基时,$C$可以写成矩阵群$\left\{\matbt{*}{0}{0}{*}\right\}$.
由此可以看到$|C|=(p-1)^2$.

称$C_1 =\{s\in C\mid \forall x\in D_1, sx=x\}$为一个分裂的半Cartan子群. $C_1$可以写成矩阵群
$\left\{\matbt{1}{0}{0}{*}\right\}$,此时$|C_1|=p-1$.

如果子代数$k\subset \mathrm{End}(V)$同构于$\mathbb{F}_{p^2}$,
则称$C = k^{\times}\subset \mathrm{Aut}(V)$为一个未分裂的Cartan子群.
$|C| = p^2-1$.
将基域扩张到$\mathbb{F}_{p^2}$后,未分裂的Cartan子群可以写成矩阵群
$\left\{\matbt{a}{0}{0}{\overline{a}}, a\in \mathbb{F}_{p^2}^{é}\right\}$.

将分裂和未分裂的Cartan子群统称为Cartan子群.

设$C$是一个Cartan子群,记$N$为$C$的正规化子.

\begin{cprop}
    (i) $[N:C]=2$;

    (ii) 如果$C'$是Cartan子群,$C'\subset N$,则$C' = C$.

    (iii) 如果$C'$是半Cartan子群,$C'\subset N$,则$C' \subset C$.
\end{cprop}

\paragraph{Borel子群}

设$D$是$V$的一维子空间,称$B=\{s\in \mathrm{Aut}(V)\mid sD=D\}$为一个Borel子群.
$B$可以写成矩阵群
$\left\{\matbt{*}{*}{0}{*}\right\}$. $D$是$B$作用下唯一的一维不变子空间.
称共轭于矩阵群$\left\{\matbt{*}{*}{0}{1}\right\}$的子群$B'$为半Borel子群.
容易验证$\absn{B}=p(p-1)^2, \absn{B'}=p(p-1)$.

\begin{cprop}
    如果子群$G\subset \mathrm{GL}_2(\mathbb{F}_p)$的阶被$p$整除,且$G$不包含$\mathrm{SL}_2(\mathbb{F}_p)$,
    则$G$包含在某个Borel子群$B$中.
\end{cprop}

\begin{proof}
    %TODO
\end{proof}

\paragraph{包含Cartan子群的子群}

现在可以开始$\mathrm{GL}_2(\mathbb{F}_p$中的子群分类.
如果能够证明$\tilde{\rho}_{\ell}$的像包含一个Cartan子群,那么像的可能性就不多了.
具体来说,
\begin{cprop}
    如果$G\subset \mathrm{GL}_2(\mathbb{F}_p)$包含了一个Cartan子群,或者一个半Cartan子群,
    那么$G$或者是整个$ \mathrm{GL}_2(\mathbb{F}_p$,
    或者包含在一个Borel子群中,
    或者包含在一个Cartan子群的正规化子中.\label{subgroup_class}
\end{cprop}

\begin{proof}
    %TODO
\end{proof}


\subsection{对惯性子群的控制}

$\tilde{\rho}_{\ell}$作为表示能给出的信息当然比$\rho_{\ell}$本身弱得多了,
因此用模表示限制椭圆曲线的性质比用$\rho_{\ell}$困难许多.
比如说,由$\tilde{\rho}_{\ell}$交换不能直接推出$\rho_{\ell}$交换.
所以还需要多挖掘一些$\tilde{\rho}_{\ell}$的信息.

具体来说,令$v\mid \ell$是$K$的素点,且$v$在$K/\mathbb{Q}$中非分歧,$E$在$v$处有好约化.
$w$是$v$在$\overline{K}$上的延拓.
(对几乎所有的素数$\ell$都可以找到这样的素点$v, w$)
记$G_w=\mathrm{Gal}(\overline{K}_w/K_v)$,$I_w$是$w/v$的惯性子群.
这一小节中要证明
\begin{cprop}
    $\tilde{\rho}_{\ell}(I_w)$或者是一个半Cartan子群,
    或者是一个半Borel子群,
    或者是一个Borel子群. \label{local::inertia}
\end{cprop}

于是上一小节中的结果能够完全分类可能成为$\mathrm{Im}\ \tilde{\rho}_{\ell}$的子群,
接着就可以逐一讨论并排除.
命题\ref{local::inertia}的证明主要是群论的. 或者说,是惯性子群本身的结构限制了表示的可能性.

为了记号方便,固定$v,w$并记$G = G_w, I = I_w$.
{\bfseries 在本小节中,暂时用$K$来记$K_v$}. 记$k,K_{nr},K_{t}$分别是$K$的剩余类域,极大非分歧扩张和极大\tame 扩张.
记$I_p = \mathrm{Gal}(\overline{K}/K_t), I_t = I/I_p$.

首先说明,在$\tilde{\rho}_{\ell}$中起作用的是\tame 部分.

\begin{cprop}
    如果$V$是特征$\ell$的离散的有限或者代数闭域$k$上的有限维线性空间,表示$\rho: G\to \mathrm{Aut}(V)$是半单的,
    那么$\rho(I_p) = 1$.
\end{cprop}

\begin{proof}
    只要对单表示$\rho$证明即可. 令$V'$是$\rho(I_p)$固定的向量的子空间.
    如果$x\in \rho(G),t\in \rho(I_p), v\in V'$,那么由于$I_p$是$G$的正规子群,
    $t(xv) = x(t'v)=xv$,因此$xv\in V'$. 即$V'$是$G$作用不变的子空间.
    由不可约性,只要证明$V'\neq 0$即可.
    
    注意到$\rho(I_p)$是有限$p$群.

    首先,如果$x\in\mathrm{Aut}(V)$是$p$阶元素,则$x$固定了一个非零的向量$v$.
    令$x=s+n$是Jordan-Chevalley分解,则$s^p = 1, n^p = 0$. 那么$s = 1$.
    而$n$的秩小于$\dim V$,因此存在$v\in V$使得$nv=0$. 那么$xv = v$.

    对$m$归纳证明,$p^m$阶群一定固定一个非零元素$v$. 当$m=0$时是显然的,$m=1$时已经证明过了.
    假设$m\geq 1$,命题对不大于$m$的自然数都成立. 考虑一个$p^{m+1}$阶群$H$.
    令$N$是$I$的正规子群,且$N\neq 1,N\neq H$. 这样的$N$总是能找到的.
    例如,当$H$交换时任意取一个,当$H$非交换时取$N$为$H$的中心.
    由归纳假设,$W = \{v\mid Nv=v\}\neq 0$.
    此时,$H/N$作用在$W$上,还是由归纳假设,存在$w'\in W$使得$H/N w' = w'$.
    此时$Hw' = w'$.
\end{proof}

这就是说,要研究所有的半单表示$\rho: I\to \mathrm{Aut}(V)$,只要研究$I_t$的特征就可以了.

先来构造一些$I_t$的特征. 设$d$是正整数,且$(d,\ell) = 1$.
记$\mu_d \subset K_{nr}$为$d$次单位根群.
$\mu_d$在模$\mathfrak{m}_{\overline{K}}$映射下同构于$\overline{k}$中的$d$次单位根群.
令$x\in K_{nr}$使得$v(x) = 1$,$K_d = K_{nr}(x^{\frac{1}{d}})$.
$K_d/K_{nr}$是完全分歧,\tame 的$d$次扩张.
对$s\in \mathrm{Gal}(K_d/K_{nr})$,令$\theta_d(s)$为满足
\begin{equation}
    s(x^{\frac{1}{d}}) = \theta_d(s) x^{\frac{1}{d}}
\end{equation}
则$\theta_d$的定义与$x$和$x^{\frac{1}{d}}$的选取都无关.
$\theta_d$是$\mathrm{Gal}(K_d/K_{ur})$到$\mu_d$的同构.

$K_{t}$是所有$K_d,(d,\ell)=1$的并,因此$\theta_d$的逆向极限
$\theta: I_t \to \lim\limits_{\longleftarrow} \mu_d$是同构.

接下来分类$I_t$的特征,即$\mathrm{Hom}(I_t, \overline{k}^{\times})$.
记$(\mathbb{Q}/\mathbb{Z})'$是$\mathbb{Q}/\mathbb{Z}$中阶与$\ell$互素的元素的集合.
对$\alpha \in (\mathbb{Q}/\mathbb{Z})'$,设其最简分式表示为$\alpha = \frac{a}{d}$.
令$\chi_{\alpha} = \chi_{d}^{a}$.

\begin{cprop}
    $\alpha\mapsto \chi_{\alpha}$是$(\mathbb{Q}/\mathbb{Z})'\to \mathrm{Hom}(I_t, \overline{k}^{\times})$的同构.
\end{cprop}

\begin{proof}
    $I_t$是$\mu_d$的逆向极限,那么$\mathrm{Hom}(I_t, \overline{k}^{\times})$是
    $\mathrm{Hom}(\mu_d, \overline{k}^{\times})$的正向极限.
    但是$\mathrm{Hom}(\mu_d, \overline{k}^{\times}) \cong \frac{1}{d}\mathbb{Z}/\mathbb{Z}$.
\end{proof}

接下来考虑$I_t$的一些具体的作用,并计算这些作用对应的特征.

\paragraph{在$\mathfrak{m}_{\alpha}/\mathfrak{m}_{\alpha}^{+}$上的作用}
对$\alpha\in \mathbb{Q}$,设
\begin{align}
    \mathfrak{m}_{\alpha} &= \{x\in \overline{K}\mid w(x) \geq \alpha \} \\
    \mathfrak{m}_{\alpha} &= \{x\in \overline{K}\mid w(x) > \alpha \}
\end{align}
则$V_{\alpha} = \mathfrak{m}_{\alpha} / \mathfrak{m}_{\alpha}^{+}$是$\overline{k}$上的一维线性空间.
$G$的作用保持$w$不变,因此$G$可以作用在$V_{\alpha}$上.
惯性子群$I$在$\overline{k}$上的作用平凡,从而定义了一个$\overline{k}$线性的作用.
那么就有$\varphi_{\alpha}: I\to \overline{k}^{\times}$.
已经证明过此时$I_p$的作用是平凡的,于是得到特征$\varphi_{\alpha}: I_t\to \overline{k}^{\times}$.

\begin{cprop}
    当$\alpha\in (\mathbb{Q}/\mathbb{Z})'$时,$\varphi_{\alpha} = \chi_{\alpha}$.
\end{cprop}

\begin{proof}
    当$\alpha,\beta\in\mathbb{Q}$时,$\mathfrak{m}_{\alpha} \times \mathfrak{m}_{\beta} \to \mathfrak{m}_{\alpha + \beta}$,$(x, y)\mapsto xy$定义了同构$V_{\alpha} \otimes V_{\beta}\to V_{\alpha + \beta}$.
    且这个同构与$G$的作用交换.
    那么$\varphi_{\alpha+\beta} = \varphi_{\alpha}\varphi_{\beta}$.

    假设$d$是和$\ell$互素的正整数. 令$x$是$K$的素元的$d$次根,则当$s\in I_t$时$s(x) = \theta_d(s) x$.
    那么$\varphi_{\frac{1}{d}} = \chi_{\frac{1}{d}}$.
\end{proof}

特别地,如果$\mu_{\ell}$是$\overline{K}$中的$\ell$次单位根的群,
则$\mu_{\ell}\to V_{\frac{1}{\ell - 1}}, x\to x - 1$是一个保持$G$作用的单射.
从而$I_t$在$\mu_{\ell}$上通过$\theta_{\ell - 1}$作用.

\paragraph{在形式群上的作用}
令$F(X,Y)$是一个$\order_K$系数的形式群.
记$[\ell](X)=\sum_{i=1}^{\infty} a_i X^i$为$F$的$\ell$倍映射.
假设$F$的高度为$h$,即,$a_i\equiv 0\pmod{\mathfrak{m}},\forall i<q$且$a_q\not\equiv 0\pmod{\mathfrak{m}}$,
其中$q=\ell^h$.
$F$在极大理想$\overline{m} = \{x\in \overline{K}\mid w(x)>0\}$上定义了一个群结构.
令$V$是$\ell$倍映射的核,则$V$是$h$维的$\mathbb{F}_{\ell}$向量空间. 此时

\begin{cprop}
    $V$上有一个$\mathbb{F}_q$线性空间的结构,使得$I_p$在$V$上的作用平凡,
    而且$I_t$在$V$上的作用$I_t\to \mathbb{F}_q^{*}$就等于$\theta_{q-1}$.
\end{cprop}

\begin{proof}
    如果$x\in V$,则
    \begin{equation}
        \ell + a_2x + \cdots + a_{q}x^{q-1} + \cdots = 0
    \end{equation}
    因此$v(x) = \frac{1}{q-1}$.
    同时,$F(x,y)\equiv x + y \pmod{\mathfrak{m}_{\alpha}^{+}}$.
    剩下的证明和$\mu_{\ell}$时是一样的.
\end{proof}

现在回到对$\tilde{\rho}_{\ell}$的讨论.
注意到Weil配对给出了保持$G$作用的同构$\wedge^2 E[\ell] \to \mu_{\ell}$.
那么$\tilde{\rho}_{\ell}$的行列式$G\to \mathbb{F}_{\ell}^{\times}$
就是$G$在$\mu_{\ell}$上的作用.
已经证明这个作用限制在$I$上就是$\theta_{\ell-1}$.

\begin{cprop}
    假设$E$在$v$处有高度$1$的约化.
    那么$I_t$的作用半单化之后是两个特征$1$和$\theta_{\ell - 1}$的直和.

    进一步地,如果$I_p$在$E[\ell]$上的作用平凡,则$I$的像是半Cartan子群;
    否则,$I$的像是半Borel子群.
\end{cprop}

\begin{proof}
    模$\overline{\mathfrak{m}}$给出了正合列
    \begin{equation}
        0\to X \to E[\ell] \to \tilde{E}[\ell] \to 0
    \end{equation}
    其中$X$是$\ell$阶循环群. $G$的作用保持$w$,因此保持$X$不变.
    选取$E[\ell]$的一组基$(e_1,e_2)$,使得$X = \mathbb{F}_{\ell}e_1$.
    在这组基下$G$的像包含在Borel子群$\left\{\matbt{*}{*}{0}{*}\right\}$中.
    $I_p$的像是有限$\ell$群,那么一定包含在矩阵群$\left\{\matbt{1}{*}{0}{1}\right\}$中.
    此时$I_p$在$X$和$\tilde{E}[\ell]$上的作用平凡,即$I_t$可以作用在$X,\tilde{E}[\ell]$上.

    设$\chi_X,\chi_Y: I_t\to \mathbb{F}_{\ell}$分别为$I_t$在$X,\tilde{E}[\ell]$上的作用.
    惯性子群$I$在$\tilde{E}[\ell]$上的作用平凡,因此$\chi_Y = 1$.
    而$\chi_X\chi_Y = \theta_{\ell-1}$,从而$\chi_X = \theta_{\ell - 1}$.
\end{proof}

接下来考虑$E$有高度为$2$的好约化的情形. 这时候$\tilde{E}[\ell] = 0$,不能从这里得到什么信息.
但是可以将$E$的群运算写成一个形式群,并利用对$I_t$在形式群上作用的讨论. 此时相应的形式群高度为$2$.
\begin{cprop}
    (i) $I_p$在$E[\ell]$上的作用平凡;

    (ii) $E[\ell]$上有一个$\mathbb{F}_{\ell^2}$线性空间的结构,使得$I_t$在$E[\ell]$上通过$\theta_{\ell^2-1}$作用;

    (iii) $I$的像是一个未分裂的Cartan子群$C$;

    (iv) $G$的像或者是$C$,或者是$N(C)$,分别对应于$k$包含或者不包含$\mathbb{F}_{\ell^2}$的情况.
\end{cprop}

最后考虑$E$在$v$处有坏约化的情形.
只考虑乘性约化,即$\tilde{E}(\overline{k})$的光滑点同构于乘法群$\overline{k}^{\times}$的情况.
此时(\parencite{serre1997abelian, p. IV-31})有$G$模的正合列
\begin{equation}
    0 \to \mu_{\ell} \to E[\ell] \to \mathbb{Z}/\ell \mathbb{Z} \to 0
\end{equation}

类似于高度$1$的好约化,
\begin{cprop}
    此时$I$的像为一个半Borel子群,半单化之后是两个特征$1$和$\theta_{\ell-1}$的直和.
\end{cprop}


\subsection{主定理的证明}

选择$\overline{\mathbb{Q}}$的素点$v\mid \ell$.
$\overline{\mathbb{Q}}$在$v$出的完备化是$\mathbb{Q}_{\ell}$的一个代数闭包.
令$\order_{\ell}, \mathfrak{p}_{\ell}$是$v$对应的整数环和极大理想,
$k_{\ell} = \order_{\ell} / \mathfrak{p}_{\ell}$.

记$\Gamma$是$K\to \overline{\mathbb{Q}}$的嵌入的群.
如果$\sigma\in \Gamma$,则$\sigma$可以延拓为$\mathbb{Q}_{\ell}$代数的映射
$\sigma_{\ell}: K_{\ell} = K\otimes \mathbb{Q}_{\ell}\to \overline{\mathbb{Q}}_{\ell}$.

接下来证明一个类似于局部代数性质的条件.
主要的想法就是,既然一个$\tilde{\rho}_{\ell}$的力量不足以对$\rho_{\ell}$本身产生足够的限制,
那就考虑无穷多个$\tilde{\rho}_{\ell}$.
再利用一个事实,即当一个等式模无穷多个素数$\ell$成立时,它可以被提升到特征$0$的情况.
于是无穷多个$\tilde{\rho}_{\ell}$一起了发挥威力.

假设当素数$\ell$变化时$\{\theta_{\ell}\}$,是一族Galois群的特征
$\theta_{\ell}: \mathrm{Gal}(\overline{K}/K)^{ab}\to k_{\ell}^{\times}$.

\begin{cprop}
    固定一个理想$\mathfrak{m}$.
    如果存在$n:\Gamma: \mathbb{Z}$以及无穷多个素数的集合$L$使得
    对每个$\ell\in L, a\in U_{\mathfrak{m}}$都有
    \begin{equation}
        \theta_{\ell}(a) \equiv \prod_{\sigma\in \Gamma} \sigma_{\ell}(a_{\ell}^{-1})^{n(\sigma)} \pmod{\mathfrak{p}_{\ell}}
    \end{equation}

    那么存在$\psi\in X^{*}(S_{\mathfrak{m}})$使得$\tilde{\phi}_{\ell} = \theta_{\ell}$对无穷多个$\ell\in L$成立.
\end{cprop}

\begin{proof}
    令$\phi$为$T$的特征$\phi = \prod_{\sigma\in \Gamma} \sigma^{n_{\sigma}}$.
    先来验证$\phi$是$T_{\mathfrak{m}}$的特征. 设$x\in E_{\mathfrak{m}}$.
    此时$x\in K$,因此
    \begin{equation}
        \phi(x^{-1}) = \prod_{\sigma\in\Gamma}\sigma(x^{-1})^{n_{\sigma}}
        \equiv \theta_{\ell}(x) \pmod{\mathfrak{p}_{\ell}}
    \end{equation}
    而$\theta_{\ell}(x) = 1$. 那么$\phi(x)\equiv 1\pmod{\mathfrak{p}_{\ell}}$对无穷多个$\ell$成立,
    因此只能是$\phi(x) = 1$.

    取一个$\chi\in X^{*}(S_{\mathfrak{m}})$使得$\chi$在$X^{*}(S_{\mathfrak{m}})\to X^{*}(T_{\mathfrak{m}})$
    下的像是$\phi$. 令$\chi_{\ell}$为$\chi$定义的$\ell$进特征,
    $\tilde{\chi}_{\ell}$为$\chi_{\ell}$模$\mathfrak{p}_{\ell}$的特征.
    令$\theta'_{\ell} = \tilde{\chi}_{\ell} \theta_{\ell}^{-1}$.

    由$\chi_{\ell}$的定义,当$a\in U_{\mathfrak{m}}$时,$\theta'_{\ell}=1$.
    那么$\theta'_{\ell}$可通过$C_{\mathfrak{m}}$分解.
    设$\ell > h_{\mathfrak{m}}$.
    此时模$\mathfrak{p}_{\ell}$的映射是
    $\overline{\mathbb{Q}}$和$\overline{k}_{\ell}$的$h_{\mathfrak{m}}$次单位根之间的双射.
    那么$\theta'_{\mathfrak{m}}$是$C_{\mathfrak{m}}$
    到$\overline{\mathbb{Q}}^{\times}$中的$h_{\mathfrak{m}}$单位根群$\mu_{h_{\mathfrak{m}}}$的映射.
    $C_{\mathfrak{m}}$和$\mu_{h_{\mathfrak{m}}}$都是有限群,这样的映射只有有限多个.
    那么存在$L$的无穷子集$L'$使得每个$\theta'_{\ell},\ell\in L'$都是某一个$\theta''$.
    此时$\psi = \theta^{''-1}\chi$满足当$\ell\in L'$时,$\tilde{\psi}_{\ell} = \theta_{\ell}$.
\end{proof}

完成了对特征的讨论之后,可以来考虑一般的表示.

\begin{cthm}
    如果$\{\rho_{\ell}\}$是一族严格相容的半单、有理$\ell$进表示,$\dim \rho_{\ell} = d$,
    且存在正整数$N$和素数的无穷集合$L$使得
    对所有的$\ell\in L$,$\tilde{\rho}_{\ell}$是交换的. 记$\theta_{\ell}^i$为$\tilde{\rho}_{\ell}$分解中的特征.
    如果存在绝对值不大于$N$的整数
    $n(\sigma,i,\ell)$使得当$a\in U_{\mathfrak{m}}$时,
    \begin{equation}
        \theta_{\ell}^i \equiv \prod_{\sigma\in \Gamma} \sigma_{\ell}(a_{\ell}^{-1})^{n(\sigma, i, \ell)} \pmod{\mathfrak{p}_{\ell}}
    \end{equation}

    那么存在$\phi:S_{\mathfrak{m}}\to \mathrm{GL}_d$使得$\rho_{\ell}\cong \phi_{\ell}$. \label{surj::main_lemma}
\end{cthm}

\begin{proof}
    $n(-,\ell,-)$是有限集合到有限集合的映射,因此可以取出$L$的无穷子集$L'$使得$\ell\in L'$时$n(-,\ell,-)$都是同一个.
    那么${\theta_{\ell}^{1}}$满足上一个命题的条件,
    从而有$L'$的无穷子集$L'_1$和$\psi^1\in X^{*}(S_{\mathfrak{m}})$使得
    当$\ell\in L'_1$时$\tilde{\psi}^1_{\ell} = \theta_{\ell}^{1}$.
    同样地,继续对$L'_i$和${\theta_{\ell}^{i+1}}$用上一个命题,得到$L'_{i}$的无穷子集$L'_{i+1}$
    和$\psi^{i+1}\in X^{*}(S_{\mathfrak{m}})$使得
    当$\ell\in L'_{i+1}$时$\tilde{\psi}^j_{\ell} = \theta_{\ell}^{j},\forall j\leq i+1$.
    最终得到$L$的无穷子集$L''$和$\psi_1,\ldots,\psi_d$使得
    当$\ell\in L''$时$\tilde{\psi}^j_{\ell} = \theta_{\ell}^{j},\forall 1\leq j\leq d$.

    令$\phi:\basechange{S_{\mathfrak{m}}}{\overline{Q}}\to \basechange{\mathrm{GL}_d}{\overline{Q}}$
    是所有$\psi_i$的直和. 接下来验证$\{\phi_{\ell}\}$和$\{\rho_{\ell}\}$相容.
    取一个$\ell\in L''$.
    令$S$是$\mathrm{supp}(\mathfrak{m})$和$\{\rho_{\ell}\}$的例外素点集合的并.
    设$v\in \Sigma_{K}-S$,$f_v$是$v$分量取素元,其它分量取$1$的idèle.
    记$F_v$为$f_v$在$S_{\mathfrak{m}}(\mathbb{Q})$中的像.

    当$v\in \Sigma-S,\ell\neq p_v,\ell\in L''$时,
    \begin{align}
        P'_{v}(t)
        &= \det (1-t\phi(F_v))
        = \prod_{i}(1-t\psi^i(F_v))\\
        &= \prod_{i}(1-t\psi_{\ell}^i(f_v))\\
        &\equiv \prod_{i} (1 - t\theta_{\ell}^{i} (f_v))\pmod{\mathfrak{p}_{\ell}} 
    \end{align}
    即$P'_v(t)\equiv P_{v,\rho_{\ell}}(t) \pmod{\mathfrak{p}_{\ell}}$对无穷多个$\ell$成立.
    注意到此时$P_{v,\rho_{\ell}}(t)$和$\ell$无关. 记其为$P_{v}(t)$.
    那么有$P'_v(t)=P_v(t)$.

    这就对无穷多个素数$\ell$验证了$\phi_{\ell}$和$\rho_{\ell}$相容.
    但两者都是严格相容、半单的,于是对所有的$\ell$都有$\phi_{\ell}\cong \rho_{\ell}$.

    此时$\phi(F_v)$的迹都是有理数,而$F_v$在$S_{\mathfrak{m}}$中稠密,因此$\phi$可以定义在$\mathbb{Q}$上.
\end{proof}

下面开始主定理\ref{main::surjective}的证明.
\begin{proof}[]
    通过将$K$取成一个有限扩张,不妨设$E$是半稳定的,即当$E$在$v$处有坏约化时,
    这个坏约化是乘性的.

    假设有无穷多个$\ell$使得$\mathrm{Im}\ \tilde{\rho}_{\ell}\neq \mathrm{GL}_2(\mathbb{F}_{\ell})$.
    进一步假设$\ell\geq 7$,在$K$中非分歧,且$E$在所有$v\mid \ell$处有好约化.

    固定$K$的素点$v\mid \ell$和$\overline{K}$的素点$w\mid v$.
    当$E$在$v$处有高度$1$的好约化时,$\tilde{\rho}_{\ell}(I_w)$是半Cartan子群或者半Borel子群;
    当$E$在$v$处有高度$2$的好约化时,$\tilde{\rho}_{\ell}(I_w)$是未分裂的Cartan子群.
    
    无论如何,由命题\ref{subgroup_class},$\mathrm{Im}\ \tilde{\rho}_{\ell}$或者包含在一个Cartan子群的正规化子中,或者包含在一个Borel子群中.
    \vskip0.3cm

    先来排除$\mathrm{Im}\ \tilde{\rho}_{\ell}$包含在Cartan子群$C$的正规化子$N_{\ell}$中,却不包含在$C_{\ell}$中的情形.
    假设有无穷多个素数$\ell$落在这一情形中,记这些素数的集合为$L'$. 对每个$\ell\in L'$,
    由于$[N_{\ell}:C_{\ell}]=2$,此时$\tilde{\rho}_{\ell}$定义了映射$\epsilon_{\ell}:\mathrm{Gal}(\overline{K}/K) \to N_{\ell}\to N_{\ell}/C_{\ell} \cong \{\pm 1\}$. 且$\epsilon_{\ell}$是满射.
    $\epsilon_{\ell}$对应于$K$的一个二次扩张$F_{\ell}$.

    下面验证$F_{\ell}$在素点$v$,$E$在$v$处有好约化,以外非分歧.
    如果$p_v=\ell$,那么$\tilde{\rho}_{\ell}(I_w)$或者是一个半Cartan子群,或者是一个未分裂的Cartan子群,因此包含在$C_{\ell}$中. 如果$p_v\neq \ell$且$E$在$v$处有好约化;那么$\rho_{\ell}(I_w) = 1$.

    而$K$的在固定的有限多个素点以外非分歧的二次扩张只有有限多个,那么存在一个$F$使得$F_{\ell} =F$对无穷多个$\ell$成立.
    如果$v$在$F$上惯性且$E$在$v$处有好约化,则$\mathrm{Tr}(F_v) = 0$.
    ($F_v$记$\tilde{E}$上的Frobenius作用)
    这是因为,取一个$\ell\in L',F_{\ell}=F$.
    令$\pi_w$为$w$处$\tilde{\rho}_{\ell}$的Frobenius元素.
    由于$\epsilon_{\ell}(\pi_{w})=-1$,$\pi_{w}\in N_{\ell}-C_{\ell}$,此时$\mathrm{Tr}(\pi_w)=0$.
    那么$\mathrm{Tr}(F_v)\equiv \mathrm{Tr}(\pi_{w})\equiv 0\pmod{\ell}$.
    这个同余关于对无穷多个$\ell$都成立,因此$\mathrm{Tr}(F_v) = 0$.

    由Chebotarev密度定理,使得$\mathrm{Tr}(F_v)= 0$的$v$的密度是$\frac{1}{2}$.
    但是,当$E$没有复乘时,已知$\mathrm{Im}\ \rho_{\ell}$是开子群.
    在$\ell$进李群$\mathrm{Im}\ \rho_{\ell}$中,
    迹为$0$的元素的子集维数不大于$3$,从而Haar测度为$0$.
    取有限扩张塔$\{K(E[\ell^n])\}$并对每个用Chebotarev密度定理可以知道,
    满足$\mathrm{Tr}(F_v)=0$的$v$的密度是$0$. 矛盾.

    \vskip0.3cm

    接下来处理$\mathrm{Im}\ \tilde{\rho}_{\ell}$包含在一个Cartan子群或者一个Borel子群中的情况.
    令$\phi_{\ell}$是$\tilde{\rho}_{\ell}$的半单化,则$\phi_{\ell}$的像是一个Cartan子群.
    特别地,$\phi_{\ell}$是交换的.

    取$\mathfrak{m}=1$,$N=1$. 我们希望验证定理\ref{surj::main_lemma}的条件.
    \begin{equation}
        \theta_{\ell}^i (a) \equiv \prod_{\sigma\in \Gamma} \sigma_{\ell}(a_{\ell}^{-1})^{n(\sigma, \ell,i)}
        \pmod{\mathfrak{p}_{\ell}}
    \end{equation}
    设$p_v\neq \ell$. 当$E$在$v$处有好约化时,$\phi_{\ell}$在$v$处非分歧.
    而当$E$在$v$处有坏约化时,$\tilde{\rho}_{\ell}(I_w)$或者是$1$,或者是一个$\ell$阶循环群.
    因此$\phi_{\ell}(I_w)=1$. 即$\phi_{\ell}$在$p_v\neq \ell$时非分歧.
    那么$p_v\neq \ell$,$a\in U_{\mathfrak{m}, v}$时,$\theta_{\ell}^{i} = 1$.

    因此只要验证存在$n(\sigma, \ell,i)\in \{0, 1\}$使得当$a\in U_{\ell}$时有
    \begin{equation}
        \theta_{\ell}^i (a) \equiv \prod_{\sigma\in \Gamma} \sigma_{\ell}(a^{-1})^{n(\sigma, \ell,i)}
        \pmod{\mathfrak{p}_{\ell}}
    \end{equation}

    %TODO
\end{proof}



    \section{结论}

可以看到,就$\rho_{\ell}$的像的大小,或者说Tate模中各个元素的“代数独立性”来说,
带复乘的椭圆曲线和不带复乘的椭圆曲线有本质的差别,而所有不带复乘的椭圆曲线都是类似的.

复乘现象能够很好地被$\rho_{\ell}$交换来刻画.

    \printbibliography[heading=bibliography,title=参考文献]
\end{document}