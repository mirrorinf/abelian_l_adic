\chapter{一些定量的结果}

假设$E$没有复乘. 此时存在一个$\ell_E$使得当$\ell>\ell_E$时$\tilde{\rho}_{\ell}$是满射.
接下来最自然的问题就是$\ell_{E}$可以有多大.
Serre在\parencite{serre1981quelques}中对$K=\mathbb{Q}$的情况进行了详细的分析,并在GRH下证明了
$\ell_E \leq c(\log N_E) (\log\log 2N_E)^3$. \parencite[][p. 196]{serre1981quelques}

证明的思路与定理\ref{main::surjective}是类似的,也是依靠对$\mathrm{GL}_2(\mathbb{F}_{\ell})$子群的分类.
但是论证的过程要复杂得多.
其中需要用到Chebotarev密度定理和素数定理共同的应用.
这里只对最关键的一步进行介绍.

设$Y$是$N$维$\ell$进解析空间$X = (\mathbb{Z}_{\ell})^N$中的闭子集.
记$X_n,Y_n$分别是$X,Y$模$\ell^n$之后的有限集合.
那么$\lim\limits_{\longleftarrow} Y_n$.
如果$d\in \mathbb{R}_{\geq 0}$,称$Y$的Minkowski维数不大于$d$是在说,
\begin{equation}
    \absn{Y_n} = O(\ell^{nd}),n\to +\infty
\end{equation}
定义$Y$的Minkowski维数为满足以上条件的$d$的上确界.
形如$(\mathbb{Z}_{\ell})^{N}\to (\mathbb{Z}_{\ell})^{N'}$的解析映射都保持Minkowski维数不增加,于是可以
通过解析映射来定义$\ell$进李群中子集的Minkowski维数.
这里的Minkowski维数可以看作是Hausdorff维数的一个变种.

设$E/K$是Galois扩张,其Galois群是$N$维的$\ell$进李群$G$. 设$C$是$G$中在共轭下不变的闭集.
我们想要知道$\pi_C(x)$,即非分歧的,$q_v$不大于$x$的,且Frobenius元素落在$C$中的素点$v$的数量.

记$G_n$是“$G$模$\ell^n$之后的有限群”.
这里$G_n$的具体定义比较复杂,是通过李代数实现的,可以参看\parencite[][p. 151]{serre1981quelques}.
取$K_n$为$\ker G\to G_n$的固定域,则$K_n/K$的Galois群是$G_n$.
对所有的$K_n/K$用(有效形式的)Chebotarev密度定理,并将信息综合起来,就可以得到
\begin{cthm}
    假设GRH.
    设实数$0\leq d<N$,且$C$的Minkowski维数不大于$d$.
    记$\alpha=\frac{N-d}{N}$,则
    \begin{equation}
        \pi_C(x) = O(\mathrm{Li}(x)/\varepsilon(x)^{\alpha}), x\to +\infty
    \end{equation}
    其中$\varepsilon(x) = x^{\frac{1}{2}} (\log x)^{-2}$.
\end{cthm}

由此可以精确地限制Frobenius元素的迹等于某个给定的值的素数的密度,进而对椭圆曲线的几何性质做出限制.

Serre还提出了一致性猜想,即$\ell_{E}$仅与基域$K$有关. 并且当$K = \mathbb{Q}$时,还有猜测$\ell_{E} = 37$.
Bilu等人解决了一致性猜想的某些情形\parencite{bilu2011serre},但是一般的情况仍然未被解决.
Sutherland\parencite{sutherland}借助计算机对Galois表示的像做了非常多的计算,
目前还没有发现$\mathbb{Q}$上使得$\ell_{E}>37$的椭圆曲线$E$.

关于$K\neq \mathbb{Q}$的情况,目前知道的很少.
