\chapter{一些定量的结果}

假设$E$没有复乘. 此时存在一个$\ell_E$使得当$\ell>\ell_E$时$\tilde{\rho}_{\ell}$是满射.
接下来最自然的问题就是$\ell_{E}$可以有多大.
Serre在\parencite{serre1981quelques}中对$K=\mathbb{Q}$的情况进行了详细的分析,并在GRH下证明了可以取
$\ell_E = O((\log N_e) (\log\log 2N_E)^3)$. (\parencite{serre1981quelques}, p. 196)

Serre提出了一致性猜想,即$\ell_{E}$仅与基域$K$有关. 并且当$K = \mathbb{Q}$时,还有猜测$\ell_{E} = 37$.
Sutherland(\parencite{sutherland})对Galois表示的像做了非常多的计算,
目前还没有发现$\mathbb{Q}$上使得$\ell_{E}>37$的椭圆曲线$E$.
关于$K\neq \mathbb{Q}$的情况,目前知道的很少.



