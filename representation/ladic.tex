\subsection{\texorpdfstring{$\ell$}{ELL}进表示的基本定义}

令$\ell$是一个素数,$k$是一个域,$k_s$是$k$的一个可分闭包. 记$G = \mathrm{Gal}(k_{s}/k)$.

\begin{cdef}
    $k$的一个$\ell$进表示是一个连续的同态$\rho: G\to GL(V)$,其中$V$是一个有限维$\mathbb{Q}_{\ell}$向量空间.
\end{cdef}

令$K$是一个代数数域. 记$\Sigma_K$是$K$的有限素点的集合.

\begin{cdef}
    设$v\in \Sigma_K$. 称一个$\ell$进表示$\rho$在$v$处非分歧,当且仅当对某一个$K_{s}$的素点$w\mid v$(等价于对所有的$w\mid v$)有$\rho(I_w) = 1$,$I_w$是惯性子群. 如果$\rho$在$v$处非分歧,则$D_w$中所有Frobenius元素在$\rho$下的像是同一个,称其为$\rho$下的Frobenius元素,记为$F_{w, \rho}$. 所有$w\mid v$定义的Frobenius元素形成$G$的一个共轭类,称为$v$处的Frobenius共轭类.
\end{cdef}

\begin{crem}
    如果$L$是$\ker \rho$的固定域,则$\rho$在$v$处非分歧当且仅当$L$在$v$处非分歧.
\end{crem}

我们需要\Chebotarev 密度定理的一个无穷版本,这里的密度指自然密度:

\begin{cthm}
    令$L$是$K$的Galois扩张,只在有限个素点处分歧. 给$H = \mathrm{Gal}(L/K)$赋Krull拓扑,令$\mu$是相应的Haar测度(注意到紧群都是幺模的),满足$\mu(H) = 1$,那么:

    (1) $L$的非分歧素点定义的Frobenius元素在$H$中稠密

    (2) 令$X$是$H$的一个在共轭下不变的子集,并假设$X$是开集或者$X$是闭集或者$X$的边界集在$\mu$下是零测集. 那么使得Frobenius元素落在$X$中的非分歧素点$v$的密度恰好为$\mu(X)$.
\end{cthm}

\begin{proof}
    只要证明(2). 令$K\subset L_1\subset L_2\subset \cdots \subset L$使得$L = \bigcup_i L_i$. 记$p_i: H\to \mathrm{Gal}(L_i/K)$为自然的投射. 如果$F$是闭集,则$F = \bigcap_i p_i^{-1}(p_i(F))$,因此由有限扩张的\Chebotarev 密度定理,结论对$F$成立. 如果$U$是开集,且$U$在共轭下不变,则$H-U$是在共轭下不变的闭集,结论对$H-U$成立,从而对$U$也成立. 取$F = \overline{X}, U = H - \overline{H-X}$. 但是$\mu(F) = \mu(U)$,因此结论对$X$成立.
\end{proof}

\begin{crem}
    如果$\rho$是一个$\ell$进表示,且在有限多个素点以外非分歧. 令$L$是$\ker(\rho)$的固定域,则$L$也在有限多个素点以外非分歧. 由\Chebotarev 密度定理,Frobenius元素在$\mathrm{Im}(\rho)$中稠密.
\end{crem}

\subsection{有理\texorpdfstring{$\ell$}{ELL}进表示}

令$\rho$是一个$\ell$进表示,$v$是一个素点使得$\rho$在$v$处非分歧. 记$P_{v, \rho}(T) = \det (1 - F_{w, \rho}T)$,其中$w\mid v$.

\begin{cdef}
    称$\rho$为一个有理$\ell$进表示当且仅当存在有限集合$S\subset \Sigma_K$使得当$v\in \Sigma - S$时,$\rho$在$v$处非分歧,而且$P_{v, \rho}$的系数都是有理数. 如果进一步的,$P_{v,\rho}$的系数都是整数,则称$\rho$为整$\ell$进表示.
\end{cdef}

\begin{cdef}
    令$\ell'$为一个素数(不一定和$\ell$不同),$\rho'$为$K$的$\ell'$进表示. 假设$\rho, \rho'$都是有理的. 称$\rho, \rho'$相容当且仅当存在有限集合$S\subset \Sigma_K$使得$\rho, \rho'$在所有$v\in \Sigma_K - S$处都非分歧,且$P_{v,\rho} = P_{v, \rho'}$.
\end{cdef}

令$\rho$为一个$\ell$进表示,$V$是表示空间. 则$V$有合成序列$0 = V_0\subset \cdots \subset V_q = V$,其中$V_{i+1}/V_i$是单$G$模. 令$V' = \bigoplus_i V_{i+1}/V_i$,则$V'$上定义了半单的有理$\ell$进表示,且和$\rho$相容. 称这个表示为$\rho$的半单化.

\begin{cdef}
    假设对每个素数$\ell$有一个有理$\ell$进表示$\rho_{\ell}$. 称$\{\rho_{\ell}\}$为严格相容的,当且仅当存在有限集合$S\subset \Sigma_K$满足:

    (1) 对任意的素数$\ell$,记$S_{\ell} = \{v\mid p_v = \ell\}$. 对所有的$v\in \Sigma_K - S\bigcup S_{\ell}$,$\rho_{\ell}$在$v$处非分歧,且$P_{v, \rho_{\ell}}$的系数都是有理数

    (2) 对任意的素数对$\ell, \ell'$,当$v\in \Sigma_K - S\bigcup S_{\ell}\bigcup S_{\ell'}$时有$P_{v, \rho_{\ell}} = P_{v, \rho_{\ell'}}$.
\end{cdef}

半单的表示可以被特征多项式决定,具体来说:

\begin{clem}
    令$k$是一个特征$0$的域,$A$是$k$代数,$M_1, M_2$是$A$的两个$k$-有限维半单模. 如果$\mathrm{Tr}\circ \rho_1 = \mathrm{Tr}\circ \rho_2$,则$M_1, M_2$同构.
\end{clem}

\begin{proof}
    只需要证明,如果$\{M_i\}$是有限多个互不同构的$k$-有限维不可约模,则$\mathrm{Tr}\ M_i: A\to k$线性无关. 令$N_i\subset A$是$M_i$的零化子,则$N_i$是双边理想,且作为左理想都是极大的. 由于$M_i$互不同构,$N_i$互不相同. 那么对每个$i$,存在$f_i\in A$使得$f_i\equiv 1\pmod{M_i}, f_i\equiv 0\pmod{M_j}, j\neq i$. 此时$\mathrm{Tr}\ M_i(f_i) = \dim_k M_i, \mathrm{Tr}\ M_j(f_i) = 0, j\neq i$.
\end{proof}

\begin{cthm}
    令$\rho$是有理$\ell$进表示,$\ell'$是一个素数(不一定和$\ell$不同). 那么在同构意义下至多有一个半单$\ell'$进表示$\rho'$,使得$\rho$和$\rho'$相容.
\end{cthm}

\begin{proof}
    令$\rho'_1, \rho'_2$是两个半单有理$\ell'$进表示,且都和$\rho$相容. 先证明$\mathrm{Tr}(\rho'_1(g)) = \mathrm{Tr}(\rho'_2(g))$对所有$g\in G$都成立. 记$J = \ker(\rho'_1)\bigcap \ker(\rho'_2)$,$H = G/J$,$M$是$J$对应的扩张. 那么$\mathrm{Gal}(M/K) = H$且$M/K$在有点多个素点以外非分歧. 由\Chebotarev 密度定理,Frobenius元素在$H$中稠密;而$\mathrm{Tr}\circ \rho'_1, \mathrm{Tr}\circ \rho'_1$可以定义为$H$上的连续函数,且在Frobenius元素上都相等,因此在整个$H$上相等,也就在整个$G$上相等. 在引理中取$k = \mathbb{Q}_{\ell'}, A = k[H]$,就得到$\rho'_1, \rho'_2$同构.
\end{proof}

\paragraph{例子:单位根}
假设$\ell\neq \mathrm{char}(K)$. 此时$x^{l^m}-1$是可分多项式,因此$K_s$中的$\ell^m$次单位根的乘法群$\mu_m$同构于$\mathbb{Z}/\ell^m \mathbb{Z}$. $G$作用在$\mu_m$上,因此作用在$\mathbb{Z}_{\ell} \cong T_{\ell}(\mu) = \invlim\mu_m$. 那么有$\chi_{\ell}:G\to \mathbb{Z}_{\ell}^{\times}\subset \mathbb{Q}_{\ell}^{\times}$,即一个一维的$\ell$进表示. 而如果$K$是代数数域,$v\in \Sigma_K, v\nmid \ell$,则$\chi_{\ell}$在$v$处非分歧,而且$F_{v, \chi_{\ell}} = Nv$. 因此$\chi_{\ell}$是一个整的$\ell$进表示. 当$\ell$取遍所有素数时,$\{\chi_{\ell}\}$形成了一族严格相容的有理$\ell$进表示,定义中的$S$可以取为空集.

\subsection{定义在线性代数群中的表示}

\begin{cdef}
    令$H$是一个$k$上的线性代数群,$k[H]$是$H$的坐标环. 若$f\in k[H]$满足对任意的交换$k$代数$k'$以及$x,y\in H(k')$,都有$f(xy)=f(yx)$,则称$f$是中心函数. 若$x\in H(k')$,且对任意的中心函数$f\in k[H]$都有$f(x)\in k$,则称$x$的共轭类是有理的.
\end{cdef}

\begin{cdef}
    令$H$是一个$\mathbb{Q}$上线性代数群. $K$的一个在$H$中取值的$\ell$进表示是指一个连续的$\rho: \mathrm{Gal}(\overline{K}/K)\to H(\mathbb{Q}_{\ell})$.
\end{cdef}

显然地定义非分歧、Frobenius元素.

\begin{cdef}
    称一个在$H$中取值的$\ell$进表示$\rho$为有理的,当且仅当存在有限子集$S\subset \Sigma_K$使得当$v\in \Sigma_K - S$时,$\rho$在$v$处非分歧,而且$F_{v, \rho}$的共轭类是有理的. ($k=\mathbb{Q}, k'=\mathbb{Q}_{\ell}$)
\end{cdef}

显然地定义相容性和严格相容性.

\begin{crem}
    如果$H$是交换的,则$F_{v,\rho}$的共轭类是有理的当且仅当$F_{v,\rho} \in H(\mathbb{Q})$.
\end{crem}

\begin{crem}
    如果$H = GL_n$,则由Chevalley关于不变多项式的定理,$H$上的中心函数形成的子代数就是$k[t_0,\ldots,t_{n-1}, (\det)^{-1}]$,其中$t_0=\det,t_1,\ldots,t_{n-2},t_{n-1}=\mathrm{tr}$是特征多项式的系数. 因此前一节定义的$\ell$进表示的有理性和本节定义的是一致的.
\end{crem}
