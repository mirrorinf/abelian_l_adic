\subsection{局部代数表示}

这一小节给出一个$\ell$进表示可以从$S_{\mathfrak{m}}$定义的充分条件.

先看局部域的情况.
令$L$是$\mathbb{Q}_p$的有限扩张,$T = \mathrm{Res}_{L/\mathbb{Q}_p}(\mathbb{G}_m / L)$.
假设$V$是有限维$\mathbb{Q}_p$向量空间,$\rho$是一个Abel的$p$进表示$\rho: \mathrm{Gal}(\overline{L}/L)^{ab} \to \mathrm{Aut}(V)$.

\begin{cdef}
    称$\rho$是局部代数表示是在说,存在代数群的态射$r:T\to \mathrm{GL}(V)$使得当$x\in L^{\times}$足够接近于$1$时,$\rho\circ \iota(x) = r(x^{-1})$,其中$\iota: L^{\times}\to \mathrm{Gal}(\overline{L}/L) $是局部Artin映射.(这里局部Artin映射约定为$\pi$映射到Frobenius元素的逆)
\end{cdef}

一维的局部代数表示就是代数Hecke特征的$\ell$进类比. 由于$T$是环面,而$\rho$“差不多”是一个$T$的代数表示,可以将$\rho$对角化为一系列一维表示来处理.

\begin{cprop}
    如果$\rho: \mathrm{Gal}(\overline{L}/L)^{ab} \to \mathrm{Aut}(V)$是局部代数的表示,则$\rho$限制在惯性子群上的表示是半单的.
\end{cprop}

\begin{proof}
    由定义,存在$U_L$的开子群$U'$和代数群的态射$r: T\to \mathrm{GL}(V)$使得当$x\in U'$时$\rho\circ \iota(x) = r(x^{-1})$. 如果$W$是$V$的$\rho\circ\iota(U_L)$-不变的子空间,那么$W$也是$\rho\circ\iota(U')=r(U')$-不变的. 又因为$U'$在$T$中Zariski-稠密,$W$也是$r(T)$不变的. 但是$T$是环面群,所有$T$的表示都是半单的. 因此$W$有补空间,即存在投影映射$\pi: V\to W$使得$\pi$和$r(T)$的作用交换. 此时$\pi$也和$\rho\circ\iota(U')$的作用交换. 再令$\pi' = \frac{1}{[U_L:U']} \sum_{s\in U_L/U'} \rho\circ \iota(s) \pi \rho\circ \iota(s^{-1})$,则$\pi'$是和$\rho\circ\iota(U_L)$的作用交换的投影映射. 从而$W$在$V$中有$\rho\circ\iota(U_L)$-不变的补空间.
\end{proof}

固定一个$\rho$,取一个足够大的有限扩张$E/\mathbb{Q}_p$使得(i)$\rho$在$U_L$上的限制在$E$上可以对角化;(ii)$E$包含了$L$的正规闭包. 令$\Gamma_{L}$为所有域嵌入$L\to E$形成的群. 记$\chi_i: U_L\to E^{\times}, i=1,2,\ldots,\mathrm{dim}\ \rho$满足$\rho \cong_{E} \mathrm{diag}(\chi_1,\ldots, \chi_{\mathrm{dim}\ \rho})$.

\begin{cprop}
    $\rho$是局部代数表示当且仅当存在整数$n_{\sigma}(i)$使得对所有的$i$以及足够接近于$1$的$u$都有
    \begin{equation}
        \chi_i(u) = \prod_{\sigma\in \Gamma_K} \sigma(u)^{-n_{\sigma}(i)} \label{eqns::temp::1}
    \end{equation}
\end{cprop}

\begin{proof}
    如果$\rho$是局部代数的,则每个$\chi_i$都是局部代数的,这就等价于\refeq{eqns::temp::1}.

    反之,\refeq{eqns::temp::1}定义了$T$的$E$上的态射$r$,且满足和$\rho$的相容性关系. 只要证明$r$可以定义在$\mathbb{Q}_p$上. 但是在某个小邻域$U'$上,$r$的迹就是$\rho$的迹,从而是$\mathbb{Q}_p$的元素;同时,$U'$在$T$中Zariski-稠密.
\end{proof}

现在回到整体的情况. 仍然令$K$为代数数域. 给定$\ell$进表示$\rho: \mathrm{Gal}(\overline{K}/K)^{ab}\to \mathrm{Aut}(V_{\ell})$. 令$v$是$K$的有限素点,且$p_v = \ell$. 选定一个$\overline{K}$的素点$w\mid v$,则有
\begin{equation}
    \mathrm{Gal}(\overline{K_v}/K_v) \xrightarrow{D_{w}} \mathrm{Gal}(\overline{K}/K) \to \mathrm{Gal}(\overline{K}/K)^{ab}
\end{equation}
由于最终的像是交换群,映射与$w$的选取无关,并可以通过$\mathrm{Gal}(\overline{K_v}/K_v)^{ab}$分解. 因此有表示
\begin{equation}
    \rho_v: \mathrm{Gal}(\overline{K_v}/K_v)^{ab} \to \mathrm{Gal}(\overline{K}/K)^{ab} \xrightarrow{\rho} \mathrm{Aut}(V_{\ell})
\end{equation}

\begin{cdef}
    称$\rho$是局部代数表示是指对所有满足$p_v=\ell$的素点$v$,表示$\rho_v$是局部代数的.
\end{cdef}

局部代数表示也可以用整体的方式来刻画. 记$\iota_{\ell}: K_{\ell}^{\times}=(K\otimes \mathbb{Q}_{\ell})^{\times} \to \mathbb{A}_{K}^{\times} \to \mathrm{Gal}(\overline{K}/K)^{ab}$为显然的单射与整体Artin映射的复合.

\begin{cprop}
    $\rho$是局部代数表示当且仅当存在态射$f: T/\mathbb{Q}_{\ell} \to \mathrm{GL}(V_{\ell})$使得对所有足够接近$1$的$x\in K_{\ell}^{\times}$都有$\rho\circ \iota_{\ell}(x) = f(x^{-1})$.
\end{cprop}

\begin{cdef}
    令$\mathfrak{m}$是$K$的整理想. 称$\rho$可以模$\mathfrak{m}$定义,或者说$\mathfrak{m}$是$\rho$的定义理想,是在说

    (i) 当$p_v\neq \ell$时,$\rho\circ \iota$在$U_{v, \mathfrak{m}}$上平凡;

    (ii) 当$x\in \prod_{v\mid \ell} U_{v, \mathfrak{m}}$时,$\rho\circ\iota_{\ell} (x) = f(x^{-1})$.
\end{cdef}

\begin{cprop}
    每个局部代数表示$\rho$可以模某个$\mathfrak{m}$定义.
\end{cprop}

要证明这个命题,只需要利用一下拓扑性质:

\begin{clem}
    令$H$是$\ell$进李群,设有连续映射$\alpha: \mathbb{A}_{K}^{\times}\to H$. 那么:

    (i) 如果$p_v\neq \ell$,则$\alpha$限制在$K_v^{\times}$上之后,在某个开子群上平凡;

    (ii) 对除了有限多个以外的素点$v$,$\alpha$限制在$U_v$上平凡.

    特别地,所有Abel的$\ell$进表示都在一个有限的素点集合以外非分歧.
\end{clem}

\begin{proof}
    (i) 注意到$K_v^{\times}$是$p_v$进李群,而$p_v\neq \ell$,此时任何连续映射$K_v^{\times}\to H$都通过有限群分解.

    (ii) 令$N$是$H$中$1$的邻域,且$N$中不包含非平凡的有限子群. 对几乎所有的$v$,$\alpha(U_v)\in N$,而且当$p_v\neq \ell$时像是有限子群. 此时$\alpha(U_v)=1$.
\end{proof}

满足$\rho$可以模$\mathfrak{m}$定义的最小理想$\mathfrak{m}$称为$\rho$的导子.

现在可以证明这一小节的主定理:

\begin{cthm}
    设$\rho: \mathrm{Gal}(\overline{K}/K)^{ab} \to \mathrm{Aut}(V_{\ell})$是有理$\ell$进表示,且$\rho$是局部代数的,$\mathfrak{m}$是$\rho$的定义\modulus . 那么存在$V_{\ell}$的$\mathbb{Q}$-向量子空间$V$使得$V_{\ell} = V\otimes \mathbb{Q}_{\ell}$,以及代数群的态射$\phi: S_{\mathfrak{m}}\to \mathrm{GL}(V)$使得$\rho=\phi_{\ell}$. \label{galois::when_sm}
\end{cthm}


\begin{proof}
    设$r: T/\mathbb{Q}_{\ell} \to GL(V_{\ell})$是代数群的态射使得当$x\in K_{\ell}^{\times}\bigcap U_{\mathfrak{m}}$时$\rho\circ\iota(x) = r(x^{-1})$.

    令$\psi: \mathbb{A}_{K}^{\times}\to \mathrm{Aut}(V_{\ell})$为
    \begin{equation}
        \psi(x) = \rho\circ \iota(x) \cdot r(x_{\ell})
    \end{equation}
    其中$x_{\ell}$是$x$的$\ell$部分. 此时$\psi$在$U_{\mathfrak{m}}$上平凡,且在$K^{\times}$上等于$r$.

    那么$r$在$E_{\mathfrak{m}}$($=K^{\times} \bigcap U_{\mathfrak{m}}$)上平凡,即可以通过$T_{\mathfrak{m}}/\mathbb{Q}_{\ell}$分解. 由$S_{\mathfrak{m}}$的定义,存在代数群态射$\phi: S_{\mathfrak{m}}\to \mathrm{GL}(V_{\ell})$使得

    (i) 态射$T_{\mathfrak{m}}/\mathbb{Q}_{\ell} \to S_{\mathfrak{m}}/\mathbb{Q}_{\ell}\xrightarrow{\phi} \mathrm{GL}(V_{\ell})$等于$r_{\mathfrak{m}}$;

    (ii) 映射$\mathbb{A}_{K}^{\times}\to S_{\mathfrak{m}}(\mathbb{Q}_{\ell})\xrightarrow{\phi} \mathrm{Aut}(V_{\ell})$等于$\psi$.

    此时对$a\in \mathbb{A}_{K}^{\times}$都有
    \begin{align}
        \phi_{\ell}\circ\iota(a)
        &= \phi(\epsilon_{\ell}(a)) = \phi(\epsilon(a))\phi(\pi_{\ell}(a_{\ell}^{-1}))\\
        &= \psi(a)\phi(\pi_{\ell}(a_{\ell}^{-1}))\\
        &= \rho\circ\iota(a) r(a_{\ell})\phi(\pi_{\ell}(a_{\ell}^{-1}))\\
        &= \rho\circ\iota(a)
    \end{align}
    因此$\phi_{\ell}=\rho$.
    而$\rho$是有理表示,从而$\phi$可以定义在$\mathbb{Q}$上.
\end{proof}
