\subsection{一些背景知识}

如果$G$是$k$上的仿射代数群,$k'$是$k$的扩张,则记$\basechange{G}{k'}$为$G$通过基域扩张到$k'$得到的$k'$上的代数群.

现在来回顾一些代数数论和代数群理论的内容,用以固定后面要用到的记号.

\subsubsection{代数数论}

令$K$是代数数域,$\mathfrak{m}$是$\order_K$的整理想. 记$U_{\mathfrak{m}, v}$为:当$K_v \cong \mathbb{C}$时,$K_v^{\times}$;当$K_v\cong \mathbb{R}$时,$K_v^{+}$;当$v\nmid\mathfrak{m}\infty$时,$U_{v}$;当$v(\mathfrak{m})=t>0$时,$1+\mathfrak{p}_v^{t}$. 记$U_{\mathfrak{m}}=\prod_v U_{\mathfrak{m}, v}$,$U_{\mathfrak{m}, finite} = \prod_{v\nmid \infty}U_{\mathfrak{m}, v}$(看作$\mathbb{A}_K^{\times}$的子群时,指要求$v\mid \infty$处取值为$1$). $E_{\mathfrak{m}} = K^{\times} \bigcap U_{\mathfrak{m}}$,$I_{\mathfrak{m}} = \mathbb{A}_K^{\times}/U_{\mathfrak{m}}$,$C_{\mathfrak{m}} = \mathbb{A}_K^{\times} / K^{\times}U_{\mathfrak{m}}$.

$T = \mathrm{Res}_{K/\mathbb{Q}}(\basechange{\mathbb{G}_m}{K})$,$T_{\mathfrak{m}} = T / \overline{E_{\mathfrak{m}}}$.

先给出Hecke特征的理想版本和idèle版本一一对应的精确形式:

\begin{cthm}
    以下两类对象之间存在一一对应:

    (1) $\chi: \mathbb{A}_K^{\times}/K^{\times} \to \mathbb{C}^{\times}$,满足$\chi(U_{\mathfrak{m}, finite}) = 1$

    (2) $\tilde{\chi}: J^{\mathfrak{m}}\to \mathbb{C}^{\times}$,使得存在$\chi_{\infty}:\prod_{v\mid \infty}K_v^{\times} \to \mathbb{C}^{\times}$,满足当$a\in K, a\equiv 1\pmod{\mathfrak{m}}$时,$\tilde{\chi}((a)) = \chi_{\infty}(a)$.

    其中映射$\chi\mapsto \tilde{\chi}$是,对所有$v\nmid \mathfrak{m}\infty$,选定一个素元$\pi_v\in \mathcal{O}_v$得到映射$J^{\mathfrak{m}}\to \mathbb{A}_K^{\times}$,再与$\chi$复合.
\end{cthm}

\begin{crem}
    在(2)中对$a$的要求再加上全正,对$\tilde{\chi}$的要求看上去是减弱了,但事实上是一样的. 可以通过强逼近定理转化到idèle版本再还原来看,也可以直接证明:令$K^{\mathfrak{m}} = \{\alpha\in K^{\times}, \alpha\equiv 1\pmod{\mathfrak{m}}\}$,$K^{\mathfrak{m}}_{+} = \{\alpha\in K^{\mathfrak{m}}, \alpha 全正\}$. 如果$\chi_{\infty}$满足当$a\in K^{\mathfrak{m}}_{+}$时,$\tilde{\chi}((a)) = \chi_{\infty}(a)$,则$\tilde{\chi}\chi_{\infty}^{-1}$定义了$K^{\mathfrak{m}}/K^{\mathfrak{m}}_{+}$的一个特征,但是则$K^{\mathfrak{m}}/K^{\mathfrak{m}}_{+} \cong \prod_{v实素点} \{\pm 1\}$. 因此存在一个$\phi: \prod_{v\mid \infty} K_v^{\times}\to \{\pm 1\}$使得在$K^{\mathfrak{m}}$上有$\tilde{\chi}\chi_{\infty}^{-1} = \phi$,于是$\chi_{\infty}\phi$满足定理中(看起来更强的)条件(2).
\end{crem}

\begin{cdef}
    Serre群$S_{\mathfrak{m}}$是一个$\mathbb{Q}$代数群,带着一个代数群态射$T_{\mathfrak{m}}\to S_{\mathfrak{m}}$和一个群同态$\mathbb{A}_K^{\times}/U_{\mathfrak{m}}\to S_{\mathfrak{m}}(\mathbb{Q})$使得图表交换,且是以上条件定义的范畴中的始对象
    \begin{figure}[H]
        \centering
        \begin{tikzcd}
            K^{\times}/E_{\mathfrak{m}}\arrow[r]\arrow[d] &\mathbb{A}_K^{\times}/U_{\mathfrak{m}}\arrow[d]\\
            T_{\mathfrak{m}}(\mathbb{Q})\arrow[r] &S_{\mathfrak{m}}(\mathbb{Q})
        \end{tikzcd}
    \end{figure}
\end{cdef}

\begin{cprop}
    如果$R$是$\mathbb{Q}$代数,且$\mathrm{Spec}\ R$连通,则有交换图表
    \begin{figure}[H]
        \centering
        \begin{tikzcd}
            0\arrow[r] &K^{\times}/E_{\mathfrak{m}}\arrow[r]\arrow[d] &\mathbb{A}_K^{\times}/U_{\mathfrak{m}}\arrow[d] \arrow[r] &C_{\mathfrak{m}}\arrow[d]\arrow[r] &0\\
            0\arrow[r] &T_{\mathfrak{m}}(R)\arrow[r] &S_{\mathfrak{m}}(R) \arrow[r] &C_{\mathfrak{m}}\arrow[r] &0
        \end{tikzcd}
    \end{figure}
    且图表中的两行都是正合列.
\end{cprop}

\begin{cprop}
    $X^{*}(S_{\mathfrak{m}}) = \mathrm{Hom}_{\overline{\mathbb{Q}}}(\basechange{S_{\mathfrak{m}}}{\overline{\mathbb{Q}}}, \basechange{\mathbb{G}_{m}}{\overline{\mathbb{Q}})}$可以被描述为以下对象的集合:$(\phi, \chi)$,其中$\phi: I_{\mathfrak{m}}\to \overline{\mathbb{Q}}^{\times}$,$\chi\in X^{*}(T_{\mathfrak{m}})$满足$\phi(y) = \chi(y), y\in K^{\times}/ E_{\mathfrak{m}}$.
\end{cprop}

以上两个命题可以直接通过$S_{\mathfrak{m}}$的构造来证明.

\begin{cprop}
    固定$\mathfrak{m}$. 存在$\mathbb{Q}$的有限扩张$E$使得对任意有限维表示$\rho:S_{\mathfrak{m}}\to GL(V)$,$\basechange{\rho}{E}$可以对角化. \label{reps::split_finite}
\end{cprop}

\subsubsection{乘法型代数群}

命题\ref{reps::split_finite}的证明需要用到可对角化群和乘法型群的概念,因此专门开一小节来写.
假设$k$是一个特征$0$的域,以下讨论的代数群都是定义在$k$上的仿射代数群.

\paragraph*{可对角化群}

\begin{cprop}
    对任意有限生成交换群$M$(群运算用乘法记),函子$D(M): R\to \mathrm{Hom}(M, R^{\times})$是可由$\mathrm{Spec}\ k[M]$表出.
\end{cprop}

\begin{proof}
    要给出一个$k$代数的态射$k[M]\to R$就是给出一个群同态$M\to R^{\times}$.
\end{proof}

\begin{cdef}
    若一个代数群$G$同构于某个$D(M)$,则称$G$为可对角化的.
\end{cdef}

\begin{cprop}
    $M\to D(M)$和$X: G\mapsto \mathrm{Hom}_k(G, \mathbb{G}_m)$互为伪逆,定义了可对角化代数群和有限生成交换群范畴的反变等价,且$D,X$都是正合的.
\end{cprop}

\begin{proof}
    \cite{milne2017algebraic}, Theorem 12.9.
\end{proof}

\begin{cprop}
    如果$G$可对角化,则$G$的任何有限维表示都(在同一个域$F$上)可对角化,即可以通过$\sigma\in\mathrm{GL}_n(k)$的共轭成为对角矩阵.
\end{cprop}

\begin{proof}
    \cite{milne2017algebraic}, Theorem 12.12.
\end{proof}

\paragraph*{环面群}

\begin{cdef}
    若$G$在$k$的某个扩张上同构于有限个$\mathbb{G}_m$的乘积,则称$G$是环面群. 如果在$k$上就已经是同构的,则称$G$是分裂的环面群.
\end{cdef}

\paragraph*{乘法型群}

\begin{cdef}
    如果$G$在$k$的某个扩张下可对角化,则称$G$为乘法型的.
\end{cdef}

\begin{cprop}
    如果$G$是乘法型群,则$G$在$k$的某个有限扩张上可对角化.
\end{cprop}

\begin{cprop}
    如果$G',G''$都是乘法型群,$G$是交换代数群,且有正合列
    \begin{equation}
        1\to G'\to G\to G''\to 1
    \end{equation}
    则$G$也是乘法型群.
\end{cprop}

\begin{proof}
    \cite{milne2017algebraic}, Corollary 12.22.
\end{proof}

\begin{cprop}
    如果$G$是乘法型群,那么$G$是群扩张
    \begin{equation}
        1\to G'\to G\to G''\to 1
    \end{equation}
    其中$G'$是环面群,$G''$是有限乘法型群.
\end{cprop}

\begin{proof}
    \cite{milne2017algebraic}, Corollary 12.24.
\end{proof}

\subsection{代数群的表示}

设$\Lambda = \mathscr{O}(G)$,$\overline{\Lambda} = \Lambda \otimes \overline{k} = \mathscr{O}(\basechange{G}{\overline{k}})$.
每个$\chi\in X^{*}(G)$都可以看作是$\overline{\Lambda}$的元素.
那么有映射$\alpha: \overline{k}[X^{*}(G)] \to \overline{\Lambda}$,其中$\overline{k}[X^{*}(G)]$是$X^{*}$的群代数.(这里$X^{*}(G)$用乘法记号)

$G = \mathrm{Gal}(\overline{k}/k)$通过$s(\sum a_{\chi} \chi) = \sum s(a_{\chi}) s(\chi)$作用在$\overline{k}[X^{*}(H)]$上.
通过直接计算可以验证$G$的作用和$\alpha$交换.
当$G$是乘法型群时,$\basechange{G}{\overline{k}}$是对角化群,此时$\alpha$是双射.

设$V$是$k$上的有限维线性空间,$\phi: G\to \mathrm{GL}(V)$是一个半单表示.
记$\theta_{\phi}$为$\phi$的迹$\theta_{\phi} = \sum n_{\chi}(\phi)\chi$,
其中$n_{\phi}(\chi)$是$\phi$在$\overline{k}$上的分解中$\chi$的重数.
对于$h\in G(R)$,$\theta_{\phi}(h) = \mathrm{Tr}(\phi(h))$.

记$\mathsf{Rep}_{k}(G)$为$G$的半单表示的范畴. 如果$k'/k$是域扩张,
那么通过基域扩张可以得到映射$\mathsf{Rep}_{k}(G)\to \mathsf{Rep}_{k'}(\basechange{G}{k'})$.
称$V\in \mathsf{Rep}_{k'}(G)$可以定义在$k$上是指$V$落在基域扩张的像(essential image)中.

\begin{cprop}
    $\phi\mapsto \theta_{\phi}$是$\mathsf{Rep}_{k}(G)$的骨架和$S \subset \mathbb{Z}[X^{*}(G)]$之间的双射,
    其中$S$包含了满足以下条件的元素$\theta$:

    (i) $\theta$在$G$的作用下不变;

    (ii) 所有的$n_{\chi}\geq 0$.
\end{cprop}

\begin{ccor}
    $\phi'\in \mathsf{Rep}_{k'}(\basechange{G}{k'})$可以定义在$k$上当且仅当$\theta_{\phi'}\in \Lambda \otimes k'$属于$\Lambda$.
\end{ccor}
