\subsection{在\texorpdfstring{$S_{\mathfrak{m}}$}{Sm}中取值的表示}

记$\varepsilon : \mathbb{A}_K^{\times} \to I_{\mathfrak{m}}\to S_{\mathfrak{m}}(\mathbb{Q})$,$\pi: T \to T_{\mathfrak{m}}\to S_{\mathfrak{m}}$为定义$S_{\mathfrak{m}}$的映射. 对$\pi$取$\mathbb{Q}_{\ell}$点,得到$\pi_{\ell} : T(\mathbb{Q}_{\ell}) \to S_{\mathfrak{m}}(\mathbb{Q}_{\ell})$. 但是$T(\mathbb{Q}_{\ell}) = (K\otimes \mathbb{Q}_l)^{\times} = \prod_{v\mid l} K_v^{\times}$,由此得到映射$\alpha_{\ell} : \mathbb{A}_K^{\times} \xrightarrow{proj} T(\mathbb{Q}_{\ell}) \xrightarrow{\pi_{\ell}} S_{\mathfrak{m}}(\mathbb{Q}_{\ell})$.

\begin{clem}
    以下图表交换,其中的映射或者是上面定义的,或者是自然嵌入和投射
    \begin{figure}[H]
        \centering
        \begin{tikzcd}
            \mathbb{A}_K^{\times} \arrow[ddr, bend right] \arrow[drr, bend left]& & & \\
             &K^{\times} \arrow[ul] \arrow[r]\arrow[d] &K^{\times}/E_{\mathfrak{m}} \arrow[r]\arrow[d] &I_{\mathfrak{m}} \arrow[d]\\
             &T(\mathbb{Q}_{\ell}) \arrow[r] &T_{\mathfrak{m}}(\mathbb{Q}_{\ell}) \arrow[r] &S_{\mathfrak{m}}(\mathbb{Q}_{\ell})
        \end{tikzcd}
    \end{figure}
\end{clem}

因此,$\varepsilon_{\ell}(a) = \varepsilon(a) \alpha_{\ell}(a^{-1})$定义了$C_K \to S_{\mathfrak{m}}(\mathbb{Q}_{\ell})$的映射. 但是$S_{\mathfrak{m}}(\mathbb{Q}_{\ell})$的拓扑是完全不连通的,因此$C_K$的连通分支$D_K$的像为$1$. 那么,由类域论,$\varepsilon_{\ell}$定义了$G^{\mathrm{ab}}\to S_{\mathfrak{m}}(\mathbb{Q}_{\ell})$的映射.

记$F_v = \varepsilon(f_v) \in S_{\mathfrak{m}}(\mathbb{Q})$,其中$f_v$是任何一个在$v$处取素元,在其它地方取$1$的idèle.

\begin{cthm}
    (1) $\varepsilon_{\ell}$是一个取值在$S_{\mathfrak{m}}$中的有理$\ell$进表示

    (2) $\varepsilon_{\ell}$在$v\in \Sigma_K - \mathrm{supp}(\mathfrak{m})\bigcup S_{\ell}$处非分歧,且$F_{v, \varepsilon_{\ell}} = F_v \in S_{\mathfrak{m}}(\mathbb{Q})$.

    (3) $\{\varepsilon_{\ell}\}$形成了一族取值在$S_{\mathfrak{m}}$中的严格相容的$\ell$进表示.
\end{cthm}

\begin{proof}
    如果$v\in \Sigma_K - \mathrm{supp}(\mathfrak{m}), a\in U_v$,则$\epsilon(a) = 1$. 如果进一步地,$v\nmid \ell$,则$\alpha_{\ell}(a) = 1$,因此$\varepsilon_{\ell}$在$v$处非分歧. 同时,$\varepsilon_{\ell}(f_v)=\varepsilon(f_v) = F_v$,即$v$处的Frobenius元素是$F_v$.
\end{proof}

\begin{cthm}
    $\mathrm{Im}(\varepsilon_{\ell})$在$\basechange{S_{\mathfrak{m}}}{\mathbb{Q}_{\ell}}$中Zariski稠密.
\end{cthm}

\begin{proof}
    $\varepsilon$在$U_{\ell, \mathfrak{m}} = \prod_{v\mid \ell} U_{v, \mathfrak{m}}$上平凡,故$\epsilon_{\ell}(U_{\ell, \mathfrak{m}}) = \pi_{\ell}(U_{\ell, \mathfrak{m}})\subset T_{\mathfrak{m}}(\mathbb{Q}_{\ell}) \subset S_{\mathfrak{m}}(\mathbb{Q}_{\ell})$. 因此,$\mathrm{Im}(\varepsilon_{\ell})$是$S_{\mathfrak{m}}(\mathbb{Q}_{\ell})$的($\ell$进拓扑下的)开子群. 另一方面,由素理想在$C_{\mathfrak{m}}$中分布的经典结果,$f_v\in I_{\mathfrak{m}}$在$C_{\mathfrak{m}}$中的像取遍了整个$C_{\mathfrak{m}}$.
    因此$\mathrm{Im}(\varepsilon_{\ell})$在$S_{\mathfrak{m}}(\mathbb{Q}_{\ell})$的Zariski拓扑下稠密.
\end{proof}

\begin{ccor}
    $\{F_v\}$在$S_{\mathfrak{m}}$中Zariski稠密. \label{frob_dense}
\end{ccor}

\begin{proof}
    令$X$表示所有$F_v$的集合. 令$\ell$是素数. 令$\overline{X}, \overline{X}_{\ell}$分别为$X$在$S_{\mathfrak{m}}$(Zariski拓扑),$S_{\mathfrak{m}}(\mathbb{Q}_{\ell})$($\ell$进拓扑)中的闭包. 那么$\overline{X}_{\ell}\subset \overline{X}(\mathbb{Q}_{\ell})$. 但是\Chebotarev 密度定理说明,$X$在$\mathrm{Im}(\varepsilon_{\ell})$中($\ell$进)稠密,即$\overline{X}_{\ell} = \mathrm{Im}(\varepsilon_{\ell})$. $\mathrm{Im}(\varepsilon_{\ell})$在$\basechange{S_{\mathfrak{m}}}{\mathbb{Q}_{\ell}}$(Zariski)稠密,因此$\overline{X}(\mathbb{Q}_{\ell}) = S_{\mathfrak{m}}(\mathbb{Q}_{\ell})$.
\end{proof}

\subsection{通过\texorpdfstring{$S_{\mathfrak{m}}$}{Sm}定义的表示}

设$V_{\ell}$是$\mathbb{Q}_{\ell}$上的有限维线性空间,$\varphi: \basechange{S_{\mathfrak{m}}}{\mathbb{Q}_{\ell}} \to GL(V_{\ell})$是$\basechange{S_{\mathfrak{m}}}{\mathbb{Q}_{\ell}}$的表示. 令$\varphi_{\ell}$为映射$G^{\mathrm{ab}}\to S_{\mathfrak{m}}(\mathbb{Q}_{\ell})\xrightarrow{\varphi} GL(V_{\ell})$.

\begin{cthm}
    (1) $\varphi_{\ell}$是半单表示

    (2) 令$v\in \Sigma_K - \mathrm{supp}(\mathfrak{m})\bigcup S_{\ell}$,则$\varphi_{\ell}$在$v$处非分歧,且$F_{v, \varphi_{\ell}} = \varphi(F_v)$

    (3) $\varphi_{\ell}$是有理表示当且仅当$\varphi$可以在$\mathbb{Q}$上定义 \label{single_ell}
\end{cthm}

\begin{proof}
    $S_{\mathfrak{m}}$是乘法型群,任意表示都可以在有限扩张后对角化,由此得到(1). 由$\epsilon_{\ell}$的非分歧性和Frobenius元素的指定,(2)是显然的.
    为了不打断行文顺序,(3)的证明留到本小节最后给出.
\end{proof}

(3)告诉我们,考察有理表示时,只需要从一个$\mathbb{Q}$上的表示$S_{\mathfrak{m}}\to GL(V)$出发.

设$V$是$\mathbb{Q}$上的有限维空间,$\varphi: S_{\mathfrak{m}}\to GL(V)$是$S_{\mathfrak{m}}$的表示. 令$V_{\ell} = V\otimes \mathbb{Q}_{\ell}$,那么有$\varphi_{\ell}: \basechange{S_{\mathfrak{m}}}{\mathbb{Q}_{\ell}}\to GL(V_{\ell})$. 按照上面的讨论,这就是在说有表示$\varphi_{\ell}: G^{\mathrm{ab}}\to GL(V_{\ell})$. 称如此定义的$\{\varphi_{\ell}\}$为通过$S_{\mathfrak{m}}$定义的.

\begin{cthm}
    (1) $\{\varphi_{\ell}\}$形成了一族严格相容的有理$\ell$进表示,例外集不大于$\mathrm{supp}(\mathfrak{m})$

    (2) 当$v\in \Sigma_K - \mathrm{supp}(\mathfrak{m})\bigcup S_{\ell}$时,$F_{v, \varphi_{\ell}} = \varphi(F_v)$

    (3) 存在无穷多个素数$\ell$使得$\varphi_{\ell}$在$\mathbb{Q}_{\ell}$上可以对角化
\end{cthm}

\begin{proof}
    (1)和(2)由定理\ref{single_ell}对单个$\ell$的计算容易得到. 由命题\ref{reps::split_finite},存在$K$的有限扩张$E$使得$\varphi$在$E$上对角化. 如果$\ell$在$E$上完全分裂,则有嵌入$E\to \mathbb{Q}_{\ell}$,此时$\varphi_{\ell}$在$\mathbb{Q}_{\ell}$上可以对角化. 由\Chebotarev 密度定理,存在无穷多个这样的$\ell$.
\end{proof}

\begin{ccor}
    如果$\{\rho_{\ell}\}$是一族通过$S_{\mathfrak{m}}$定义的$\ell$进表示,则存在无穷很多个素数$\ell$使得$\rho_{\ell}$是一维表示的直和;特别地,$\rho_{\ell}$在$\mathbb{Q}_{\ell}$上可约.\label{sm_to_reducile}
\end{ccor}

现在来证明定理\ref{single_ell}的(3). 回顾,如果$H$是$k$上的代数群,记$\mathsf{Rep}_k(H)$为$H$的有限维半单表示的范畴. 当$k_1$是$k$的扩张是,有自然的$-\otimes k_1: \mathsf{Rep}_k(H)\to \mathsf{Rep}_{k_1}(\basechange{H}{k_1})$. 如果$\rho\in \mathsf{Rep}_{k_1}(H/k_1)$落在$-\otimes k_1$的像(essential image)中,则称$\rho$可以定义在$k$上.

\begin{cprop}
    令$\mathbb{Q}\subset k\subset k_1$. 对于$\varphi\in \mathsf{Rep}_{k_1}(\basechange{S_{\mathfrak{m}}}{k_1})$,以下性质等价:

    (1) $\varphi$可以定义在$k$上

    (2) 对任意的$v\not\in \mathrm{supp}(\mathfrak{m})$,$\varphi(F_v)$的特征多项式的系数都落在$k$中

    (3) 存在密度为$1$的子集$\Sigma \subset \Sigma_K$使得当$v\in \Sigma$时,$\mathrm{Tr}(\varphi(F_v))\in k$.
\end{cprop}

\begin{proof}
    (1)$\Rightarrow$(2)$\Rightarrow$(3)是显然的.
    %TODO: 22 avr
\end{proof}
