\section{几乎所有\texorpdfstring{$\rho_{\ell}$}{rho ell}都是满射}

最后再来证明一个很有意思的结果.
显然当$\rho_{\ell}$是满射时,模表示$\tilde{\rho}_{\ell}$也是满射.
逆命题也(几乎)是成立的.

\begin{cprop}
    假设$\ell \geq 5$. 如果$\tilde{\rho}_{\ell}$是满射,那么$\rho_{\ell}$也是满射.
\end{cprop}

需要一个群论的引理
\begin{clem}[{\parencite[][p. IV-23]{serre1997abelian}}, Lemma 3]
    设$\ell\geq 5$,$X$是$\mathrm{SL}_2(\mathbb{Z}_{\ell})$的闭子群,且在$\mathrm{SL}_2(\mathbb{F}_{\ell})$
    中的像是整个$\mathrm{SL}_2(\mathbb{F}_{\ell})$. 那么$X = \mathrm{SL}_2(\mathbb{Z}_{\ell})$.
\end{clem}

\begin{proof}
    对$n$归纳证明,$X$在$\mathrm{SL}_2(\mathbb{Z}/\ell^n \mathbb{Z})$上的像是满射.
    $n=1$的情况就是假设.

    假设对$n$成立. 只要证明,对所有的$s\in \mathrm{SL}_2(\mathbb{Z}_{\ell}), s\equiv 1\pmod{\ell^n}$,
    都存在一个$x\in X$使得$x\equiv s\pmod{\ell^{n+1}}$.

    将$s$写成$1+\ell^n u$的形式. 由于$\det s = 1$,$\mathrm{Tr}(u)\equiv 0\pmod{\ell^n}$.
    注意到只需要考虑$u$模$\ell$的等价类即可. 由上面的讨论,$u$是$\mathfrak{gl}_2(\mathbb{F}_{\ell})$中迹为$0$的元素.
    但是,这样的元素或者共轭于$\matbt{0}{a}{\sigma a}{0},\sigma\in \mathbb{F}_{\ell^2}-\mathbb{F}_{\ell}$,
    或者共轭于$\matbt{a}{0}{0}{-a}$,
    或者共轭于$\matbt{0}{*}{0}{0}$.
    无论如何都可以写成至多两个平方为$0$的矩阵之和.
    因此,选取一个上面的矩阵的提升,不妨设$u^2=0$.

    由归纳假设,存在$y\in X$使得$y\equiv 1+\ell^{n-1}u \pmod{\ell^n}$. 令
    $v\in \mathfrak{gl}_2(\mathbb{Z}_{\ell})$使得
    $y = 1+\ell^{n-1}u + \ell^n v$. 令$x=y^{\ell}$.
    那么 
    \begin{equation}
        x = 1 + \ell(\ell^{n-1}u + \ell^n v) + {\ell\choose 2}(\ell^{n-1}u+\ell^nv)^2 + \cdots
        + (\ell^{n-1}u+\ell^n v)^{\ell}
    \end{equation}
    如果$n\geq 2$,则$2(n-1) + 1 \geq n + 1$,因此$x\equiv 1+\ell^n u \pmod{\ell^{n+1}}$.
    如果$n=1$,由$u^2=0$,$\ell (u+\ell v)^k \equiv 0\pmod{\ell^2}$,那么
    $x\equiv 1+\ell u + (u + \ell v)^{\ell} \pmod{\ell^2}$.
    但是$\ell\geq 5$,所以
    \begin{equation}
        (u+\ell v)^{\ell} \equiv \ell(uv+vu)u^{\ell-2} \equiv 0\pmod{\ell}
    \end{equation}
    这就证明了$x\equiv 1+\ell^n u\pmod{\ell^{n+1}}$.
\end{proof}

\begin{proof}[命题的证明]
    令$\Gamma = \mathrm{Im}\ \rho_{\ell}\bigcap \mathrm{SL}_2(\mathbb{Z}_{\ell})$,
    $G = \mathrm{Im}\ \tilde{\rho}_{\ell}\bigcap \mathrm{SL}_2(\mathbb{F}_{\ell})$.
    那么$\Gamma, G$分别是$\det: \mathrm{Im}\ \rho_{\ell} \to \mathbb{Z}_{\ell}^{\times}$和
    $\det: \mathrm{Im}\ \tilde{\rho}_{\ell}\to \mathbb{F}_{\ell}^{\times}$的核.

    这两个$\det$都是满射,即有交换图
    \begin{figure}[H]
        \centering
        \begin{tikzcd}
            0 \arrow[r] &\Gamma \arrow[r]\arrow[d, "\mathrm{mod}"] &\mathrm{Im}\ \rho_{\ell} \arrow[r, "\det"]\arrow[d, "\mathrm{mod}"] &\mathbb{Z}_{\ell}^{\times} \arrow[r]\arrow[d, "\mathrm{mod}"] &0 \\
            0 \arrow[r] &G \arrow[r] &\mathrm{Im}\ \tilde{\rho}_{\ell} \arrow[r, "\det"] &\mathbb{F}_{\ell}^{\times} \arrow[r] &0
        \end{tikzcd}
    \end{figure}
    两行都是正合列,且右边两个$\mathrm{mod}$都是满射.
    于是$\mathrm{mod}:\Gamma\to G$是满射.
    
    如果$\tilde{\rho}_{\ell}$是满射,则$G = \mathrm{SL}_2(\mathbb{F}_{\ell})$. 由引理,$\Gamma = \mathrm{SL}_2(\mathbb{Z}_{\ell})$. 于是$\rho_{\ell}$是满射.
\end{proof}
