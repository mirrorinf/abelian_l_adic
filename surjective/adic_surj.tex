\subsection{几乎所有\texorpdfstring{$\rho_{\ell}$}{rho ell}都是满射}

最后再来证明一个很有意思的结果.
显然当$\rho_{\ell}$是满射时,模表示$\tilde{\rho}_{\ell}$也是满射.
逆命题也(几乎)是成立的.

\begin{cprop}
    假设$\ell \geq 5$. 如果$\tilde{\rho}_{\ell}$是满射,那么$\rho_{\ell}$也是满射.
\end{cprop}

需要一个群论的引理
\begin{clem}[\parencite{serre1997abelian}, p. IV-23]
    设$\ell\geq 5$,$X$是$\mathrm{SL}_2(\mathbb{Z}_{\ell})$的闭子群,且在$\mathrm{SL}_2(\mathbb{F}_{\ell})$
    中的像是整个$\mathrm{SL}_2(\mathbb{F}_{\ell})$. 那么$X = \mathrm{SL}_2(\mathbb{Z}_{\ell})$.
\end{clem}

\begin{proof}[命题的证明]
    令$\Gamma = \mathrm{Im}\ \rho_{\ell}\bigcap \mathrm{SL}_2(\mathbb{Z}_{\ell})$,
    $G = \mathrm{Im}\ \tilde{\rho}_{\ell}\bigcap \mathrm{SL}_2(\mathbb{F}_{\ell})$.
    那么$\Gamma, G$分别是$\det: \mathrm{Im}\ \rho_{\ell} \to \mathbb{Z}_{\ell}^{\times}$和
    $\det: \mathrm{Im}\ \tilde{\rho}_{\ell}\to \mathbb{F}_{\ell}^{\times}$的核.

    这两个$\det$都是满射,即有交换图
    \begin{figure}[H]
        \centering
        \begin{tikzcd}
            0 \arrow[r] &\Gamma \arrow[r]\arrow[d, "\mathrm{mod}"] &\mathrm{Im}\ \rho_{\ell} \arrow[r, "\det"]\arrow[d, "\mathrm{mod}"] &\mathbb{Z}_{\ell}^{\times} \arrow[r]\arrow[d, "\mathrm{mod}"] &0 \\
            0 \arrow[r] &G \arrow[r] &\mathrm{Im}\ \tilde{\rho}_{\ell} \arrow[r, "\det"] &\mathbb{F}_{\ell}^{\times} \arrow[r] &0
        \end{tikzcd}
    \end{figure}
    两行都是正合列,且右边两个$\mathrm{mod}$都是满射.
    于是$\mathrm{mod}:\Gamma\to G$是满射.
    
    如果$\tilde{\rho}_{\ell}$是满射,则$G = \mathrm{SL}_2(\mathbb{F}_{\ell})$. 由引理,$\Gamma = \mathrm{SL}_2(\mathbb{Z}_{\ell})$. 于是$\rho_{\ell}$是满射.
\end{proof}
