\subsection{对惯性子群的控制}

$\tilde{\rho}_{\ell}$作为表示能给出的信息当然比$\rho_{\ell}$本身弱得多了,
因此用模表示限制椭圆曲线的性质比用$\rho_{\ell}$困难许多.
比如说,由$\tilde{\rho}_{\ell}$交换不能直接推出$\rho_{\ell}$交换.
所以还需要多挖掘一些$\tilde{\rho}_{\ell}$的信息.

具体来说,令$v\mid \ell$是$K$的素点,且$v$在$K/\mathbb{Q}$中非分歧,$E$在$v$处有好约化.
$w$是$v$在$\overline{K}$上的延拓.
(对几乎所有的素数$\ell$都可以找到这样的素点$v, w$)
记$G_w=\mathrm{Gal}(\overline{K}_w/K_v)$,$I_w$是$w/v$的惯性子群.
这一小节中要证明
\begin{cprop}
    $\tilde{\rho}_{\ell}(I_w)$或者是一个半Cartan子群,
    或者是一个半Borel子群,
    或者是一个Borel子群. \label{local::inertia}
\end{cprop}

于是上一小节中的结果能够完全分类可能成为$\mathrm{Im}\ \tilde{\rho}_{\ell}$的子群,
接着就可以逐一讨论并排除.
命题\ref{local::inertia}的证明主要是群论的. 或者说,是惯性子群本身的结构限制了表示的可能性.

为了记号方便,固定$v,w$并记$G = G_w, I = I_w$.
{\bfseries 在本小节中,暂时用$K$来记$K_v$}. 记$k,K_{nr},K_{t}$分别是$K$的剩余类域,极大非分歧扩张和极大\tame 扩张.
记$I_p = \mathrm{Gal}(\overline{K}/K_t), I_t = I/I_p$.

首先说明,在$\tilde{\rho}_{\ell}$中起作用的是\tame 部分.

\begin{cprop}
    如果$V$是特征$\ell$的离散的有限或者代数闭域$k$上的有限维线性空间,表示$\rho: G\to \mathrm{Aut}(V)$是半单的,
    那么$\rho(I_p) = 1$.
\end{cprop}

\begin{proof}
    只要对单表示$\rho$证明即可. 令$V'$是$\rho(I_p)$固定的向量的子空间.
    如果$x\in \rho(G),t\in \rho(I_p), v\in V'$,那么由于$I_p$是$G$的正规子群,
    $t(xv) = x(t'v)=xv$,因此$xv\in V'$. 即$V'$是$G$作用不变的子空间.
    由不可约性,只要证明$V'\neq 0$即可.
    
    注意到$\rho(I_p)$是有限$p$群.

    首先,如果$x\in\mathrm{Aut}(V)$是$p$阶元素,则$x$固定了一个非零的向量$v$.
    令$x=s+n$是Jordan-Chevalley分解,则$s^p = 1, n^p = 0$. 那么$s = 1$.
    而$n$的秩小于$\dim V$,因此存在$v\in V$使得$nv=0$. 那么$xv = v$.

    对$m$归纳证明,$p^m$阶群一定固定一个非零元素$v$. 当$m=0$时是显然的,$m=1$时已经证明过了.
    假设$m\geq 1$,命题对不大于$m$的自然数都成立. 考虑一个$p^{m+1}$阶群$H$.
    令$N$是$I$的正规子群,且$N\neq 1,N\neq H$. 这样的$N$总是能找到的.
    例如,当$H$交换时任意取一个,当$H$非交换时取$N$为$H$的中心.
    由归纳假设,$W = \{v\mid Nv=v\}\neq 0$.
    此时,$H/N$作用在$W$上,还是由归纳假设,存在$w'\in W$使得$H/N w' = w'$.
    此时$Hw' = w'$.
\end{proof}

这就是说,要研究所有的半单表示$\rho: I\to \mathrm{Aut}(V)$,只要研究$I_t$的特征就可以了.

先来构造一些$I_t$的特征. 设$d$是正整数,且$(d,\ell) = 1$.
记$\mu_d \subset K_{nr}$为$d$次单位根群.
$\mu_d$在模$\mathfrak{m}_{\overline{K}}$映射下同构于$\overline{k}$中的$d$次单位根群.
令$x\in K_{nr}$使得$v(x) = 1$,$K_d = K_{nr}(x^{\frac{1}{d}})$.
$K_d/K_{nr}$是完全分歧,\tame 的$d$次扩张.
对$s\in \mathrm{Gal}(K_d/K_{nr})$,令$\theta_d(s)$为满足
\begin{equation}
    s(x^{\frac{1}{d}}) = \theta_d(s) x^{\frac{1}{d}}
\end{equation}
则$\theta_d$的定义与$x$和$x^{\frac{1}{d}}$的选取都无关.
$\theta_d$是$\mathrm{Gal}(K_d/K_{ur})$到$\mu_d$的同构.

$K_{t}$是所有$K_d,(d,\ell)=1$的并,因此$\theta_d$的逆向极限
$\theta: I_t \to \lim\limits_{\longleftarrow} \mu_d$是同构.

接下来分类$I_t$的特征,即$\mathrm{Hom}(I_t, \overline{k}^{\times})$.
记$(\mathbb{Q}/\mathbb{Z})'$是$\mathbb{Q}/\mathbb{Z}$中阶与$\ell$互素的元素的集合.
对$\alpha \in (\mathbb{Q}/\mathbb{Z})'$,设其最简分式表示为$\alpha = \frac{a}{d}$.
令$\chi_{\alpha} = \chi_{d}^{a}$.

\begin{cprop}
    $\alpha\mapsto \chi_{\alpha}$是$(\mathbb{Q}/\mathbb{Z})'\to \mathrm{Hom}(I_t, \overline{k}^{\times})$的同构.
\end{cprop}

\begin{proof}
    $I_t$是$\mu_d$的逆向极限,那么$\mathrm{Hom}(I_t, \overline{k}^{\times})$是
    $\mathrm{Hom}(\mu_d, \overline{k}^{\times})$的正向极限.
    但是$\mathrm{Hom}(\mu_d, \overline{k}^{\times}) \cong \frac{1}{d}\mathbb{Z}/\mathbb{Z}$.
\end{proof}

接下来考虑$I_t$的一些具体的作用,并计算这些作用对应的特征.

\paragraph{在$\mathfrak{m}_{\alpha}/\mathfrak{m}_{\alpha}^{+}$上的作用}
对$\alpha\in \mathbb{Q}$,设
\begin{align}
    \mathfrak{m}_{\alpha} &= \{x\in \overline{K}\mid w(x) \geq \alpha \} \\
    \mathfrak{m}_{\alpha} &= \{x\in \overline{K}\mid w(x) > \alpha \}
\end{align}
则$V_{\alpha} = \mathfrak{m}_{\alpha} / \mathfrak{m}_{\alpha}^{+}$是$\overline{k}$上的一维线性空间.
$G$的作用保持$w$不变,因此$G$可以作用在$V_{\alpha}$上.
惯性子群$I$在$\overline{k}$上的作用平凡,从而定义了一个$\overline{k}$线性的作用.
那么就有$\varphi_{\alpha}: I\to \overline{k}^{\times}$.
已经证明过此时$I_p$的作用是平凡的,于是得到特征$\varphi_{\alpha}: I_t\to \overline{k}^{\times}$.

\begin{cprop}
    当$\alpha\in (\mathbb{Q}/\mathbb{Z})'$时,$\varphi_{\alpha} = \chi_{\alpha}$.
\end{cprop}

\begin{proof}
    当$\alpha,\beta\in\mathbb{Q}$时,$\mathfrak{m}_{\alpha} \times \mathfrak{m}_{\beta} \to \mathfrak{m}_{\alpha + \beta}$,$(x, y)\mapsto xy$定义了同构$V_{\alpha} \otimes V_{\beta}\to V_{\alpha + \beta}$.
    且这个同构与$G$的作用交换.
    那么$\varphi_{\alpha+\beta} = \varphi_{\alpha}\varphi_{\beta}$.

    假设$d$是和$\ell$互素的正整数. 令$x$是$K$的素元的$d$次根,则当$s\in I_t$时$s(x) = \theta_d(s) x$.
    那么$\varphi_{\frac{1}{d}} = \chi_{\frac{1}{d}}$.
\end{proof}

特别地,如果$\mu_{\ell}$是$\overline{K}$中的$\ell$次单位根的群,
则$\mu_{\ell}\to V_{\frac{1}{\ell - 1}}, x\to x - 1$是一个保持$G$作用的单射.
从而$I_t$在$\mu_{\ell}$上通过$\theta_{\ell - 1}$作用.

\paragraph{在形式群上的作用}
令$F(X,Y)$是一个$\order_K$系数的形式群.
记$[\ell](X)=\sum_{i=1}^{\infty} a_i X^i$为$F$的$\ell$倍映射.
假设$F$的高度为$h$,即,$a_i\equiv 0\pmod{\mathfrak{m}},\forall i<q$且$a_q\not\equiv 0\pmod{\mathfrak{m}}$,
其中$q=\ell^h$.
$F$在极大理想$\overline{m} = \{x\in \overline{K}\mid w(x)>0\}$上定义了一个群结构.
令$V$是$\ell$倍映射的核,则$V$是$h$维的$\mathbb{F}_{\ell}$向量空间. 此时

\begin{cprop}
    $V$上有一个$\mathbb{F}_q$线性空间的结构,使得$I_p$在$V$上的作用平凡,
    而且$I_t$在$V$上的作用$I_t\to \mathbb{F}_q^{*}$就等于$\theta_{q-1}$.
\end{cprop}

\begin{proof}
    如果$x\in V$,则
    \begin{equation}
        \ell + a_2x + \cdots + a_{q}x^{q-1} + \cdots = 0
    \end{equation}
    因此$v(x) = \frac{1}{q-1}$.
    同时,$F(x,y)\equiv x + y \pmod{\mathfrak{m}_{\alpha}^{+}}$.
    剩下的证明和$\mu_{\ell}$时是一样的.
\end{proof}

现在回到对$\tilde{\rho}_{\ell}$的讨论.
注意到Weil配对给出了保持$G$作用的同构$\wedge^2 E[\ell] \to \mu_{\ell}$.
那么$\tilde{\rho}_{\ell}$的行列式$G\to \mathbb{F}_{\ell}^{\times}$
就是$G$在$\mu_{\ell}$上的作用.
已经证明这个作用限制在$I$上就是$\theta_{\ell-1}$.

\begin{cprop}
    假设$E$在$v$处有高度$1$的约化.
    那么$I_t$的作用半单化之后是两个特征$1$和$\theta_{\ell - 1}$的直和.

    进一步地,如果$I_p$在$E[\ell]$上的作用平凡,则$I$的像是半Cartan子群;
    否则,$I$的像是半Borel子群.
\end{cprop}

\begin{proof}
    模$\overline{\mathfrak{m}}$给出了正合列
    \begin{equation}
        0\to X \to E[\ell] \to \tilde{E}[\ell] \to 0
    \end{equation}
    其中$X$是$\ell$阶循环群. $G$的作用保持$w$,因此保持$X$不变.
    选取$E[\ell]$的一组基$(e_1,e_2)$,使得$X = \mathbb{F}_{\ell}e_1$.
    在这组基下$G$的像包含在Borel子群$\left\{\matbt{*}{*}{0}{*}\right\}$中.
    $I_p$的像是有限$\ell$群,那么一定包含在矩阵群$\left\{\matbt{1}{*}{0}{1}\right\}$中.
    此时$I_p$在$X$和$\tilde{E}[\ell]$上的作用平凡,即$I_t$可以作用在$X,\tilde{E}[\ell]$上.

    设$\chi_X,\chi_Y: I_t\to \mathbb{F}_{\ell}$分别为$I_t$在$X,\tilde{E}[\ell]$上的作用.
    惯性子群$I$在$\tilde{E}[\ell]$上的作用平凡,因此$\chi_Y = 1$.
    而$\chi_X\chi_Y = \theta_{\ell-1}$,从而$\chi_X = \theta_{\ell - 1}$.
\end{proof}

接下来考虑$E$有高度为$2$的好约化的情形. 这时候$\tilde{E}[\ell] = 0$,不能从这里得到什么信息.
但是可以将$E$的群运算写成一个形式群,并利用对$I_t$在形式群上作用的讨论. 此时相应的形式群高度为$2$.
\begin{cprop}
    (i) $I_p$在$E[\ell]$上的作用平凡;

    (ii) $E[\ell]$上有一个$\mathbb{F}_{\ell^2}$线性空间的结构,使得$I_t$在$E[\ell]$上通过$\theta_{\ell^2-1}$作用;

    (iii) $I$的像是一个未分裂的Cartan子群$C$;

    (iv) $G$的像或者是$C$,或者是$N(C)$,分别对应于$k$包含或者不包含$\mathbb{F}_{\ell^2}$的情况.
\end{cprop}

最后考虑$E$在$v$处有坏约化的情形.
只考虑乘性约化,即$\tilde{E}(\overline{k})$的光滑点同构于乘法群$\overline{k}^{\times}$的情况.
此时(\parencite{serre1997abelian, p. IV-31})有$G$模的正合列
\begin{equation}
    0 \to \mu_{\ell} \to E[\ell] \to \mathbb{Z}/\ell \mathbb{Z} \to 0
\end{equation}

类似于高度$1$的好约化,
\begin{cprop}
    此时$I$的像为一个半Borel子群,半单化之后是两个特征$1$和$\theta_{\ell-1}$的直和.
\end{cprop}
