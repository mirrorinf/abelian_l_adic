\section{模表示和满射性质}

接下来来考虑模$\ell$的表示. $\rho_{\ell}$的像包含在$\mathrm{GL}_2(\mathbb{Z}_{\ell})$中.
令$\tilde{\rho}_{\ell}$为$\rho_{\ell}$模$\ell$的表示
$\tilde{\rho}_{\ell}: \mathrm{Gal}(\overline{K}/K)\to \mathrm{GL}_2(\mathbb{F}_{\ell})$.
由构造,$\tilde{\rho}_{\ell}$就是$\mathrm{Gal}(\overline{K}/K)$作用在$E[\ell]$上得到的表示.

一般地,设$\rho$是任意$n$维$\ell$进表示$\mathrm{Gal}(\overline{K}/K)\to \mathrm{GL}(V)$.
$\rho$的像是紧的,

因为在$E$没有复乘时,已知$\rho_{\ell}$的像是开子群. 如果$\tilde{\rho}_{\ell}$是满射,则$\mathrm{Im}\ \rho_{\ell} = \mathrm{GL}_2{\mathbb{Z}_{\ell}} \subset \mathbb{GL}_2(\mathbb{Q}_{\ell})$.

主定理的证明还需要对惯性子群的像以及$\mathrm{GL}_2(\mathbb{F}_{\ell})$的子群进行相当精细的讨论.

\subsection{\texorpdfstring{$\mathrm{GL}_2(\mathbb{F}_p)$}{GL2Fp}的子群}

记$V$是$\mathbb{F}_p$上的二维线性空间,将$\mathrm{GL}_2(\mathbb{F}_p)$看作是$\mathrm{Aut}(V)$.
为了避免小特征带来的麻烦,只考虑$p\geq 7$的情况.

\paragraph{Cartan子群}

Cartan子群是Cartan子代数的类比. 由于基域不是代数闭的,还需要讨论是否分裂的问题.

设$D_1,D_2$是$V$的一维子空间,使得$V=D_1\oplus D_2$.
称$C=\{s\in \mathrm{Aut}(V)\mid sD_1=D_1, sD_2=D_2\}$
为一个分裂的Cartan子群.
选取$D_1, D_2$中的非零元素作为基时,$C$可以写成矩阵群$\left\{\matbt{*}{0}{0}{*}\right\}$.
由此可以看到$|C|=(p-1)^2$.

称$C_1 =\{s\in C\mid \forall x\in D_1, sx=x\}$为一个分裂的半Cartan子群. $C_1$可以写成矩阵群
$\left\{\matbt{1}{0}{0}{*}\right\}$,此时$|C_1|=p-1$.

如果子代数$k\subset \mathrm{End}(V)$同构于$\mathbb{F}_{p^2}$,
则称$C = k^{\times}\subset \mathrm{Aut}(V)$为一个未分裂的Cartan子群.
$|C| = p^2-1$.
将基域扩张到$\mathbb{F}_{p^2}$后,未分裂的Cartan子群可以写成矩阵群
$\left\{\matbt{a}{0}{0}{\overline{a}}, a\in \mathbb{F}_{p^2}^{é}\right\}$.

将分裂和未分裂的Cartan子群统称为Cartan子群.

设$C$是一个Cartan子群,记$N$为$C$的正规化子.

\begin{cprop}
    (i) $[N:C]=2$;

    (ii) 如果$C'$是Cartan子群,$C'\subset N$,则$C' = C$.

    (iii) 如果$C'$是半Cartan子群,$C'\subset N$,则$C' \subset C$.
\end{cprop}

\paragraph{Borel子群}

设$D$是$V$的一维子空间,称$B=\{s\in \mathrm{Aut}(V)\mid sD=D\}$为一个Borel子群.
$B$可以写成矩阵群
$\left\{\matbt{*}{*}{0}{*}\right\}$. $D$是$B$作用下唯一的一维不变子空间.
称共轭于矩阵群$\left\{\matbt{*}{*}{0}{1}\right\}$的子群$B'$为半Borel子群.
容易验证$\absn{B}=p(p-1)^2, \absn{B'}=p(p-1)$.

\begin{cprop}
    如果子群$G\subset \mathrm{GL}_2(\mathbb{F}_p)$的阶被$p$整除,且$G$不包含$\mathrm{SL}_2(\mathbb{F}_p)$,
    则$G$包含在某个Borel子群$B$中.
\end{cprop}

\begin{proof}
    %TODO
\end{proof}

\paragraph{包含Cartan子群的子群}

现在可以开始$\mathrm{GL}_2(\mathbb{F}_p$中的子群分类.
如果能够证明$\tilde{\rho}_{\ell}$的像包含一个Cartan子群,那么像的可能性就不多了.
具体来说,
\begin{cprop}
    如果$G\subset \mathrm{GL}_2(\mathbb{F}_p)$包含了一个Cartan子群,或者一个半Cartan子群,
    那么$G$或者是整个$ \mathrm{GL}_2(\mathbb{F}_p$,
    或者包含在一个Borel子群中,
    或者包含在一个Cartan子群的正规化子中.\label{subgroup_class}
\end{cprop}

\begin{proof}
    %TODO
\end{proof}


\subsection{对惯性子群的控制}

$\tilde{\rho}_{\ell}$作为表示能给出的信息当然比$\rho_{\ell}$本身弱得多了,
因此用模表示限制椭圆曲线的性质比用$\rho_{\ell}$困难许多.
比如说,由$\tilde{\rho}_{\ell}$交换不能直接推出$\rho_{\ell}$交换.
所以还需要多挖掘一些$\tilde{\rho}_{\ell}$的信息.

具体来说,令$v\mid \ell$是$K$的素点,且$v$在$K/\mathbb{Q}$中非分歧,$E$在$v$处有好约化.
$w$是$v$在$\overline{K}$上的延拓.
(对几乎所有的素数$\ell$都可以找到这样的素点$v, w$)
记$G_w=\mathrm{Gal}(\overline{K}_w/K_v)$,$I_w$是$w/v$的惯性子群.
这一小节中要证明
\begin{cprop}
    $\tilde{\rho}_{\ell}(I_w)$或者是一个半Cartan子群,
    或者是一个半Borel子群,
    或者是一个Borel子群. \label{local::inertia}
\end{cprop}

于是上一小节中的结果能够完全分类可能成为$\mathrm{Im}\ \tilde{\rho}_{\ell}$的子群,
接着就可以逐一讨论并排除.
命题\ref{local::inertia}的证明主要是群论的. 或者说,是惯性子群本身的结构限制了表示的可能性.

为了记号方便,固定$v,w$并记$G = G_w, I = I_w$.
{\bfseries 在本小节中,暂时用$K$来记$K_v$}. 记$k,K_{nr},K_{t}$分别是$K$的剩余类域,极大非分歧扩张和极大\tame 扩张.
记$I_p = \mathrm{Gal}(\overline{K}/K_t), I_t = I/I_p$.

首先说明,在$\tilde{\rho}_{\ell}$中起作用的是\tame 部分.

\begin{cprop}
    如果$V$是特征$\ell$的离散的有限或者代数闭域$k$上的有限维线性空间,表示$\rho: G\to \mathrm{Aut}(V)$是半单的,
    那么$\rho(I_p) = 1$.
\end{cprop}

\begin{proof}
    只要对单表示$\rho$证明即可. 令$V'$是$\rho(I_p)$固定的向量的子空间.
    如果$x\in \rho(G),t\in \rho(I_p), v\in V'$,那么由于$I_p$是$G$的正规子群,
    $t(xv) = x(t'v)=xv$,因此$xv\in V'$. 即$V'$是$G$作用不变的子空间.
    由不可约性,只要证明$V'\neq 0$即可.
    
    注意到$\rho(I_p)$是有限$p$群.

    首先,如果$x\in\mathrm{Aut}(V)$是$p$阶元素,则$x$固定了一个非零的向量$v$.
    令$x=s+n$是Jordan-Chevalley分解,则$s^p = 1, n^p = 0$. 那么$s = 1$.
    而$n$的秩小于$\dim V$,因此存在$v\in V$使得$nv=0$. 那么$xv = v$.

    对$m$归纳证明,$p^m$阶群一定固定一个非零元素$v$. 当$m=0$时是显然的,$m=1$时已经证明过了.
    假设$m\geq 1$,命题对不大于$m$的自然数都成立. 考虑一个$p^{m+1}$阶群$H$.
    令$N$是$I$的正规子群,且$N\neq 1,N\neq H$. 这样的$N$总是能找到的.
    例如,当$H$交换时任意取一个,当$H$非交换时取$N$为$H$的中心.
    由归纳假设,$W = \{v\mid Nv=v\}\neq 0$.
    此时,$H/N$作用在$W$上,还是由归纳假设,存在$w'\in W$使得$H/N w' = w'$.
    此时$Hw' = w'$.
\end{proof}

这就是说,要研究所有的半单表示$\rho: I\to \mathrm{Aut}(V)$,只要研究$I_t$的特征就可以了.

先来构造一些$I_t$的特征. 设$d$是正整数,且$(d,\ell) = 1$.
记$\mu_d \subset K_{nr}$为$d$次单位根群.
$\mu_d$在模$\mathfrak{m}_{\overline{K}}$映射下同构于$\overline{k}$中的$d$次单位根群.
令$x\in K_{nr}$使得$v(x) = 1$,$K_d = K_{nr}(x^{\frac{1}{d}})$.
$K_d/K_{nr}$是完全分歧,\tame 的$d$次扩张.
对$s\in \mathrm{Gal}(K_d/K_{nr})$,令$\theta_d(s)$为满足
\begin{equation}
    s(x^{\frac{1}{d}}) = \theta_d(s) x^{\frac{1}{d}}
\end{equation}
则$\theta_d$的定义与$x$和$x^{\frac{1}{d}}$的选取都无关.
$\theta_d$是$\mathrm{Gal}(K_d/K_{ur})$到$\mu_d$的同构.

$K_{t}$是所有$K_d,(d,\ell)=1$的并,因此$\theta_d$的逆向极限
$\theta: I_t \to \lim\limits_{\longleftarrow} \mu_d$是同构.

接下来分类$I_t$的特征,即$\mathrm{Hom}(I_t, \overline{k}^{\times})$.
记$(\mathbb{Q}/\mathbb{Z})'$是$\mathbb{Q}/\mathbb{Z}$中阶与$\ell$互素的元素的集合.
对$\alpha \in (\mathbb{Q}/\mathbb{Z})'$,设其最简分式表示为$\alpha = \frac{a}{d}$.
令$\chi_{\alpha} = \chi_{d}^{a}$.

\begin{cprop}
    $\alpha\mapsto \chi_{\alpha}$是$(\mathbb{Q}/\mathbb{Z})'\to \mathrm{Hom}(I_t, \overline{k}^{\times})$的同构.
\end{cprop}

\begin{proof}
    $I_t$是$\mu_d$的逆向极限,那么$\mathrm{Hom}(I_t, \overline{k}^{\times})$是
    $\mathrm{Hom}(\mu_d, \overline{k}^{\times})$的正向极限.
    但是$\mathrm{Hom}(\mu_d, \overline{k}^{\times}) \cong \frac{1}{d}\mathbb{Z}/\mathbb{Z}$.
\end{proof}

接下来考虑$I_t$的一些具体的作用,并计算这些作用对应的特征.

\paragraph{在$\mathfrak{m}_{\alpha}/\mathfrak{m}_{\alpha}^{+}$上的作用}
对$\alpha\in \mathbb{Q}$,设
\begin{align}
    \mathfrak{m}_{\alpha} &= \{x\in \overline{K}\mid w(x) \geq \alpha \} \\
    \mathfrak{m}_{\alpha} &= \{x\in \overline{K}\mid w(x) > \alpha \}
\end{align}
则$V_{\alpha} = \mathfrak{m}_{\alpha} / \mathfrak{m}_{\alpha}^{+}$是$\overline{k}$上的一维线性空间.
$G$的作用保持$w$不变,因此$G$可以作用在$V_{\alpha}$上.
惯性子群$I$在$\overline{k}$上的作用平凡,从而定义了一个$\overline{k}$线性的作用.
那么就有$\varphi_{\alpha}: I\to \overline{k}^{\times}$.
已经证明过此时$I_p$的作用是平凡的,于是得到特征$\varphi_{\alpha}: I_t\to \overline{k}^{\times}$.

\begin{cprop}
    当$\alpha\in (\mathbb{Q}/\mathbb{Z})'$时,$\varphi_{\alpha} = \chi_{\alpha}$.
\end{cprop}

\begin{proof}
    当$\alpha,\beta\in\mathbb{Q}$时,$\mathfrak{m}_{\alpha} \times \mathfrak{m}_{\beta} \to \mathfrak{m}_{\alpha + \beta}$,$(x, y)\mapsto xy$定义了同构$V_{\alpha} \otimes V_{\beta}\to V_{\alpha + \beta}$.
    且这个同构与$G$的作用交换.
    那么$\varphi_{\alpha+\beta} = \varphi_{\alpha}\varphi_{\beta}$.

    假设$d$是和$\ell$互素的正整数. 令$x$是$K$的素元的$d$次根,则当$s\in I_t$时$s(x) = \theta_d(s) x$.
    那么$\varphi_{\frac{1}{d}} = \chi_{\frac{1}{d}}$.
\end{proof}

特别地,如果$\mu_{\ell}$是$\overline{K}$中的$\ell$次单位根的群,
则$\mu_{\ell}\to V_{\frac{1}{\ell - 1}}, x\to x - 1$是一个保持$G$作用的单射.
从而$I_t$在$\mu_{\ell}$上通过$\theta_{\ell - 1}$作用.

\paragraph{在形式群上的作用}
令$F(X,Y)$是一个$\order_K$系数的形式群.
记$[\ell](X)=\sum_{i=1}^{\infty} a_i X^i$为$F$的$\ell$倍映射.
假设$F$的高度为$h$,即,$a_i\equiv 0\pmod{\mathfrak{m}},\forall i<q$且$a_q\not\equiv 0\pmod{\mathfrak{m}}$,
其中$q=\ell^h$.
$F$在极大理想$\overline{m} = \{x\in \overline{K}\mid w(x)>0\}$上定义了一个群结构.
令$V$是$\ell$倍映射的核,则$V$是$h$维的$\mathbb{F}_{\ell}$向量空间. 此时

\begin{cprop}
    $V$上有一个$\mathbb{F}_q$线性空间的结构,使得$I_p$在$V$上的作用平凡,
    而且$I_t$在$V$上的作用$I_t\to \mathbb{F}_q^{*}$就等于$\theta_{q-1}$.
\end{cprop}

\begin{proof}
    如果$x\in V$,则
    \begin{equation}
        \ell + a_2x + \cdots + a_{q}x^{q-1} + \cdots = 0
    \end{equation}
    因此$v(x) = \frac{1}{q-1}$.
    同时,$F(x,y)\equiv x + y \pmod{\mathfrak{m}_{\alpha}^{+}}$.
    剩下的证明和$\mu_{\ell}$时是一样的.
\end{proof}

现在回到对$\tilde{\rho}_{\ell}$的讨论.
注意到Weil配对给出了保持$G$作用的同构$\wedge^2 E[\ell] \to \mu_{\ell}$.
那么$\tilde{\rho}_{\ell}$的行列式$G\to \mathbb{F}_{\ell}^{\times}$
就是$G$在$\mu_{\ell}$上的作用.
已经证明这个作用限制在$I$上就是$\theta_{\ell-1}$.

\begin{cprop}
    假设$E$在$v$处有高度$1$的约化.
    那么$I_t$的作用半单化之后是两个特征$1$和$\theta_{\ell - 1}$的直和.

    进一步地,如果$I_p$在$E[\ell]$上的作用平凡,则$I$的像是半Cartan子群;
    否则,$I$的像是半Borel子群.
\end{cprop}

\begin{proof}
    模$\overline{\mathfrak{m}}$给出了正合列
    \begin{equation}
        0\to X \to E[\ell] \to \tilde{E}[\ell] \to 0
    \end{equation}
    其中$X$是$\ell$阶循环群. $G$的作用保持$w$,因此保持$X$不变.
    选取$E[\ell]$的一组基$(e_1,e_2)$,使得$X = \mathbb{F}_{\ell}e_1$.
    在这组基下$G$的像包含在Borel子群$\left\{\matbt{*}{*}{0}{*}\right\}$中.
    $I_p$的像是有限$\ell$群,那么一定包含在矩阵群$\left\{\matbt{1}{*}{0}{1}\right\}$中.
    此时$I_p$在$X$和$\tilde{E}[\ell]$上的作用平凡,即$I_t$可以作用在$X,\tilde{E}[\ell]$上.

    设$\chi_X,\chi_Y: I_t\to \mathbb{F}_{\ell}$分别为$I_t$在$X,\tilde{E}[\ell]$上的作用.
    惯性子群$I$在$\tilde{E}[\ell]$上的作用平凡,因此$\chi_Y = 1$.
    而$\chi_X\chi_Y = \theta_{\ell-1}$,从而$\chi_X = \theta_{\ell - 1}$.
\end{proof}

接下来考虑$E$有高度为$2$的好约化的情形. 这时候$\tilde{E}[\ell] = 0$,不能从这里得到什么信息.
但是可以将$E$的群运算写成一个形式群,并利用对$I_t$在形式群上作用的讨论. 此时相应的形式群高度为$2$.
\begin{cprop}
    (i) $I_p$在$E[\ell]$上的作用平凡;

    (ii) $E[\ell]$上有一个$\mathbb{F}_{\ell^2}$线性空间的结构,使得$I_t$在$E[\ell]$上通过$\theta_{\ell^2-1}$作用;

    (iii) $I$的像是一个未分裂的Cartan子群$C$;

    (iv) $G$的像或者是$C$,或者是$N(C)$,分别对应于$k$包含或者不包含$\mathbb{F}_{\ell^2}$的情况.
\end{cprop}

最后考虑$E$在$v$处有坏约化的情形.
只考虑乘性约化,即$\tilde{E}(\overline{k})$的光滑点同构于乘法群$\overline{k}^{\times}$的情况.
此时(\parencite{serre1997abelian, p. IV-31})有$G$模的正合列
\begin{equation}
    0 \to \mu_{\ell} \to E[\ell] \to \mathbb{Z}/\ell \mathbb{Z} \to 0
\end{equation}

类似于高度$1$的好约化,
\begin{cprop}
    此时$I$的像为一个半Borel子群,半单化之后是两个特征$1$和$\theta_{\ell-1}$的直和.
\end{cprop}


\subsection{主定理的证明}

选择$\overline{\mathbb{Q}}$的素点$v\mid \ell$.
$\overline{\mathbb{Q}}$在$v$出的完备化是$\mathbb{Q}_{\ell}$的一个代数闭包.
令$\order_{\ell}, \mathfrak{p}_{\ell}$是$v$对应的整数环和极大理想,
$k_{\ell} = \order_{\ell} / \mathfrak{p}_{\ell}$.

记$\Gamma$是$K\to \overline{\mathbb{Q}}$的嵌入的群.
如果$\sigma\in \Gamma$,则$\sigma$可以延拓为$\mathbb{Q}_{\ell}$代数的映射
$\sigma_{\ell}: K_{\ell} = K\otimes \mathbb{Q}_{\ell}\to \overline{\mathbb{Q}}_{\ell}$.

接下来证明一个类似于局部代数性质的条件.
主要的想法就是,既然一个$\tilde{\rho}_{\ell}$的力量不足以对$\rho_{\ell}$本身产生足够的限制,
那就考虑无穷多个$\tilde{\rho}_{\ell}$.
再利用一个事实,即当一个等式模无穷多个素数$\ell$成立时,它可以被提升到特征$0$的情况.
于是无穷多个$\tilde{\rho}_{\ell}$一起了发挥威力.

假设当素数$\ell$变化时$\{\theta_{\ell}\}$,是一族Galois群的特征
$\theta_{\ell}: \mathrm{Gal}(\overline{K}/K)^{ab}\to k_{\ell}^{\times}$.

\begin{cprop}
    固定一个理想$\mathfrak{m}$.
    如果存在$n:\Gamma: \mathbb{Z}$以及无穷多个素数的集合$L$使得
    对每个$\ell\in L, a\in U_{\mathfrak{m}}$都有
    \begin{equation}
        \theta_{\ell}(a) \equiv \prod_{\sigma\in \Gamma} \sigma_{\ell}(a_{\ell}^{-1})^{n(\sigma)} \pmod{\mathfrak{p}_{\ell}}
    \end{equation}

    那么存在$\psi\in X^{*}(S_{\mathfrak{m}})$使得$\tilde{\phi}_{\ell} = \theta_{\ell}$对无穷多个$\ell\in L$成立.
\end{cprop}

\begin{proof}
    令$\phi$为$T$的特征$\phi = \prod_{\sigma\in \Gamma} \sigma^{n_{\sigma}}$.
    先来验证$\phi$是$T_{\mathfrak{m}}$的特征. 设$x\in E_{\mathfrak{m}}$.
    此时$x\in K$,因此
    \begin{equation}
        \phi(x^{-1}) = \prod_{\sigma\in\Gamma}\sigma(x^{-1})^{n_{\sigma}}
        \equiv \theta_{\ell}(x) \pmod{\mathfrak{p}_{\ell}}
    \end{equation}
    而$\theta_{\ell}(x) = 1$. 那么$\phi(x)\equiv 1\pmod{\mathfrak{p}_{\ell}}$对无穷多个$\ell$成立,
    因此只能是$\phi(x) = 1$.

    取一个$\chi\in X^{*}(S_{\mathfrak{m}})$使得$\chi$在$X^{*}(S_{\mathfrak{m}})\to X^{*}(T_{\mathfrak{m}})$
    下的像是$\phi$. 令$\chi_{\ell}$为$\chi$定义的$\ell$进特征,
    $\tilde{\chi}_{\ell}$为$\chi_{\ell}$模$\mathfrak{p}_{\ell}$的特征.
    令$\theta'_{\ell} = \tilde{\chi}_{\ell} \theta_{\ell}^{-1}$.

    由$\chi_{\ell}$的定义,当$a\in U_{\mathfrak{m}}$时,$\theta'_{\ell}=1$.
    那么$\theta'_{\ell}$可通过$C_{\mathfrak{m}}$分解.
    设$\ell > h_{\mathfrak{m}}$.
    此时模$\mathfrak{p}_{\ell}$的映射是
    $\overline{\mathbb{Q}}$和$\overline{k}_{\ell}$的$h_{\mathfrak{m}}$次单位根之间的双射.
    那么$\theta'_{\mathfrak{m}}$是$C_{\mathfrak{m}}$
    到$\overline{\mathbb{Q}}^{\times}$中的$h_{\mathfrak{m}}$单位根群$\mu_{h_{\mathfrak{m}}}$的映射.
    $C_{\mathfrak{m}}$和$\mu_{h_{\mathfrak{m}}}$都是有限群,这样的映射只有有限多个.
    那么存在$L$的无穷子集$L'$使得每个$\theta'_{\ell},\ell\in L'$都是某一个$\theta''$.
    此时$\psi = \theta^{''-1}\chi$满足当$\ell\in L'$时,$\tilde{\psi}_{\ell} = \theta_{\ell}$.
\end{proof}

完成了对特征的讨论之后,可以来考虑一般的表示.

\begin{cthm}
    如果$\{\rho_{\ell}\}$是一族严格相容的半单、有理$\ell$进表示,$\dim \rho_{\ell} = d$,
    且存在正整数$N$和素数的无穷集合$L$使得
    对所有的$\ell\in L$,$\tilde{\rho}_{\ell}$是交换的. 记$\theta_{\ell}^i$为$\tilde{\rho}_{\ell}$分解中的特征.
    如果存在绝对值不大于$N$的整数
    $n(\sigma,i,\ell)$使得当$a\in U_{\mathfrak{m}}$时,
    \begin{equation}
        \theta_{\ell}^i \equiv \prod_{\sigma\in \Gamma} \sigma_{\ell}(a_{\ell}^{-1})^{n(\sigma, i, \ell)} \pmod{\mathfrak{p}_{\ell}}
    \end{equation}

    那么存在$\phi:S_{\mathfrak{m}}\to \mathrm{GL}_d$使得$\rho_{\ell}\cong \phi_{\ell}$. \label{surj::main_lemma}
\end{cthm}

\begin{proof}
    $n(-,\ell,-)$是有限集合到有限集合的映射,因此可以取出$L$的无穷子集$L'$使得$\ell\in L'$时$n(-,\ell,-)$都是同一个.
    那么${\theta_{\ell}^{1}}$满足上一个命题的条件,
    从而有$L'$的无穷子集$L'_1$和$\psi^1\in X^{*}(S_{\mathfrak{m}})$使得
    当$\ell\in L'_1$时$\tilde{\psi}^1_{\ell} = \theta_{\ell}^{1}$.
    同样地,继续对$L'_i$和${\theta_{\ell}^{i+1}}$用上一个命题,得到$L'_{i}$的无穷子集$L'_{i+1}$
    和$\psi^{i+1}\in X^{*}(S_{\mathfrak{m}})$使得
    当$\ell\in L'_{i+1}$时$\tilde{\psi}^j_{\ell} = \theta_{\ell}^{j},\forall j\leq i+1$.
    最终得到$L$的无穷子集$L''$和$\psi_1,\ldots,\psi_d$使得
    当$\ell\in L''$时$\tilde{\psi}^j_{\ell} = \theta_{\ell}^{j},\forall 1\leq j\leq d$.

    令$\phi:\basechange{S_{\mathfrak{m}}}{\overline{Q}}\to \basechange{\mathrm{GL}_d}{\overline{Q}}$
    是所有$\psi_i$的直和. 接下来验证$\{\phi_{\ell}\}$和$\{\rho_{\ell}\}$相容.
    取一个$\ell\in L''$.
    令$S$是$\mathrm{supp}(\mathfrak{m})$和$\{\rho_{\ell}\}$的例外素点集合的并.
    设$v\in \Sigma_{K}-S$,$f_v$是$v$分量取素元,其它分量取$1$的idèle.
    记$F_v$为$f_v$在$S_{\mathfrak{m}}(\mathbb{Q})$中的像.

    当$v\in \Sigma-S,\ell\neq p_v,\ell\in L''$时,
    \begin{align}
        P'_{v}(t)
        &= \det (1-t\phi(F_v))
        = \prod_{i}(1-t\psi^i(F_v))\\
        &= \prod_{i}(1-t\psi_{\ell}^i(f_v))\\
        &\equiv \prod_{i} (1 - t\theta_{\ell}^{i} (f_v))\pmod{\mathfrak{p}_{\ell}} 
    \end{align}
    即$P'_v(t)\equiv P_{v,\rho_{\ell}}(t) \pmod{\mathfrak{p}_{\ell}}$对无穷多个$\ell$成立.
    注意到此时$P_{v,\rho_{\ell}}(t)$和$\ell$无关. 记其为$P_{v}(t)$.
    那么有$P'_v(t)=P_v(t)$.

    这就对无穷多个素数$\ell$验证了$\phi_{\ell}$和$\rho_{\ell}$相容.
    但两者都是严格相容、半单的,于是对所有的$\ell$都有$\phi_{\ell}\cong \rho_{\ell}$.

    此时$\phi(F_v)$的迹都是有理数,而$F_v$在$S_{\mathfrak{m}}$中稠密,因此$\phi$可以定义在$\mathbb{Q}$上.
\end{proof}

下面开始主定理\ref{main::surjective}的证明.
\begin{proof}[]
    通过将$K$取成一个有限扩张,不妨设$E$是半稳定的,即当$E$在$v$处有坏约化时,
    这个坏约化是乘性的.

    假设有无穷多个$\ell$使得$\mathrm{Im}\ \tilde{\rho}_{\ell}\neq \mathrm{GL}_2(\mathbb{F}_{\ell})$.
    进一步假设$\ell\geq 7$,在$K$中非分歧,且$E$在所有$v\mid \ell$处有好约化.

    固定$K$的素点$v\mid \ell$和$\overline{K}$的素点$w\mid v$.
    当$E$在$v$处有高度$1$的好约化时,$\tilde{\rho}_{\ell}(I_w)$是半Cartan子群或者半Borel子群;
    当$E$在$v$处有高度$2$的好约化时,$\tilde{\rho}_{\ell}(I_w)$是未分裂的Cartan子群.
    
    无论如何,由命题\ref{subgroup_class},$\mathrm{Im}\ \tilde{\rho}_{\ell}$或者包含在一个Cartan子群的正规化子中,或者包含在一个Borel子群中.
    \vskip0.3cm

    先来排除$\mathrm{Im}\ \tilde{\rho}_{\ell}$包含在Cartan子群$C$的正规化子$N_{\ell}$中,却不包含在$C_{\ell}$中的情形.
    假设有无穷多个素数$\ell$落在这一情形中,记这些素数的集合为$L'$. 对每个$\ell\in L'$,
    由于$[N_{\ell}:C_{\ell}]=2$,此时$\tilde{\rho}_{\ell}$定义了映射$\epsilon_{\ell}:\mathrm{Gal}(\overline{K}/K) \to N_{\ell}\to N_{\ell}/C_{\ell} \cong \{\pm 1\}$. 且$\epsilon_{\ell}$是满射.
    $\epsilon_{\ell}$对应于$K$的一个二次扩张$F_{\ell}$.

    下面验证$F_{\ell}$在素点$v$,$E$在$v$处有好约化,以外非分歧.
    如果$p_v=\ell$,那么$\tilde{\rho}_{\ell}(I_w)$或者是一个半Cartan子群,或者是一个未分裂的Cartan子群,因此包含在$C_{\ell}$中. 如果$p_v\neq \ell$且$E$在$v$处有好约化;那么$\rho_{\ell}(I_w) = 1$.

    而$K$的在固定的有限多个素点以外非分歧的二次扩张只有有限多个,那么存在一个$F$使得$F_{\ell} =F$对无穷多个$\ell$成立.
    如果$v$在$F$上惯性且$E$在$v$处有好约化,则$\mathrm{Tr}(F_v) = 0$.
    ($F_v$记$\tilde{E}$上的Frobenius作用)
    这是因为,取一个$\ell\in L',F_{\ell}=F$.
    令$\pi_w$为$w$处$\tilde{\rho}_{\ell}$的Frobenius元素.
    由于$\epsilon_{\ell}(\pi_{w})=-1$,$\pi_{w}\in N_{\ell}-C_{\ell}$,此时$\mathrm{Tr}(\pi_w)=0$.
    那么$\mathrm{Tr}(F_v)\equiv \mathrm{Tr}(\pi_{w})\equiv 0\pmod{\ell}$.
    这个同余关于对无穷多个$\ell$都成立,因此$\mathrm{Tr}(F_v) = 0$.

    由Chebotarev密度定理,使得$\mathrm{Tr}(F_v)= 0$的$v$的密度是$\frac{1}{2}$.
    但是,当$E$没有复乘时,已知$\mathrm{Im}\ \rho_{\ell}$是开子群.
    在$\ell$进李群$\mathrm{Im}\ \rho_{\ell}$中,
    迹为$0$的元素的子集维数不大于$3$,从而Haar测度为$0$.
    取有限扩张塔$\{K(E[\ell^n])\}$并对每个用Chebotarev密度定理可以知道,
    满足$\mathrm{Tr}(F_v)=0$的$v$的密度是$0$. 矛盾.

    \vskip0.3cm

    接下来处理$\mathrm{Im}\ \tilde{\rho}_{\ell}$包含在一个Cartan子群或者一个Borel子群中的情况.
    令$\phi_{\ell}$是$\tilde{\rho}_{\ell}$的半单化,则$\phi_{\ell}$的像是一个Cartan子群.
    特别地,$\phi_{\ell}$是交换的.

    取$\mathfrak{m}=1$,$N=1$. 我们希望验证定理\ref{surj::main_lemma}的条件.
    \begin{equation}
        \theta_{\ell}^i (a) \equiv \prod_{\sigma\in \Gamma} \sigma_{\ell}(a_{\ell}^{-1})^{n(\sigma, \ell,i)}
        \pmod{\mathfrak{p}_{\ell}}
    \end{equation}
    设$p_v\neq \ell$. 当$E$在$v$处有好约化时,$\phi_{\ell}$在$v$处非分歧.
    而当$E$在$v$处有坏约化时,$\tilde{\rho}_{\ell}(I_w)$或者是$1$,或者是一个$\ell$阶循环群.
    因此$\phi_{\ell}(I_w)=1$. 即$\phi_{\ell}$在$p_v\neq \ell$时非分歧.
    那么$p_v\neq \ell$,$a\in U_{\mathfrak{m}, v}$时,$\theta_{\ell}^{i} = 1$.

    因此只要验证存在$n(\sigma, \ell,i)\in \{0, 1\}$使得当$a\in U_{\ell}$时有
    \begin{equation}
        \theta_{\ell}^i (a) \equiv \prod_{\sigma\in \Gamma} \sigma_{\ell}(a^{-1})^{n(\sigma, \ell,i)}
        \pmod{\mathfrak{p}_{\ell}}
    \end{equation}

    %TODO
\end{proof}


%\section{几乎所有\texorpdfstring{$\rho_{\ell}$}{rho ell}都是满射}

最后再来证明一个很有意思的结果.
显然当$\rho_{\ell}$是满射时,模表示$\tilde{\rho}_{\ell}$也是满射.
逆命题也(几乎)是成立的.

\begin{cprop}
    假设$\ell \geq 5$. 如果$\tilde{\rho}_{\ell}$是满射,那么$\rho_{\ell}$也是满射.
\end{cprop}

需要一个群论的引理
\begin{clem}[{\parencite[][p. IV-23]{serre1997abelian}}, Lemma 3]
    设$\ell\geq 5$,$X$是$\mathrm{SL}_2(\mathbb{Z}_{\ell})$的闭子群,且在$\mathrm{SL}_2(\mathbb{F}_{\ell})$
    中的像是整个$\mathrm{SL}_2(\mathbb{F}_{\ell})$. 那么$X = \mathrm{SL}_2(\mathbb{Z}_{\ell})$.
\end{clem}

\begin{proof}
    对$n$归纳证明,$X$在$\mathrm{SL}_2(\mathbb{Z}/\ell^n \mathbb{Z})$上的像是满射.
    $n=1$的情况就是假设.

    假设对$n$成立. 只要证明,对所有的$s\in \mathrm{SL}_2(\mathbb{Z}_{\ell}), s\equiv 1\pmod{\ell^n}$,
    都存在一个$x\in X$使得$x\equiv s\pmod{\ell^{n+1}}$.

    将$s$写成$1+\ell^n u$的形式. 由于$\det s = 1$,$\mathrm{Tr}(u)\equiv 0\pmod{\ell^n}$.
    注意到只需要考虑$u$模$\ell$的等价类即可. 由上面的讨论,$u$是$\mathfrak{gl}_2(\mathbb{F}_{\ell})$中迹为$0$的元素.
    但是,这样的元素或者共轭于$\matbt{0}{a}{\sigma a}{0},\sigma\in \mathbb{F}_{\ell^2}-\mathbb{F}_{\ell}$,
    或者共轭于$\matbt{a}{0}{0}{-a}$,
    或者共轭于$\matbt{0}{*}{0}{0}$.
    无论如何都可以写成至多两个平方为$0$的矩阵之和.
    因此,选取一个上面的矩阵的提升,不妨设$u^2=0$.

    由归纳假设,存在$y\in X$使得$y\equiv 1+\ell^{n-1}u \pmod{\ell^n}$. 令
    $v\in \mathfrak{gl}_2(\mathbb{Z}_{\ell})$使得
    $y = 1+\ell^{n-1}u + \ell^n v$. 令$x=y^{\ell}$.
    那么 
    \begin{equation}
        x = 1 + \ell(\ell^{n-1}u + \ell^n v) + {\ell\choose 2}(\ell^{n-1}u+\ell^nv)^2 + \cdots
        + (\ell^{n-1}u+\ell^n v)^{\ell}
    \end{equation}
    如果$n\geq 2$,则$2(n-1) + 1 \geq n + 1$,因此$x\equiv 1+\ell^n u \pmod{\ell^{n+1}}$.
    如果$n=1$,由$u^2=0$,$\ell (u+\ell v)^k \equiv 0\pmod{\ell^2}$,那么
    $x\equiv 1+\ell u + (u + \ell v)^{\ell} \pmod{\ell^2}$.
    但是$\ell\geq 5$,所以
    \begin{equation}
        (u+\ell v)^{\ell} \equiv \ell(uv+vu)u^{\ell-2} \equiv 0\pmod{\ell}
    \end{equation}
    这就证明了$x\equiv 1+\ell^n u\pmod{\ell^{n+1}}$.
\end{proof}

\begin{proof}[命题的证明]
    令$\Gamma = \mathrm{Im}\ \rho_{\ell}\bigcap \mathrm{SL}_2(\mathbb{Z}_{\ell})$,
    $G = \mathrm{Im}\ \tilde{\rho}_{\ell}\bigcap \mathrm{SL}_2(\mathbb{F}_{\ell})$.
    那么$\Gamma, G$分别是$\det: \mathrm{Im}\ \rho_{\ell} \to \mathbb{Z}_{\ell}^{\times}$和
    $\det: \mathrm{Im}\ \tilde{\rho}_{\ell}\to \mathbb{F}_{\ell}^{\times}$的核.

    这两个$\det$都是满射,即有交换图
    \begin{figure}[H]
        \centering
        \begin{tikzcd}
            0 \arrow[r] &\Gamma \arrow[r]\arrow[d, "\mathrm{mod}"] &\mathrm{Im}\ \rho_{\ell} \arrow[r, "\det"]\arrow[d, "\mathrm{mod}"] &\mathbb{Z}_{\ell}^{\times} \arrow[r]\arrow[d, "\mathrm{mod}"] &0 \\
            0 \arrow[r] &G \arrow[r] &\mathrm{Im}\ \tilde{\rho}_{\ell} \arrow[r, "\det"] &\mathbb{F}_{\ell}^{\times} \arrow[r] &0
        \end{tikzcd}
    \end{figure}
    两行都是正合列,且右边两个$\mathrm{mod}$都是满射.
    于是$\mathrm{mod}:\Gamma\to G$是满射.
    
    如果$\tilde{\rho}_{\ell}$是满射,则$G = \mathrm{SL}_2(\mathbb{F}_{\ell})$. 由引理,$\Gamma = \mathrm{SL}_2(\mathbb{Z}_{\ell})$. 于是$\rho_{\ell}$是满射.
\end{proof}

