\subsection{\texorpdfstring{$\mathrm{GL}_2(\mathbb{F}_p)$}{GL2Fp}的子群}

记$V$是$\mathbb{F}_p$上的二维线性空间,将$\mathrm{GL}_2(\mathbb{F}_p)$看作是$\mathrm{Aut}(V)$.
为了避免小特征带来的麻烦,只考虑$p\geq 7$的情况.

\paragraph{Cartan子群}

Cartan子群是Cartan子代数的类比. 由于基域不是代数闭的,还需要讨论是否分裂的问题.

设$D_1,D_2$是$V$的一维子空间,使得$V=D_1\oplus D_2$.
称$C=\{s\in \mathrm{Aut}(V)\mid sD_1=D_1, sD_2=D_2\}$
为一个分裂的Cartan子群.
选取$D_1, D_2$中的非零元素作为基时,$C$可以写成矩阵群$\left\{\matbt{*}{0}{0}{*}\right\}$.
由此可以看到$|C|=(p-1)^2$.

称$C_1 =\{s\in C\mid \forall x\in D_1, sx=x\}$为一个分裂的半Cartan子群. $C_1$可以写成矩阵群
$\left\{\matbt{1}{0}{0}{*}\right\}$,此时$|C_1|=p-1$.

如果子代数$k\subset \mathrm{End}(V)$同构于$\mathbb{F}_{p^2}$,
则称$C = k^{\times}\subset \mathrm{Aut}(V)$为一个未分裂的Cartan子群.
$|C| = p^2-1$.
将基域扩张到$\mathbb{F}_{p^2}$后,未分裂的Cartan子群可以写成矩阵群
$\left\{\matbt{a}{0}{0}{\overline{a}}, a\in \mathbb{F}_{p^2}^{é}\right\}$.

将分裂和未分裂的Cartan子群统称为Cartan子群.

设$C$是一个Cartan子群,记$N$为$C$的正规化子.

\begin{cprop}
    (i) $[N:C]=2$;

    (ii) 如果$C'$是Cartan子群,$C'\subset N$,则$C' = C$.

    (iii) 如果$C'$是半Cartan子群,$C'\subset N$,则$C' \subset C$.
\end{cprop}

\paragraph{Borel子群}

设$D$是$V$的一维子空间,称$B=\{s\in \mathrm{Aut}(V)\mid sD=D\}$为一个Borel子群.
$B$可以写成矩阵群
$\left\{\matbt{*}{*}{0}{*}\right\}$. $D$是$B$作用下唯一的一维不变子空间.
称共轭于矩阵群$\left\{\matbt{*}{*}{0}{1}\right\}$的子群$B'$为半Borel子群.
容易验证$\absn{B}=p(p-1)^2, \absn{B'}=p(p-1)$.

\begin{cprop}
    如果子群$G\subset \mathrm{GL}_2(\mathbb{F}_p)$的阶被$p$整除,且$G$不包含$\mathrm{SL}_2(\mathbb{F}_p)$,
    则$G$包含在某个Borel子群$B$中.
\end{cprop}

\begin{proof}
    %TODO
\end{proof}

\paragraph{包含Cartan子群的子群}

现在可以开始$\mathrm{GL}_2(\mathbb{F}_p$中的子群分类.
如果能够证明$\tilde{\rho}_{\ell}$的像包含一个Cartan子群,那么像的可能性就不多了.
具体来说,
\begin{cprop}
    如果$G\subset \mathrm{GL}_2(\mathbb{F}_p)$包含了一个Cartan子群,或者一个半Cartan子群,
    那么$G$或者是整个$ \mathrm{GL}_2(\mathbb{F}_p$,
    或者包含在一个Borel子群中,
    或者包含在一个Cartan子群的正规化子中.\label{subgroup_class}
\end{cprop}

\begin{proof}
    %TODO
\end{proof}
