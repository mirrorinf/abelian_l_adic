\section{\texorpdfstring{$\mathrm{GL}_2(\mathbb{F}_p)$}{GL2Fp}的子群}

记$V$是$\mathbb{F}_p$上的二维线性空间,将$\mathrm{GL}_2(\mathbb{F}_p)$看作是$\mathrm{Aut}(V)$.
为了避免小特征带来的麻烦,只考虑$p\geq 7$的情况.

\paragraph{Cartan子群}

Cartan子群是Cartan子代数的类比. 由于基域不是代数闭的,还需要讨论是否分裂的问题.

设$D_1,D_2$是$V$的一维子空间,使得$V=D_1\oplus D_2$.
称$C=\{s\in \mathrm{Aut}(V)\mid sD_1=D_1, sD_2=D_2\}$
为一个分裂的Cartan子群.
选取$D_1, D_2$中的非零元素作为基时,$C$可以写成矩阵群$\left\{\matbt{*}{0}{0}{*}\right\}$.
由此可以看到$|C|=(p-1)^2$.

称$C_1 =\{s\in C\mid \forall x\in D_1, sx=x\}$为一个半Cartan子群. $C_1$可以写成矩阵群
$\left\{\matbt{1}{0}{0}{*}\right\}$,此时$|C_1|=p-1$.

如果子代数$k\subset \mathrm{End}(V)$同构于$\mathbb{F}_{p^2}$,
则称$C = k^{\times}\subset \mathrm{Aut}(V)$为一个未分裂的Cartan子群.
$|C| = p^2-1$.
将基域扩张到$\mathbb{F}_{p^2}$后,未分裂的Cartan子群可以写成矩阵群
$\left\{\matbt{a}{0}{0}{\overline{a}}, a\in \mathbb{F}_{p^2}^{é}\right\}$.

将分裂和未分裂的Cartan子群统称为Cartan子群.

注意到半Cartan子群乘上中心之后得到唯一的包含它的分裂Cartan子群.
分裂的Cartan子群在$\mathrm{PGL}_2(\mathbb{F}_p)$中的像是$p-1$阶的循环群,
而未分裂的Cartan子群在$\mathrm{PGL}_2(\mathbb{F}_p)$中的像是$p+1$阶的循环群.

设$C$是一个Cartan子群,记$N$为$C$的正规化子.

\begin{cprop}
    (i) $[N:C]=2$;

    (ii) 如果$C'$是Cartan子群,$C'\subset N$,则$C' = C$.

    (iii) 如果$C'$是半Cartan子群,$C'\subset N$,则$C' \subset C$.
\end{cprop}

\begin{proof}
    (i) 通过简单的矩阵计算就知道如果$C$是分裂的Cartan子群,则当$C$的元素写成形如$\matbt{*}{0}{0}{*}$的形式时,
    $N$的元素或者属于$C$,或者写成$\matbt{0}{*}{*}{0}$的形式.
    而如果$C$是未分裂的Cartan子群,则基域扩张到$\mathbb{F}_{p^2}$之后,$C$的元素可以写成$\matbt{x}{0}{0}{x^t}$的形式,
    此时$N-C$的元素可以写成$\matbt{0}{y}{y^t}{0}$的形式.

    (ii)(iii) 令$C_1,N_1,C'_1$分别是$C,N,C'$在$\mathrm{PGL}_2(\mathbb{F}_p)$中的像.
    (如果$C'$是半Cartan子群,则令$C_1'$为$C'\mathbb{F}_p$在$\mathrm{PGL}_2(\mathbb{F}_p)$中的像)
    那么$C'_1$是$p\pm 1$阶的循环群. 而$\mathrm{PGL}_2(\mathbb{F}_p)$中两个不同的Cartan子群的像的交只能是$1$,
    因此只能是$C_1=C'_1$.
\end{proof}

\paragraph{Borel子群}

设$D$是$V$的一维子空间,称$B=\{s\in \mathrm{Aut}(V)\mid sD=D\}$为一个Borel子群.
$B$可以写成矩阵群
$\left\{\matbt{*}{*}{0}{*}\right\}$. $D$是$B$作用下唯一的一维不变子空间.
称共轭于矩阵群$\left\{\matbt{*}{*}{0}{1}\right\}$的子群$B'$为半Borel子群.
容易验证$\absn{B}=p(p-1)^2, \absn{B'}=p(p-1)$.

\begin{cprop}
    如果子群$G\subset \mathrm{GL}_2(\mathbb{F}_p)$的阶被$p$整除,且$G$不包含$\mathrm{SL}_2(\mathbb{F}_p)$,
    则$G$包含在某个Borel子群$B$中.
\end{cprop}

\begin{proof}
    $\mathrm{GL}_2(\mathbb{F}_p)$中的$p$阶元素$x$都共轭到$\matbt{1}{1}{0}{1}$,
    因此逐点固定唯一的一维子空间$D_x$.

    如果有两个$p$阶元$x,y$对应的$D_x,D_y$不相同,则将这两个子空间的生成元选为一组基,就可以把$x,y$写成矩阵
    $\matbt{1}{a}{0}{1}$和$\matbt{1}{0}{b}{1}$.
    这两个元素可以生成$\mathrm{SL}_2(\mathbb{F_p})$,因为
    \begin{equation}
        \matbt{1}{1}{0}{1}
        \matbt{1}{0}{-1}{1}
        \matbt{1}{1}{0}{1}
        = \matbt{0}{1}{-1}{0}
    \end{equation}

    假设所有$p$阶元$x$对应的$D_x$是同一个. 对任何$w\in G$,$wxw^{-1}$仍然是$p$阶元素.
    从而$wD_x = D_x$. 于是$GD_x=D_x$,即$G$包含在一个Borel子群中.
\end{proof}

\paragraph{包含Cartan子群的子群}

现在可以开始$\mathrm{GL}_2(\mathbb{F}_p)$中的子群分类.
如果能够证明$\tilde{\rho}_{\ell}$的像包含一个Cartan子群,那么像的可能性就不多了.
具体来说,
\begin{cprop}
    如果$G\subset \mathrm{GL}_2(\mathbb{F}_p)$包含了一个Cartan子群,或者一个半Cartan子群,
    那么$G$或者是整个$ \mathrm{GL}_2(\mathbb{F}_p)$,
    或者包含在一个Borel子群中,
    或者包含在一个Cartan子群的正规化子中.\label{subgroup_class}
\end{cprop}

证明需要用到一个经典的几何定理
\begin{cprop}
    设$k$是一个域,$H$是$\mathrm{PGL}_2(k)$的有限子群,且$\absn{H}$和$\mathrm{char}\ k$互素.
    如果$H$既不是循环群,也不是二面体群,则$H$同构于$S_4,A_4$或者$A_5$.
\end{cprop}

证明大致是,如果$\mathrm{char}\ k>0$,则将$k$实现为一个局部环$A$的剩余类域,
利用$\absn{H}$与$\mathrm{char}$互素将$H$提升为$\mathrm{PGL}_2(A)$的子群,
从而化归到$\mathrm{char}\ k=0$的情况.

而当$\mathrm{char}\ k=0$时,取一个$\mathbb{Q}$的包含$H$中所有坐标的有限生成扩张$L$,
将$L$嵌入到$\mathbb{C}$中,从而化归到$k=\mathbb{C}$的情况.

最后,$\mathrm{SO}_3(\mathbb{R})\subset \mathrm{PGL}_2(\mathbb{C})$是极大紧子群,
从而$H$可以共轭到$\mathrm{SO}_3(\mathbb{R})$中. 再利用$\mathbb{R}^3$中有限旋转群的分类.

\begin{proof}[命题\ref{subgroup_class}的证明]
    如果$G$的阶被$p$整除,那么$G$要么包含$\mathrm{SL}_2(\mathbb{F}_p)$,要么包含在一个Borel子群中.
    但是$G$包含了一个Cartan子群,因此$\det G = \mathbb{F}_{p}^{\times}$.
    即$G$一旦包含$\mathrm{SL}_2(\mathbb{F}_p)$,那就一定等于整个$\mathrm{GL}_2(\mathbb{F}_p)$.

    反之,如果$G$的阶不被$p$整除. 令$H$是$G\mathbb{F}_{p}^{\times}$在$\mathrm{PGL}_2(\mathbb{F}_p)$中的投影.
    由于$H$包含一个$p\pm 1$阶元素,而$p\pm 1 > 5$,$H$不可能是$S_4,A_4,A_5$中的一个.
    如果$H$是循环群,则$H$包含在一个$p\pm 1$阶循环群中,因此包含在某个Cartan子群的像;那么$G$包含在一个Cartan子群中;
    如果$H$是二面体群,那么它包含一个指数为$2$的循环群,此时$G$包含在某个Cartan子群的正规化子中.
\end{proof}
